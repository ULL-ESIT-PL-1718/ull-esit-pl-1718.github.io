\sectionpractica{Conversor de Temperaturas}
\label{sectionpractica:conversordetemperaturas}

\parrafo{Donde}
\label{parrafo:dondetemperatura}
Véase 
\htmladdnormallink{https://bitbucket.org/casiano/pl-grado-temperature-converter/src}{https://bitbucket.org/casiano/pl-grado-temperature-converter/src}.
Este repo en bitbucket es privado del profesor.
El de 
\htmladdnormallink{GitHub }{https://github.com/crguezl/ull-etsii-grado-pl-1213-temperature-converter/tree/master}
es público pero no está completo.
\begin{verbatim}
~/local/src/javascript/PLgrado/temperature(master)]$ git remote -v
github  git@github.com:crguezl/ull-etsii-grado-pl-1213-temperature-converter.git (fetch)
github  git@github.com:crguezl/ull-etsii-grado-pl-1213-temperature-converter.git (push)
origin  ssh://git@bitbucket.org/casiano/pl-grado-temperature-converter.git (fetch)
origin  ssh://git@bitbucket.org/casiano/pl-grado-temperature-converter.git (push)
\end{verbatim}

Hay varias ramas (2015):

\begin{verbatim}
[~/local/src/javascript/PLgrado/temperature(master)]$ git branch -a
  gh-pages
  html5pattern
  karma
* master
  remotes/github/gh-pages
  remotes/github/master
  remotes/origin/html5pattern
  remotes/origin/karma
  remotes/origin/master
[~/local/src/javascript/PLgrado/temperature(master)]$
\end{verbatim}
\begin{itemize}
\item
En la rama \verb|master| está la versión mas simple.
\item
En la rama \verb|html5pattern| se muestra como usar el atributo \verb|pattern| (HTML5) 
en el tag \verb|input|.

En 29/09/2015 está disponible en el 
\htmladdnormallink{remoto github}{https://github.com/crguezl/ull-etsii-grado-pl-1213-temperature-converter/tree/html5pattern}.

Véase también
\htmladdnormallink{W3Schools}{http://www.w3schools.com/tags/att_input_pattern.asp}.
\item
Las pruebas están en el directorio \verb|tests/| en la rama master  que hay en GitHub 
\item
En la rama \verb|karma| (no visible al alumno, no está en GitHub en 2015) se encuentra como usar Karma para la ejecución de las pruebas.
En una práctica posterior se introduce Karma.
\end{itemize}


En mi portátil (29/09/2015) un clon del repo se encuentra en:
\begin{verbatim}
[~/srcPLgrado/temperature(master)]$ pwd -P
/Users/casiano/local/src/javascript/PLgrado/temperature # 27/01/2014
\end{verbatim}

\parrafo{index.html}

    \begin{verbatim}
<html>
  <head>
     <meta http-equiv="Content-Type" content="text/html; charset=UTF-8">
     <title>JavaScript Temperature Converter</title>
     <link href=normalize.css" rel="stylesheet" type="text/css">
     <link href="global.css" rel="stylesheet" type="text/css">

     <script type="text/javascript" src="temperature.js"></script>
  </head>
  <body>
    <h1>Temperature Converter</h1>
    <table>
      <tr>
        <th>Enter  Temperature (examples: 32F, 45C, -2.5f):</th>
        <td><input id="original" autofocus onchange="calculate();" placeholder="32F" size="50"></td>
      </tr>
      <tr>
        <th>Converted Temperature:</th>
        <td><span class="output" id="converted"></span></td>
      </tr>
    </table>
  </body>
</html>
    \end{verbatim}

\parrafo{Instale Emmet}

  Escribir HTML es farragoso. Una solución es usar algún plugin para su editor favorito.

Emmet existe para diversos editores, entre ellos para 
\begin{itemize}
\item
  En el caso de vim podemos usar 
\htmladdnormallink{Emmet-vim}{https://raw.githubusercontent.com/mattn/emmet-vim/master/TUTORIAL}
decargandolo desde 
\htmladdnormallink{http://www.vim.org/}{http://www.vim.org/scripts/script.php?script_id=2981}.

\item
Para \htmladdnormallink{Atom}{https://github.com/atom/atom}:
podemos usar
\htmladdnormallink{Plugin Emmet para Atom en GitHub}{https://github.com/emmetio/emmet-atom}
\begin{itemize}
\item
\htmladdnormallink{cheat-sheet}{http://sweetme.at/2014/03/10/atom-editor-cheat-sheet/} de Atom
\item
Véase el artículo \htmladdnormallink{Recommended GitHub Atom Packages for Web Developers}{http://www.elijahmanor.com/github-atom-packages/}.
\end{itemize}
\item
En
cloud9 
(\htmladdnormallink{c9.io}{https://c9.io/})
el plugin ya viene instalado
\end{itemize}

\begin{itemize}
\item
\htmladdnormallink{Documentación de Emmet}{http://docs.emmet.io/}
\item
\htmladdnormallink{Emmet cheat sheet}{http://docs.emmet.io/cheat-sheet/}
\end{itemize}

\parrafo{input tag}
\begin{itemize}
\item
The 
\htmladdnormallink{input}{http://www.w3schools.com/tags/tag_input.asp}
tag specifies an input field where the user can enter data.

\item
\verb|<input>| elements are used within a \verb|<form>| element to declare input controls that allow users to input data.

\item
An input field can vary in many ways, depending on the \verb|type| attribute.

\item
The \verb|type| attribute specifies the type of \verb|<input>| element to display.
The default type is \verb|text|.

Other values are:
\begin{itemize}
\item button
\item checkbox
\item color
\item date 
\item datetime 
\item datetime-local 
\item email 
\item file
\item hidden
\item image
\item month 
\item number 
\item password
\item radio
\item range 
\item reset
\item search
\item submit
\item tel
\item text
\item time 
\item url
\item week
\end{itemize}
\item
The elements used to create controls generally appear inside a \verb|<form>| 
element, but may also appear outside of a \verb|<form>| element declaration.
\end{itemize}

\parrafo{onchange}
The \verb|onchange| event occurs when the value of an element has been changed.

\parrafo{link tag}
\begin{itemize}
\item
The \verb|<link>| tag defines a link between a document and an external resource.
\begin{verbatim}
     <link href="global.css" rel="stylesheet" type="text/css">
\end{verbatim}
\item
The \verb|rel| attribute is required. It specifies the relationship between the current document and the linked document
\item
The \verb|<link>| tag is used to link to external CSS style sheets.
\begin{verbatim}
 <link href="global.css" rel="stylesheet" type="text/css">
\end{verbatim}
\end{itemize}

\parrafo{CSS}

\cei{CSS} stands for \cei{Cascading Style Sheets} and is a separate, but complementary, language to HTML. CSS is what we use to apply styles to the content on our web page.

\parrafo{global.css}


    \begin{verbatim}
[~/srcPLgrado/temperature(master)]$ cat global.css 
th, td      { vertical-align: top; text-align: right; font-size:large; }     /* Don't center table cells  */
#converted  { color: red; font-weight: bold; font-size:large;          }     /* Calculated values in bold */
input       { 
              text-align: right;       /* Align input to the right  */
              border: none; 
              border-radius: 20px 20px 20px 20px;
              padding: 5px  5px;
              font-size:large;       }
body
{
 background-color:#b0c4de;  /* blue */
 font-size:large;
 font-family: "Lucida Sans Typewriter", "Lucida Console", Monaco, "Bitstream Vera Sans Mono", monospace;
}

h1 {
    font-weight: normal;
    font-family: "Brush Script MT", cursive;
    background: #3C5681;
    padding: 5px 15px;
    color: white;
    display:inline-block;
    border-radius: 10px 10px 10px 10px;
}
    \end{verbatim}
  
\parrafo{Sintáxis CSS}

\begin{verbatim}
th, td      { vertical-align: top; text-align: right; font-size:large; }     /* Don't center table cells  */
\end{verbatim}
What you see above is referred to as a \cei{rule set}. 

\begin{itemize}
\item
Notice the curly braces. Also, notice that each declaration inside the curly braces has a semicolon. Everything
inside the curly braces is called a \cei{declaration block}.
\item
The portion prior to the first curly brace is what defines which part of the web page we are styling. This is referred to as the \cei{selector}.

Here we're using commas to separate our selectors \verb|th| and \verb|td|. 
This is a useful method to use to combine multiple selectors in a single rule set. 
In this case, the styles will apply to all \verb|<th>| and \verb|<td>| elements,
\item
Each of the three declarations in the declaration block is referred to as a \cei{declaration}. 
\item
Additionally, each declaration consists of a \cei{property} (the part before the colon)
and a \cei{value} (the part after the colon). 
\item
Each CSS declaration ends with a semicolon.
\end{itemize}

\parrafo{Introducción a los Selectores CSS}
\begin{itemize}
\item
A selector of \verb|nav| would match all HTML \verb|<nav>| elements, and a selector of \verb|ul| would match all HTML unordered lists, or \verb|<ul>| elements.
\item
An ID selector is declared using a hash, or pound symbol (\verb|#|) 
preceding a string of characters. 
The string of characters is defined by the developer. 
This selector matches any HTML element that has an
\verb1ID1 attribute with the same value as that of the selector, but minus the hash symbol.

The rule:
\begin{verbatim}
#converted  { color: red; font-weight: bold; font-size:large; } /* Calculated values in bold */
\end{verbatim}
applies to:
\begin{verbatim}
<span class="output" id="converted">
\end{verbatim}
\item
An ID element on a web page should be unique. 
\end{itemize}

\begin{exercise}
Usa 
\htmladdnormallink{jsfiddle.net}{https://jsfiddle.net}
para encontrar las respuestas a las preguntas. 
\begin{itemize}
\item \cei{Descendant Selector}:

Dado el selector
\begin{verbatim}
#container .box {
   float: left;
   padding-bottom: 15px;
}
\end{verbatim}
¿A que elementos de este HTML se aplica?
\begin{verbatim}
<div id="container">
  <div class="box"></div>
  <div class="box-2"></div>
</div>

<div class="box"></div>
\end{verbatim}
This declaration block will apply to all elements that have a class of box that are inside an element with an \verb|ID| of container. It's worth noting that the \verb|.box|
element doesn't have to be an immediate
child: there could be another element wrapping \verb|.box|, 
and the styles would still apply.
\item \cei{Child selector} (targets immediate child elements):

Dada esta regla
\begin{verbatim}
#container > .box {
   float: left;
   padding-bottom: 15px;
}
\end{verbatim}
In this example, the selector will match all elements that have a class of 
\verb|box| and that are immediate children of the 
\verb|#container| element. 
That means, unlike the descendant combinator, there can't be
another element wrapping \verb|.box|: 
it has to be a direct child element.


¿A que elementos de este HTML se aplica?
\begin{verbatim}
<div id="container">
  <div class="box"></div>
  <div>
    <div class="box"></div>
  </div>
</div>
\end{verbatim}

In this example, the CSS from the previous code example will apply only to the first 
\verb|<div>| element that has a class of \verb|box|. 

As you can see, the second \verb|<div>|
element with a class of box is inside
another \verb|<div>| element. 
As a result, the styles will not apply to that element, even though it too has a class of 
\verb|box|.

\item A general \cei{sibling combinator} matches elements based on sibling (hermanos)
relationships. That is to say, the selected elements are beside each other in the HTML.

Dada esta regla % hermanos: a los <p> que son hermanos de h2
\begin{verbatim}
h2 ~ p { margin-bottom: 20px; }
\end{verbatim}

This type of selector is declared using the tilde character (\verb|~|). 
In this example, all paragraph elements (\verb|<p>|) 
will be styled with the specified rules, but only if they are siblings of 
\verb|<h2>| elements.
There could be other elements in between the \verb|<h2>| and 
\verb|<p>|, and the styles would still apply.

¿A que elementos de este HTML se aplica?
\begin{verbatim}
<h2>Title</h2>
<p>Paragraph example 1.</p>
<p>Paragraph example 2.</p>
<p>Paragraph example 3.</p>
<div class="box">
  <p>Paragraph example 4.</p>
</div>
\end{verbatim}
\item 
The \cei{adjacent sibling combinator} uses the plus symbol (\verb|+|), 
and is almost the same as the general sibling selector. 
The difference is that the targeted element must be an immediate sibling, not just a general sibling. 

Dada esta regla 
% hermanos adyacentes: a los p que son hermanos adyacentes de un p
\begin{verbatim}
p+ p {
text-indent: 1.5em; margin-bottom: 0;
}
\end{verbatim}
This example will apply the specified styles
only to paragraph elements that immediately follow other paragraph elements.
the first paragraph element on a page would not receive these styles.
Also, if another element appeared between two paragraphs, the second paragraph of the two wouldn't have the styles applied.

¿A que elementos de este HTML se aplica?
\begin{verbatim}
<h2>Title</h2>
<p>Paragraph example 1.</p>
<p>Paragraph example 2.</p>
<p>Paragraph example 3.</p>
<div class="box">
  <p>Paragraph example 4.</p>
  <p>Paragraph example 5.</p>
</div>
\end{verbatim}
% ...the styles will apply only to the second, third, and fifth paragraphs in this section of HTML.
\item The \cei{attribute selector} 
targets elements based on the presence and/or value of HTML attributes, and is declared using square brackets.

There should not be a space before the opening square bracket unless you intend to use it along with a {\it descendant combinator}. 


Dada esta regla:
\begin{verbatim}
input[type="text"] {
   background-color: #444;
   width: 200px;
}
\end{verbatim}
The attribute selector targets elements based on the presence and/or value of HTML attributes, and is declared using square brackets.

¿A que elementos de este HTML se aplica?
\begin{verbatim}
  <input type="text">
  <input type="submit">
\end{verbatim}
\item A \cei{pseudo-class} 
uses a colon character to identify a pseudo-state that an element might be in.


Dada esta regla:
\begin{verbatim}
a:hover {
   color: red;
}
\end{verbatim}
¿Que porción del selector es conocido como \cei{pseudo-clase}?
¿Cuando se aplica la regla a un ancla \verb|<a>|?

In this case, the pseudo-class portion of the selector is the \verb|:hover| part. 
Here we've attached this pseudo-class to all anchor elements 
(\verb|<a>| elements). 
This means that when the user hovers their mouse
over an \verb|<a>| element, the color property for that element will change to red. 

This type of pseudo-class is a \cei{dynamic pseudo-class}, 
because it occurs only in response to user interaction—in this case,
the mouse moving over the targeted element.

It's important to recognize that these types of selectors do not just select elements; 
\emph{they select elements that are in a particular state}. 

% It's important to recognize that these types of selectors do not just select elements; they select elements that are in a particular state
\item 

Exprese con palabras a que elementos del documento se aplicará la siguiente regla:
\begin{verbatim}
#form [type=text] { border: solid 1px #ccc; }
\end{verbatim}

This selector combines the \verb|ID| selector with the attribute selector. 

This will target all elements with a type attribute of \verb|text| that are inside an element with an \verb|ID| of 
\verb|form|.

\end{itemize}
\end{exercise}

\begin{exercise}
\begin{itemize}
\item
Supuesto que una hoja de estilo contiene estas reglas:
\begin{verbatim}
p { font-size: 20px; }
p { font-size: 30px; } 
\end{verbatim}
¿Cual será el tamaño de font que se aplique a los elementos párrafo?

\red{Selectors targeting styles later in a CSS document have precedence over the same selectors that appear earlier in the CSS file}. 
\item
Supuesto que una hoja de estilo contiene estas reglas:
\begin{verbatim}
div p { color: blue; }
p{ color: red; }
\end{verbatim}
¿Cual será el color que se aplique a los elementos párrafo?
% the descendant selector takes precedence over the element type selector

In this instance, the color value for paragraph elements inside of \verb|<div>|
elements will be blue, despite the fact that the second color declaration
appears later in the document. \red{So although the browser
does give some importance to the order of these rule sets, that order
is superseded by the} {\bf specificity} of the first rule set.
\item
Supuesto que una hoja de estilo contiene estas reglas:
\begin{verbatim}
#main {
   color: green;
}
body div.container {
   color: pink;
}
\end{verbatim}
¿Cual será el color que se aplique a este elemento \verb|<div>|?
\begin{verbatim}
<div id="main" class="container"></div>
\end{verbatim}


\red{The ID selector has very high specificity}
and thus takes precedence over the second rule set.
\end{itemize}
\end{exercise}

\parrafo{CSS reset}
Every browser applies certain styles to elements on a web page by default. 

For example, if you use an un-ordered list (the \verb|<ul>| element) the browser will display the list with some existing
formatting styles, including bullets next to the individual list items (the \verb|<li>| elements inside the \verb|<ul>|). 

By using a \cei{CSS reset document} at the top of your CSS file, you can reset all these styles
to a bare minimum. 


Two of the most popular CSS resets are 
\htmladdnormallink{Eric Meyer's Reset }{http://meyerweb.com/eric/tools/css/reset/}
and 
\htmladdnormallink{Nicolas Gallagher's Normalize.css}{http://necolas.github.io/normalize.css/}

\begin{verbatim}
    <title>JavaScript Temperature Converter</title>
    <link href=normalize.css" rel="stylesheet" type="text/css">
    <link href="global.css" rel="stylesheet" type="text/css">
\end{verbatim}

\parrafo{El Modelo de Caja}

The \cei{box model} refers to the usually invisible rectangular area that is created for each HTML element. This area has four basic components

\begin{rawhtml}
<img src="cssboxmodel.png" align="middle" />
\end{rawhtml}


\begin{itemize}
\item
\cei{Content} 

The content portion of the box model holds the actual content. 

The content can be text, images, or
whatever else is visible on a web page.

\item
\cei{Padding}

The \verb|padding| of an element is defined using the \verb|padding| property. The \verb|padding| is the space around the content. 

It can be defined for an individual side 
(for example, \verb|padding-left: 20px|) or
for all four sides in one declaration
\verb|padding: 20px 10px 30px 20px|, for instance. 

When declaring all four sides, you’re using a \cei{shorthand property}. 

Often when a CSS property takes multiple values like this, \blue{they start at the top and go clockwise in relation to the element}. 
So, in the example just cited, this would apply \verb|20px| of padding to the
top, \verb|10px| to the right, \verb|30px| to the bottom, and \verb|20px| to the left.

  See \htmladdnormallink{padding examples at w3schools}{http://www.w3schools.com/css/css_padding.asp}.
\item
\cei{Border} 

The border of an element is defined using the border property. 

This is a \red{shorthand property} that defines the element's \verb|border-width|, \verb|border-style|, and \verb|border-color|. For example, 
\begin{verbatim}
border: 4px dashed orange.
\end{verbatim}

\item
\cei{Margin} 

Margins are similar to padding, and are defined using similar syntax 

\begin{verbatim}
margin-left: 15px
\end{verbatim}
or
\begin{verbatim}
 margin: 10px 20px 10px 20px
\end{verbatim}
The margin portion of an element exists outside the element. 

\blue{A margin creates space between the targeted element and surrounding elements}.
\end{itemize}

\begin{exercise}
\begin{itemize}
\item Dadas las declaraciones:
\begin{verbatim}
.example {
  border-style: dashed;
  border-width: 2px;
  border-color: blue;
}
.example {
  border: solid;
  color: green;
}
\end{verbatim}
¿De que color queda el borde de los elementos de clase \verb|example|? 
¿Que \verb|border-style| tendrán?
¿Que \verb|border-width| tendrán?

Here we’ve used the same selector on two different rule sets. 

The second rule set will take precedence over the first, overriding any styles that are the same in both rule sets.

In the first rule set, we’ve defined all three \verb|border-related| properties in longhand, setting the values to display a dashed border that’s 2px wide and colored blue. 

But what’s the result of these two
rule sets? Well, the border will become 3px wide 
(the default border width for a visible border,) and it'll be colored green, not blue. 

This happens because the second rule set uses shorthand to
define the \verb|border-style| as solid, \emph{but doesn’t define the other two properties} (\verb|border-width| and 
\verb|border-color|).

See
\begin{itemize}
\item
\htmladdnormallink{jsfiddle}{https://jsfiddle.net/casiano/u3nrjssa/}
\item
\htmladdnormallink{gist}{https://gist.github.com/crguezl/40c6b3fd6b95cc3d5f97}
\end{itemize}

\item Dada la declaración:
\begin{verbatim}
.example {
  margin: 10px 20px;
}
\end{verbatim}
¿De que tamaño quedarán \verb|margin-top| \verb|margin-right|, \verb|margin-bottom| y \verb|margin-left| para los elementos de clase \verb|example|?

Another thing to understand about shorthand is that for certain shorthand properties, the missing values are inherited based on the existing values.

We’re omitting the bottom and left, so they'll inherit from the top and right values.

\end{itemize}
\end{exercise}

\parrafo{Editing CSS styles in Chrome using various DevTools aid}

Depurar estilos puede ser complicado. Lea el artículo
\htmladdnormallink{Tips for Debugging HTML and CSS}{http://blog.teamtreehouse.com/tips-debugging-html-css}.

While you can not "debug" CSS, because it is not a scripting language,
you can utilize the Chrome DevTools Elements panel to inspect an element and
view the Styles pane on the right. 

This will give you
insights as to the styles being overridden or ignored (line threw). 

The Styles pane is also useful because of it's ability to LiveEdit the document being inspected, which may help you iron out the
issues. 

If the styles are being overridden, you can then view the Computed Style pane to see the CSS that is actually being utilized to style your document.

\begin{itemize}
\item
\htmladdnormallink{Editing styles}{https://developer.chrome.com/devtools/docs/elements-styles}
in Chrome developer pages.
\item
\htmladdnormallink{stackoverflow: Debugging CSS in Google Chrome}{http://stackoverflow.com/questions/11065229/debugging-css-in-google-chrome}
\item
\htmladdnormallink{Chrome DevTools for CSS - Better CSS Coding and CSS Debugging with Developer Tools}{http://youtu.be/Z3HGJsNLQ1E} en YouTube by LearnCode.academy
\end{itemize}

\parrafo{Block versus Inline}

HTML elements fall under two categories: \cei{block} or \cei{inline}.


\begin{itemize}
\item
Block-level elements include elements like \verb|<div>|, \verb|<p>|, \verb|h1|, \verb|li|
and \verb|<section>|.
A block-level element is more of a structural, layout related element.

\red{A block element is an element that takes up the full width available, and has a line break before and after it}.
\item
\blue{An inline element only takes up as much width as necessary, and does not force line breaks}.
Inline elements behave like words and letters within of a paragraph.

Inline elements include \verb|<span>|, \verb|<b>|, and \verb|<em>|.

It's worth noting that inline elements are subject to CSS properties that affect text. 

For example, \verb|line-height| and \verb|letter-spacing| are CSS properties that can be used to style inline elements.

However, those same properties wouldn't affect block elements.

\item
See 
\begin{itemize}
\item
\htmladdnormallink{CSS display Property}{http://www.w3schools.com/cssref/pr_class_display.asp}
\item
\htmladdnormallink{CSS Display - Block and Inline Elements at w3schools}{http://www.w3schools.com/css/css_display_visibility.asp}
\end{itemize}
\end{itemize}

\parrafo{Propiedades CSS}
\begin{itemize}
\item
The 
\htmladdnormallink{vertical-align}{http://www.w3schools.com/cssref/pr_pos_vertical-align.asp}
property sets the vertical alignment of an element.
\item
The 
\htmladdnormallink{text-align}{http://www.w3schools.com/cssref/pr_text_text-align.asp}
property specifies the horizontal alignment of text in an element.
\item
The 
\htmladdnormallink{font-family}{http://www.w3schools.com/cssref/pr_font_font-family.asp}
property specifies the font for an element.

The font-family property can hold several font names as a "fallback" system. 
If the browser does not support the first font, it tries the next font.
\end{itemize}

\parrafo{temperature.js}


  \begin{latexonly}
    \begin{verbatim}

"use strict"; // Use ECMAScript 5 strict mode in browsers that support it
function calculate() {
  var result;
  var original       = document.getElementById("........");
  var temp = original.value;
  var regexp = /.............................../;
  
  var m = temp.match(......);
  
  if (m) {
    var num = ....;
    var type = ....;
    num = parseFloat(num);
    if (type == 'c' || type == 'C') {
      result = (num * 9/5)+32;
      result = ..............................
    }
    else {
      result = (num - 32)*5/9;
      result = ............................
    }
    converted.innerHTML = result;
  }
  else {
    converted.innerHTML = "ERROR! Try something like '-4.2C' instead";
  }
}

    \end{verbatim}
  \end{latexonly}
    \begin{rawhtml}
    <pre>
<span class="s2">&quot;use strict&quot;</span><span class="p">;</span> <span class="c1">// Use ECMAScript 5 strict mode in browsers that support it</span>
<span class="kd">function</span> <span class="nx">calculate</span><span class="p">()</span> <span class="p">{</span>
  <span class="kd">var</span> <span class="nx">result</span><span class="p">;</span>
  <span class="kd">var</span> <span class="nx">original</span>       <span class="o">=</span> <span class="nb">document</span><span class="p">.</span><span class="nx">getElementById</span><span class="p">(</span><span class="s2">&quot;........&quot;</span><span class="p">);</span>
  <span class="kd">var</span> <span class="nx">temp</span> <span class="o">=</span> <span class="nx">original</span><span class="p">.</span><span class="nx">value</span><span class="p">;</span>
  <span class="kd">var</span> <span class="nx">regexp</span> <span class="o">=</span> <span class="sr">/.............................../</span><span class="p">;</span>
  
  <span class="kd">var</span> <span class="nx">m</span> <span class="o">=</span> <span class="nx">temp</span><span class="p">.</span><span class="nx">match</span><span class="p">(......);</span>
  
  <span class="k">if</span> <span class="p">(</span><span class="nx">m</span><span class="p">)</span> <span class="p">{</span>
    <span class="kd">var</span> <span class="nx">num</span> <span class="o">=</span> <span class="p">....;</span>
    <span class="kd">var</span> <span class="nx">type</span> <span class="o">=</span> <span class="p">....;</span>
    <span class="nx">num</span> <span class="o">=</span> <span class="nb">parseFloat</span><span class="p">(</span><span class="nx">num</span><span class="p">);</span>
    <span class="k">if</span> <span class="p">(</span><span class="nx">type</span> <span class="o">==</span> <span class="s1">&#39;c&#39;</span> <span class="o">||</span> <span class="nx">type</span> <span class="o">==</span> <span class="s1">&#39;C&#39;</span><span class="p">)</span> <span class="p">{</span>
      <span class="nx">result</span> <span class="o">=</span> <span class="p">(</span><span class="nx">num</span> <span class="o">*</span> <span class="mi">9</span><span class="o">/</span><span class="mi">5</span><span class="p">)</span><span class="o">+</span><span class="mi">32</span><span class="p">;</span>
      <span class="nx">result</span> <span class="o">=</span> <span class="p">..............................</span>
    <span class="p">}</span>
    <span class="k">else</span> <span class="p">{</span>
      <span class="nx">result</span> <span class="o">=</span> <span class="p">(</span><span class="nx">num</span> <span class="o">-</span> <span class="mi">32</span><span class="p">)</span><span class="o">*</span><span class="mi">5</span><span class="o">/</span><span class="mi">9</span><span class="p">;</span>
      <span class="nx">result</span> <span class="o">=</span> <span class="p">............................</span>
    <span class="p">}</span>
    <span class="nx">converted</span><span class="p">.</span><span class="nx">innerHTML</span> <span class="o">=</span> <span class="nx">result</span><span class="p">;</span>
  <span class="p">}</span>
  <span class="k">else</span> <span class="p">{</span>
    <span class="nx">converted</span><span class="p">.</span><span class="nx">innerHTML</span> <span class="o">=</span> <span class="s2">&quot;ERROR! Try something like &#39;-4.2C&#39; instead&quot;</span><span class="p">;</span>
  <span class="p">}</span>
<span class="p">}</span>
    </pre>
    \end{rawhtml}
  
\parrafo{Despliegue}
\begin{itemize}
\item
Deberá desplegar la aplicación en GitHub Pages como página de proyecto.
Vea la sección {\it GitHub Project Pages} \ref{subsection:githubprojectpages}.
\end{itemize}

\parrafo{Creando un fichero package.json}

Para saber todo sobre 
\htmladdnormallink{ipackage.json}{https://docs.npmjs.com/files/package.json}
visite este manual de npm o bien escriba \verb|npm help json| en la línea de comandos.

The command:

\begin{verbatim}
npm init [-f|--force|-y|--yes]
\end{verbatim}

Will ask you a bunch of questions, and then write a \verb|package.json| for you.

If you already have a \verb|package.json| file, it'll read that first, and default to the options in there.

It is strictly additive, so it does not delete options from your \verb|package.json|
without a really good reason to do so.

If you invoke it with \verb|-f|, \verb|--force|, it will use only defaults and not prompt you for any options.

\begin{verbatim}
[/tmp/pl-grado-temperature-converter(karma)]$ npm init
This utility will walk you through creating a package.json file.
It only covers the most common items, and tries to guess sane defaults.

See `npm help json` for definitive documentation on these fields
and exactly what they do.

Use `npm install <pkg> --save` afterwards to install a package and
save it as a dependency in the package.json file.

Press ^C at any time to quit.
name: (pl-grado-temperature-converter) 
version: (0.0.0) 0.0.1
description: ULL ESIT Grado de Informática. 3º. PL. Lab "Temperature Converter"
entry point: (temperature.js) 
test command: open tests/index.html
git repository: (ssh://git@bitbucket.org/casiano/pl-grado-temperature-converter.git) 
keywords: regexp
author: Casiano
license: (ISC) 
About to write to /private/tmp/pl-grado-temperature-converter/package.json:

{
  "name": "pl-grado-temperature-converter",
  "version": "0.0.1",
  "description": "ULL ESIT Grado de Informática. 3º. PL. Lab \"Temperature Converter\"",
  "main": "temperature.js",
  "directories": {
    "test": "tests"
  },
  "scripts": {
    "test": "open tests/index.html"
  },
  "repository": {
    "type": "git",
    "url": "ssh://git@bitbucket.org/casiano/pl-grado-temperature-converter.git"
  },
  "keywords": [
    "regexp"
  ],
  "author": "Casiano",
  "license": "ISC"
}


Is this ok? (yes) y
\end{verbatim}

Esto genera el fichero \verb|package.json|:
\begin{verbatim}
[/tmp/pl-grado-temperature-converter(karma)]$ ls -ltr | tail -1
-rw-r--r--  1 casiano  wheel   487  5 feb 18:22 package.json
\end{verbatim}
Si ahora escribo:
\begin{verbatim}
[/tmp/pl-grado-temperature-converter(karma)]$ npm test

> pl-grado-temperature-converter@0.0.1 test /private/tmp/pl-grado-temperature-converter
> open tests/index.html
\end{verbatim}
Ejecutamos las pruebas en el navegador (en Mac OS X) supuesto que ya estuvieran escritas.




%\parrafo{Pruebas: Mocha y Chai}
\parrafo{Pruebas: Mocha y Chai}
\label{parrafo:mochaychai}
Mocha is a test framework while Chai is an expectation one. 

Mocha is the \red{simple, flexible, and fun} JavaScript unit-testing framework
that runs in Node.js or in the browser. 

It is open source (MIT licensed),
and we can learn more about it at
\htmladdnormallink{https://github.com/mochajs/mocha}{https://github.com/mochajs/mocha}

Let's say
Mocha setups and describes test suites and Chai provides convenient
helpers to perform all kinds of assertions against your JavaScript code.

\parrafo{Pruebas: Estructura}

Podemos instalar \verb|mocha| globalmente:
\begin{verbatim}
$ npm install -g mocha
\end{verbatim}
pero podemos también añadirlo en \verb|package.json| como una \verb|devDependencies|:
\begin{verbatim}
[/tmp/pl-grado-temperature-converter(karma)]$ head -n 5 package.json 
{
  "dependencies": {},
  "devDependencies": {
    "mocha": "latest"
  },
\end{verbatim}

Y ahora podemos instalar todas las dependencias usando  \verb|npm install|:
\begin{verbatim}
$ npm install
npm http GET https://registry.npmjs.org/mocha
npm http 200 https://registry.npmjs.org/mocha
npm http GET https://registry.npmjs.org/commander/2.3.0
...
\end{verbatim}

En este caso \verb|mocha| es instalado localmente, no globalmente:
\begin{verbatim}
[/tmp/pl-grado-temperature-converter(karma)]$ ls -ltr node_modules/
total 0
drwxr-xr-x  12 casiano  staff  408  5 feb 18:40 mocha
\end{verbatim}

Una vez instalado Mocha, creamos la estructura para las pruebas:

\begin{verbatim}
$ mocha init tests
\end{verbatim}
esto en el caso de que lo hayamos instalado globalmente o bien
\begin{verbatim}
$ node_modules/mocha/bin/mocha init tests
\end{verbatim}
si lo hemos instalado localmente.

\begin{verbatim}
$ tree tests
tests
|-- index.html
|-- mocha.css
|-- mocha.js
`-- tests.js
\end{verbatim}

Añadimos \verb|chai.js|
(Véase 
\htmladdnormallink{http://chaijs.com/guide/installation/}{http://chaijs.com/guide/installation/}) al directorio \verb|tests|.

Chai is a platform-agnostic BDD/TDD assertion library featuring several interfaces 
(for example, should, expect, and assert). 
It is open source (MIT licensed), and we can learn more about it at
\htmladdnormallink{http://chaijs.com/}{http://chaijs.com/}

We can also install Chai on the command line using npm, as follows:
\begin{verbatim}
            npm install chai --save-dev
\end{verbatim}


The latest tagged version will be available for hot-linking at 
\htmladdnormallink{http://chaijs.com/chai.js}{http://chaijs.com/chai.js}.

If you prefer to host yourself, use the \verb|chai.js| file from the root of the 
\htmladdnormallink{github project at https://github.com/chaijs/chai}{https://github.com/chaijs/chai}. 
\begin{verbatim}
[/tmp/pl-grado-temperature-converter(karma)]$ 
$ curl https://raw.githubusercontent.com/chaijs/chai/master/chai.js -o tests/chai.js
  % Total    % Received % Xferd  Average Speed   Time    Time     Time  Current
                                 Dload  Upload   Total   Spent    Left  Speed
100  118k  100  118k    0     0  65521      0  0:00:01  0:00:01 --:--:-- 65500
\end{verbatim}
Ya tenemos nuestro fichero \verb|tests/chai.js|:
\begin{verbatim}
[/tmp/pl-grado-temperature-converter(karma)]$ head tests/chai.js 

;(function(){

/**
 * Require the module at `name`.
 *
 * @param {String} name
 * @return {Object} exports
 * @api public
 */
\end{verbatim}
Quedando el árbol como sigue:
\begin{verbatim}
[~/srcPLgrado/temperature(master)]$ tree tests/
tests/
|-- chai.js
|-- index.html
|-- mocha.css
|-- mocha.js
`-- tests.js

0 directories, 5 files
\end{verbatim}

\parrafo{Pruebas: {\tt index.html}}

Modificamos el fichero \verb|tests/index.html| que fué generado por \verb|mocha init|
para 
\begin{itemize}
\item
Cargar \verb|chai.js|
\item
Cargar \verb|temperature.js|
\item
Usar el estilo \verb|mocha.setup('tdd')|:
\item
Imitar la página \verb|index.html| con los correspondientes \verb|input| y 
\verb|span|:
\begin{verbatim}
    <input id="original" placeholder="32F" size="50">
    <span class="output" id="converted"></span>
\end{verbatim}
\end{itemize}
quedando así:
\begin{verbatim}
[~/srcPLgrado/temperature(master)]$ cat tests/index.html 
<!DOCTYPE html>
<html>
  <head>
    <title>Mocha</title>
    <meta http-equiv="Content-Type" content="text/html; charset=UTF-8">
    <meta name="viewport" content="width=device-width, initial-scale=1.0">
    <link rel="stylesheet" href="mocha.css" />
  </head>
  <body>
    <div id="mocha"></div>
    <input id="original" placeholder="32F" size="50">
    <span class="output" id="converted"></span>

    <script src="chai.js"></script>
    <script src="mocha.js"></script>
    <script src="../temperature.js"></script>
    <script>mocha.setup('tdd')</script>
    <script src="tests.js"></script>

    <script>
      mocha.run();
    </script>
  </body>
</html>
\end{verbatim}

\parrafo{Pruebas: Añadir los tests}

The "TDD" interface provides 
\begin{itemize}
\item \verb|suite()|
\item  \verb|test()|
\item  \verb|setup()|
\item  \verb|teardown()|.
\end{itemize}

\begin{verbatim}
[~/srcPLgrado/temperature(master)]$ cat tests/tests.js 
var assert = chai.assert;

suite('temperature', function() {
    test('32F = 0C', function() {
        original.value = "32F";
        calculate();
        assert.deepEqual(converted.innerHTML, "0.0 Celsius");
    });
    test('45C = 113.0 Farenheit', function() {
        original.value = "45C";
        calculate();
        assert.deepEqual(converted.innerHTML, "113.0 Farenheit");
    });
    test('5X = error', function() {
        original.value = "5X";
        calculate();
        assert.match(converted.innerHTML, /ERROR/);
    });
});
\end{verbatim}

The \cei{BDD} interface provides \verb|describe()|, \verb|it()|, \verb|before()|, \verb|after()|, \verb|beforeEach()|, and \verb|afterEach()|:

\begin{verbatim}
describe('Array', function(){
  before(function(){
    // ...
  });

  describe('#indexOf()', function(){
    it('should return -1 when not present', function(){
      [1,2,3].indexOf(4).should.equal(-1);
    });
  });
});
\end{verbatim}
The \cei{Chai should} style allows for the same chainable assertions as the 
\cei{expect interface}, however it extends each object with a \verb|should| 
property to start your chain. 

\parrafo{Chai Assert Style}

The \cei{assert style} is exposed through assert interface. 

This provides the classic assert-dot notation, similiar to that packaged with node.js. 

This \verb|assert| module, however, provides several additional tests and is browser compatible.

\begin{verbatim}
var assert = require('chai').assert
  , foo = 'bar'
  , beverages = { tea: [ 'chai', 'matcha', 'oolong' ] };

assert.typeOf(foo, 'string', 'foo is a string');
assert.equal(foo, 'bar', 'foo equal `bar`');
assert.lengthOf(foo, 3, 'foo`s value has a length of 3');
assert.lengthOf(beverages.tea, 3, 'beverages has 3 types of tea');
\end{verbatim}
In all cases, the assert style allows you to include an optional message as the last parameter in the assert statement. 

These will be included in the error messages should your assertion not pass.

\parrafo{Assert API, Expect/Should API}

\begin{itemize}
\item
Here  is the documentation of the 
\htmladdnormallink{Assert API}{http://chaijs.com/api/assert/}.


\item
Here  is the documentation of the 
\htmladdnormallink{Should/Expect API}{http://chaijs.com/api/bdd/}.
\end{itemize}

\parrafo{Chai Expect Style}

The BDD style is exposed through expect or should interfaces. In both scenarios, you chain together natural language assertions.

\begin{verbatim}
var expect = require('chai').expect
  , foo = 'bar'
  , beverages = { tea: [ 'chai', 'matcha', 'oolong' ] };

expect(foo).to.be.a('string');
expect(foo).to.equal('bar');
expect(foo).to.have.length(3);
expect(beverages).to.have.property('tea').with.length(3);
\end{verbatim}
Expect also allows you to include arbitrary messages to prepend to any failed assertions that might occur.

\begin{verbatim}
var answer = 43;
\end{verbatim}

\begin{verbatim}
// AssertionError: expected 43 to equal 42.
expect(answer).to.equal(42); 
\end{verbatim}

\begin{verbatim}
// AssertionError: topic [answer]: expected 43 to equal 42.
expect(answer, 'topic [answer]').to.equal(42);
\end{verbatim}
This comes in handy when being used with non-descript topics such as booleans or numbers.

\parrafo{Ejecución Simple}
Ahora podemos ejecutar las pruebas abriendo en el navegador el
fichero \verb|tests/index.html|:
\begin{verbatim}
$ open tests/index.html 
\end{verbatim}

Esta información aparece también en las secciones {\it Unit Testing: Mocha}
\ref{subsection:mocha} de
estos apuntes.




\parrafo{Manejando tareas en JS: Gulp}
\subsection{Preguntas de Repaso de
Gulp}\label{preguntas-de-repaso-de-gulp}

\begin{enumerate}
\def\labelenumi{\arabic{enumi}.}
\itemsep1pt\parskip0pt\parsep0pt
\item
  Complete las partes que faltan del siguiente \texttt{gulpfile.js} en
  el que se lleva a cabo una tarea para la optimización (uglify/minify)
  de la aplicación de la práctica de la temperatura:
\end{enumerate}

\begin{Shaded}
\begin{Highlighting}[]
\OtherTok{/tmp/pl}\NormalTok{-grado-temperature-}\FunctionTok{converter}\NormalTok{(karma)]$ cat }\OtherTok{gulpfile}\NormalTok{.}\FunctionTok{js}
\KeywordTok{var} \NormalTok{gulp    = }\FunctionTok{require}\NormalTok{(}\StringTok{'gulp'}\NormalTok{),}
    \NormalTok{gutil   = }\FunctionTok{require}\NormalTok{(}\StringTok{'gulp-util'}\NormalTok{),}
    \NormalTok{uglify  = }\FunctionTok{require}\NormalTok{(}\StringTok{'gulp-uglify'}\NormalTok{),}
    \NormalTok{concat  = }\FunctionTok{require}\NormalTok{(}\StringTok{'gulp-concat'}\NormalTok{);}
\KeywordTok{var} \NormalTok{minifyHTML = }\FunctionTok{require}\NormalTok{(}\StringTok{'gulp-minify-html'}\NormalTok{);}
\KeywordTok{var} \NormalTok{minifyCSS  = }\FunctionTok{require}\NormalTok{(}\StringTok{'gulp-minify-css'}\NormalTok{);}

\OtherTok{gulp}\NormalTok{.}\FunctionTok{____}\NormalTok{(}\StringTok{'minify'}\NormalTok{, }\KeywordTok{function} \NormalTok{() \{}
  \OtherTok{gulp}\NormalTok{.}\FunctionTok{___}\NormalTok{(}\StringTok{'temperature.js'}\NormalTok{)}
  \NormalTok{.}\FunctionTok{____}\NormalTok{(}\FunctionTok{uglify}\NormalTok{())}
  \NormalTok{.}\FunctionTok{___}\NormalTok{(}\OtherTok{gulp}\NormalTok{.}\FunctionTok{____}\NormalTok{(}\StringTok{'minified'}\NormalTok{));}

  \OtherTok{gulp}\NormalTok{.}\FunctionTok{__}\NormalTok{(}\StringTok{'./index.html'}\NormalTok{)}
    \NormalTok{.}\FunctionTok{___}\NormalTok{(}\FunctionTok{minifyHTML}\NormalTok{())}
    \NormalTok{.}\FunctionTok{___}\NormalTok{(}\OtherTok{gulp}\NormalTok{.}\FunctionTok{___}\NormalTok{(}\StringTok{'./minified/'}\NormalTok{))}

  \OtherTok{gulp}\NormalTok{.}\FunctionTok{__}\NormalTok{(}\StringTok{'./*.css'}\NormalTok{)}
   \NormalTok{.}\FunctionTok{___}\NormalTok{(}\FunctionTok{minifyCSS}\NormalTok{(\{}\DataTypeTok{keepBreaks}\NormalTok{:}\KeywordTok{true}\NormalTok{\}))}
   \NormalTok{.}\FunctionTok{___}\NormalTok{(}\OtherTok{gulp}\NormalTok{.}\FunctionTok{___}\NormalTok{(}\StringTok{'./minified/'}\NormalTok{))}
        \NormalTok{\});}
\end{Highlighting}
\end{Shaded}

\begin{itemize}
\itemsep1pt\parskip0pt\parsep0pt
\item
  Explique los pasos para publicar un libro GitBook en GitHub usando
  \texttt{gulp}
\item
  Explique los pasos para actualizar automáticamente los HTML del libro
  GitBook en su máquina virtual del iaas usando \texttt{gulp}
\end{itemize}



\parrafo{Pruebas: Véase}
\begin{itemize}
\item Mocha, Chai y Sinon
\begin{itemize}
\item
\htmladdnormallink{Testing your frontend JavaScript code using mocha, chai, and sinon}{https://nicolas.perriault.net/code/2013/testing-frontend-javascript-code-using-mocha-chai-and-sinon/} by Nicolas Perriault
\item
\htmladdnormallink{Get your Frontend JavaScript Code Covered}{https://nicolas.perriault.net/code/2013/get-your-frontend-javascript-code-covered/}
by Nicolas Perriault
\item
\htmladdnormallink{Github repo crguezl/mocha-chai-sinon--example}{://github.com/crguezl/mocha-chai-sinon--example}
with Nicolas examples
\item
Podemos encontrar un ejemplo de  unit testing en JavaScript en el browser con 
el testing framework Mocha y  Chai en el repositorio 
\htmladdnormallink{https://github.com/ludovicofischer/mocha-chai-browser-demo}{https://github.com/ludovicofischer/mocha-chai-browser-demo}:
\htmladdnormallink{An example setup for unit testing JavaScript in the browser with the Mocha testing framework and Chai assertions}{https://github.com/ludovicofischer/mocha-chai-browser-demo}
\item
\htmladdnormallink{Testing in Browsers and Node with Mocha, Chai, Sinon, and Testem
}{http://www.kenpowers.net/blog/testing-in-browsers-and-node/}
\end{itemize}
\item Gulp
\begin{itemize}
\item
\htmladdnormallink{The Front-end Tooling Book}{http://tooling.github.io/book-of-modern-frontend-tooling/build-systems/gulp/introduction.html}
\item
\htmladdnormallink{An Introduction to Gulp.js by Craig Buckler}{http://www.sitepoint.com/introduction-gulp-js/} SitePoint
\item
\htmladdnormallink{Gulp: the modern frontend factory}{http://david.nowinsky.net/gulp-book/}
\item
\htmladdnormallink{Building With Gulp by Callum Macrae}{http://www.smashingmagazine.com/2014/06/11/building-with-gulp/}
\end{itemize}
\item Karma
\begin{itemize}
\item
\htmladdnormallink{Introduction to Karma}{https://egghead.io/lessons/unit-testing-introduction-to-karma} Screencast.
\item
\htmladdnormallink{Vojta Jina: Testacular (now Karma) - JavaScript test runner}{http://youtu.be/MVw8N3hTfCI}. YouTube.
\end{itemize}
\item
\htmladdnormallink{PhantomJS}{http://phantomjs.org/}
is a headless WebKit scriptable with a JavaScript API. It has fast and native support for various web standards: DOM handling, CSS selector, JSON, Canvas, and SVG.
\end{itemize}

