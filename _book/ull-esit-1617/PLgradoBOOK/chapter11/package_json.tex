
Para saber todo sobre 
\htmladdnormallink{ipackage.json}{https://docs.npmjs.com/files/package.json}
visite este manual de npm o bien escriba \verb|npm help json| en la línea de comandos.

The command:

\begin{verbatim}
npm init [-f|--force|-y|--yes]
\end{verbatim}

Will ask you a bunch of questions, and then write a \verb|package.json| for you.

If you already have a \verb|package.json| file, it'll read that first, and default to the options in there.

It is strictly additive, so it does not delete options from your \verb|package.json|
without a really good reason to do so.

If you invoke it with \verb|-f|, \verb|--force|, it will use only defaults and not prompt you for any options.

\begin{verbatim}
[/tmp/pl-grado-temperature-converter(karma)]$ npm init
This utility will walk you through creating a package.json file.
It only covers the most common items, and tries to guess sane defaults.

See `npm help json` for definitive documentation on these fields
and exactly what they do.

Use `npm install <pkg> --save` afterwards to install a package and
save it as a dependency in the package.json file.

Press ^C at any time to quit.
name: (pl-grado-temperature-converter) 
version: (0.0.0) 0.0.1
description: ULL ESIT Grado de Informática. 3º. PL. Lab "Temperature Converter"
entry point: (temperature.js) 
test command: open tests/index.html
git repository: (ssh://git@bitbucket.org/casiano/pl-grado-temperature-converter.git) 
keywords: regexp
author: Casiano
license: (ISC) 
About to write to /private/tmp/pl-grado-temperature-converter/package.json:

{
  "name": "pl-grado-temperature-converter",
  "version": "0.0.1",
  "description": "ULL ESIT Grado de Informática. 3º. PL. Lab \"Temperature Converter\"",
  "main": "temperature.js",
  "directories": {
    "test": "tests"
  },
  "scripts": {
    "test": "open tests/index.html"
  },
  "repository": {
    "type": "git",
    "url": "ssh://git@bitbucket.org/casiano/pl-grado-temperature-converter.git"
  },
  "keywords": [
    "regexp"
  ],
  "author": "Casiano",
  "license": "ISC"
}


Is this ok? (yes) y
\end{verbatim}

Esto genera el fichero \verb|package.json|:
\begin{verbatim}
[/tmp/pl-grado-temperature-converter(karma)]$ ls -ltr | tail -1
-rw-r--r--  1 casiano  wheel   487  5 feb 18:22 package.json
\end{verbatim}
Si ahora escribo:
\begin{verbatim}
[/tmp/pl-grado-temperature-converter(karma)]$ npm test

> pl-grado-temperature-converter@0.0.1 test /private/tmp/pl-grado-temperature-converter
> open tests/index.html
\end{verbatim}
Ejecutamos las pruebas en el navegador (en Mac OS X) supuesto que ya estuvieran escritas.


