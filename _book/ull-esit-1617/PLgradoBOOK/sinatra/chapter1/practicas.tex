\sectionpractica{TicTacToe}
\label{practica:ticatactoe}

El código que sigue implanta un jugador de tres-en-raya.

\begin{enumerate}
\item 
Mejore el estilo actual usando SAAS: utilice variables, extensiones, mixins ...
\item 
Despliegue su versión en \Heroku{}
\end{enumerate}


\parrafo{Referencias}
\begin{enumerate}
\item 
\htmladdnormallink{http://sytw-tresenraya.herokuapp.com/}{http://sytw-tresenraya.herokuapp.com/}
\item 
\htmladdnormallink{https://github.com/crguezl/tictactoe-1}{https://github.com/crguezl/tictactoe-1}
\item  Sass (Syntactically Awesome StyleSheets): 
\htmladdnormallink{Sass Basics}{http://sass-lang.com/guide}
\item 
\htmladdnormallink{Un TicTacToe Simple}{https://github.com/crguezl/tictactoe} (No una webapp)
\end{enumerate}

\parrafo{Estructura}
\begin{verbatim}
[~/sinatra/sinatra-tictactoe/sinatra-tictactoe-ajax(master)]$ tree
.
|--- Gemfile
|--- Gemfile.lock
|--- Procfile
|--- Rakefile
|--- Readme.md
|--- app.rb
|--- public
|   |--- css
|   |   |--- app.css
|   |   `--- style.css
|   |--- images
|   |   |--- blackboard.jpg
|   |   |--- circle.gif
|   |   `--- cross.gif
|   `--- js
|       `--- app.js
`--- views
    |--- final.erb
    |--- final.haml
    |--- game.erb
    |--- game.haml
    |--- layout.erb
    |--- layout.haml
    `--- styles.scss

5 directories, 19 files
\end{verbatim}

\parrafo{Rakefile}
\begin{verbatim}
[~/sinatra/sinatra-tictactoe/sinatra-tictactoe-ajax(master)]$ cat Rakefile 
desc "run server"
task :default do
  sh "bundle exec ruby app.rb"
end

desc "install dependencies"
task :install do
  sh "bundle install"
end

###
desc 'build css'
task :css do
  sh "sass views/styles.scss public/css/style.css"
end
\end{verbatim}

\parrafo{HAML}
\begin{enumerate}
\item 
\htmladdnormallink{game.haml en GitHub}{https://github.com/crguezl/tictactoe-1/blob/master/views/game.haml}
\item 
\htmladdnormallink{layout.haml}{https://github.com/crguezl/tictactoe-1/blob/master/views/layout.haml}
\end{enumerate}

\begin{verbatim}
[~/sinatra/sinatra-tictactoe/sinatra-tictactoe-ajax(master)]$  cat views/game.haml 
.screen
  .gameboard
    - HORIZONTALS.each do |row| 
      .gamerow
        - row.each do |p|
          %a(href=p)
            %div{:id => "#{p}", :class => "cell #{b[p]}"}
    .message
      %h1= m
\end{verbatim}

\begin{verbatim}
[~/sinatra/sinatra-tictactoe/sinatra-tictactoe-ajax(master)]$ cat views/layout.haml 
!!!
%html
  %head
    %title tic tac toe
    -#%link{:rel=>"stylesheet", :href=>"/css/app.css", :type=>"text/css"}
    -# dynamically accessed
    -#%link{:rel=>"stylesheet", :href=>"/styles.css", :type=>"text/css"}    
    -# statically compiled
    %link{:rel=>"stylesheet", :href=>"css/style.css", :type=>"text/css"} 
    %script{:type=>"text/javascript", :src=>"http://ajax.googleapis.com/ajax/libs/jquery/1.6.4/jquery.min.js"}
    %script{:type=>"text/javascript", :src=>"/js/app.js"}
  %body
    = yield
\end{verbatim}

\begin{enumerate}
\item 
El fuente \verb|styles.scss| puede compilarse {\it dinámicamente}. Véase el fragmento de código 
que empieza por
\htmladdnormallink{{\tt get '/styles.css' do}}{https://github.com/crguezl/tictactoe-1/blob/master/app.rb\#L192-194}
en \verb|app.rb|
\item 
O puede compilarse estáticamente. Véase  el
\htmladdnormallink{Rakefile}{https://github.com/crguezl/tictactoe-1/blob/master/Rakefile\#L12-15}
\end{enumerate}

\parrafo{HTML generado}

\begin{verbatim}
<!DOCTYPE html>
<html>
  <head>
    <title>tic tac toe</title>
    <link href='css/style.css' rel='stylesheet' type='text/css'>
    <script src='http://ajax.googleapis.com/ajax/libs/jquery/1.6.4/jquery.min.js' type='text/javascript'></script>
    <script src='/js/app.js' type='text/javascript'></script>
  </head>
  <body>
    <div class='screen'>
      <div class='gameboard'>
        <div class='gamerow'>
          <a href='a1'>
            <div class='cell ' id='a1'></div>
          </a>
          <a href='a2'>
            <div class='cell ' id='a2'></div>
          </a>
          <a href='a3'>
            <div class='cell ' id='a3'></div>
          </a>
        </div>
        <div class='gamerow'>
          <a href='b1'>
            <div class='cell ' id='b1'></div>
          </a>
          <a href='b2'>
            <div class='cell circle' id='b2'></div>
          </a>
          <a href='b3'>
            <div class='cell ' id='b3'></div>
          </a>
        </div>
        <div class='gamerow'>
          <a href='c1'>
            <div class='cell ' id='c1'></div>
          </a>
          <a href='c2'>
            <div class='cell ' id='c2'></div>
          </a>
          <a href='c3'>
            <div class='cell cross' id='c3'></div>
          </a>
        </div>
        <div class='message'>
          <h1></h1>
        </div>
      </div>
    </div>
  </body>
</html>
\end{verbatim}

\parrafo{SASS}
\begin{enumerate}
\item 
\htmladdnormallink{styles.scss}{https://github.com/crguezl/tictactoe-1/blob/master/views/styles.scss}
\item  Sass (Syntactically Awesome StyleSheets): 
\htmladdnormallink{Sass Basics}{http://sass-lang.com/guide}
\item 
\htmladdnormallink{SASS documentación}{http://sass-lang.com/documentation/file.SASS\_REFERENCE.html}
\item 
\htmladdnormallink{sass man page}{http://manpages.ubuntu.com/manpages/precise/man1/sass.1.html}
\item {\it SASS (Syntactically Awesome StyleSheets)} \ref{chapter:sass}
\end{enumerate}

\begin{verbatim}
~/sinatra/sinatra-tictactoe/sinatra-tictactoe-ajax(master)]$ cat views/styles.scss 
$red:   #903;
$black: #444;
$white: #fff;
$ull:   #9900FF;
$pink:  #F9A7B0;

$main-font: Helvetica, Arial, sans-serif;
$message-font: 22px/1;

$board-left: 300px;
$board-margin: 0 auto;
$board-size: 500px;

$opacity:  0.8;

$cell-width:    $board-size/8.5;
$cell-height:   $board-size/8.5;
$cell-margin:  $cell-width/12;
$cell-padding:  $cell-width/1.3;

$background: "/images/blackboard.jpg";
$cross:      "/images/cross.gif";
$circle:     "/images/circle.gif";

body       { 
             // background-color: lightgrey; 
             font-family: $main-font;
             background: url($background) repeat; background-size: cover; 
           }
.gameboard { //margin-left: $board-left; 
             width: $board-size;
             margin: $board-margin;
             text-align:center;
           }
.gamerow   { clear: both; }
.cell      { color: blue; 
             background-color: white; 
             opacity: $opacity;
             width: $cell-width; 
             height: $cell-height; 
             margin: $cell-margin; 
             padding: $cell-padding; 
             &:hover {
               color: black ;
               background-color: $ull;
             }
             float: left; 
           }

@mixin game-piece($image) {
  background: url($image)  no-repeat; background-size: cover; 
}

.cross     { @include game-piece($cross); }
.circle    { @include game-piece($circle); }

.base-font { color: $pink; font: $message-font $main-font; }

.message   { 
             @extend .base-font;
             display: inline;
             background-color:transparent;
           }
\end{verbatim}

\parrafo{Procfile}

\htmladdnormallink{Procfile en GitHub}{https://github.com/crguezl/tictactoe-1/blob/master/Procfile\#L4}

In order to declare the processes that make our app, and scale them
individually, we need to be able to tell \Heroku{} what these processes are.

The Procfile is a simple YAML file which sits in the root of your
application code and is pushed to your application when you deploy. This
file contains a definition of every process you
require in your application, and how that process should be started. 

\begin{verbatim}
[~/sinatra/sinatra-tictactoe/sinatra-tictactoe-ajax(master)]$ cat Procfile 
#web: bundle exec unicorn -p $PORT -E $RACK_ENV
#web: bundle exec ruby app.rb -p $PORT
web: bundle exec ruby app.rb 
#web: bundle exec thin start 
\end{verbatim}
Véase 
\htmladdnormallink{The Procfile is your friend}{http://www.neilmiddleton.com/the-procfile-is-your-friend/}

\begin{enumerate}
\item 
\htmladdnormallink{Heroku, Thin and everything in between}{http://stackoverflow.com/questions/8625590/heroku-thin-and-everything-in-between}
en StackOverflow
\item 
\htmladdnormallink{Process Types and the Procfile}{https://devcenter.heroku.com/articles/procfile}
en \Heroku{}
\end{enumerate}

\parrafo{Gemfile}

\begin{verbatim}
[~/sinatra/sinatra-tictactoe/sinatra-tictactoe-ajax(master)]$ cat Gemfile
source "https://rubygems.org"

gem "sinatra"
gem 'haml'
gem "sass", :require => 'sass'
gem 'thin'
\end{verbatim}

\parrafo{La Aplicación}

\begin{verbatim}
[~/sinatra/sinatra-tictactoe/sinatra-tictactoe-ajax(master)]$  cat app.rb 
require 'sinatra'
require 'sass'
require 'pp'

settings.port = ENV['PORT'] || 4567
enable :sessions
#use Rack::Session::Pool, :expire_after => 2592000
#set :session_secret, 'super secret'

#configure :development, :test do
#  set :sessions, :domain => 'example.com'
#end

#configure :production do
#  set :sessions, :domain => 'herokuapp.com'
#end

module TicTacToe
  HUMAN = CIRCLE = "circle" # human
  COMPUTER = CROSS  = "cross"  # computer
  BLANK  = ""

  HORIZONTALS = [ %w{a1 a2 a3},  %w{b1 b2 b3}, %w{c1 c2 c3} ]
  COLUMNS     = [ %w{a1 b1 c1},  %w{a2 b2 c2}, %w{a3 b3 c3} ]
  DIAGONALS   = [ %w{a1 b2 c3},  %w{a3 b2 c1} ]
  ROWS = HORIZONTALS + COLUMNS + DIAGONALS
  MOVES       = %w{a1    a2   a3   b1   b2   b3   c1   c2   c3}

  def number_of(symbol, row)
    row.find_all{ |s| session["bs"][s] == symbol }.size 
  end

  def inicializa
    @board = {}
    MOVES.each do |k|
      @board[k] = BLANK
    end
    @board
  end

  def board
    session["bs"]
  end

  def [] key
    board[key]
  end

  def []= key, value
    board[key] = value
  end

  def each 
    MOVES.each do |move|
      yield move
    end
  end

  def legal_moves
    m = []
    MOVES.each do |key|
      m << key if board[key] == BLANK
    end
    puts "legal_moves: Tablero:  #{board.inspect}"
    puts "legal_moves: m:  #{m}"
    m # returns the set of feasible moves [ "b3", "c2", ... ]
  end

  def winner
    ROWS.each do |row|
      circles = number_of(CIRCLE, row)  
      puts "winner: circles=#{circles}"
      return CIRCLE if circles == 3  # "circle" wins
      crosses = number_of(CROSS, row)   
      puts "winner: crosses=#{crosses}"
      return CROSS  if crosses == 3
    end
    false
  end

  def smart_move
    moves = legal_moves

    ROWS.each do |row|
      if (number_of(BLANK, row) == 1) then
        if (number_of(CROSS, row) == 2) then # If I have a win, take it.  
          row.each do |e|
            return e if board[e] == BLANK
          end
        end
      end
    end
    ROWS.each do |row|
      if (number_of(BLANK, row) == 1) then
        if (number_of(CIRCLE,row) == 2) then # If he is threatening to win, stop it.
          row.each do |e|
            return e if board[e] == BLANK
          end
        end
      end
    end

    # Take the center if open.
    return "b2" if moves.include? "b2"

    # Defend opposite corners.
    if    self["a1"] != COMPUTER and self["a1"] != BLANK and self["c3"] == BLANK
      return "c3"
    elsif self["c3"] != COMPUTER and self["c3"] != BLANK and self["a1"] == BLANK
      return "a1"
    elsif self["a3"] != COMPUTER and self["a3"] != BLANK and self["c1"] == BLANK
      return "c1"
    elsif self["c1"] != COMPUTER and self["c3"] != BLANK and self["a3"] == BLANK
      return "a3"
    end
    
    # Or make a random move.
    moves[rand(moves.size)]
  end

  def human_wins?
    winner == HUMAN
  end

  def computer_wins?
    winner == COMPUTER
  end
end

helpers TicTacToe

get %r{^/([abc][123])?$} do |human|
  if human then
    puts "You played: #{human}!"
    puts "session: "
    pp session
    if legal_moves.include? human
      board[human] = TicTacToe::CIRCLE
      # computer = board.legal_moves.sample
      computer = smart_move
      redirect to ('/humanwins') if human_wins?
      redirect to('/') unless computer
      board[computer] = TicTacToe::CROSS
      puts "I played: #{computer}!"
      puts "Tablero:  #{board.inspect}"
      redirect to ('/computerwins') if computer_wins?
    end
  else
    session["bs"] = inicializa()
    puts "session = "
    pp session
  end
  haml :game, :locals => { :b => board, :m => ''  }
end

get '/humanwins' do
  puts "/humanwins session="
  pp session
  begin
    m = if human_wins? then
          'Human wins'
        else 
          redirect '/'
        end
    haml :final, :locals => { :b => board, :m => m }
  rescue
    redirect '/'
  end
end

get '/computerwins' do
  puts "/computerwins"
  pp session
  begin
    m = if computer_wins? then
          'Computer wins'
        else 
          redirect '/'
        end
    haml :final, :locals => { :b => board, :m => m }
  rescue
    redirect '/'
  end
end

not_found do
  puts "not found!!!!!!!!!!!"
  session["bs"] = inicializa()
  haml :game, :locals => { :b => board, :m => 'Let us start a new game'  }
end

get '/styles.css' do
  scss :styles
end
\end{verbatim}

\sectionpractica{TicTacToe usando DataMapper}
\label{practica:tictactoedatamapper}
Añada una base de datos a la práctica del TicTacToe
\ref{practica:ticatactoe}
de manera que se lleve la cuenta de los usuarios registrados, 
las partidas jugadas, ganadas y perdídas.
Repase la sección {\it DataMapper y Sinatra} \ref{chapter:datamapperysinatra}.

Mejore las hojas de estilo usando SAAS \ref{chapter:sass}.
Deberán mostrarse las celdas pares e impares en distintos colores.
También deberá mostrarse una lista de jugadores con sus registros.

Despliegue la aplicación en \Heroku{}.

\sectionpractica{Servicio de Syntax Highlighting}
Construya una aplicación que provee syntax higlighting
para un código que se vuelca en un formulario.
Use la gema \htmladdnormallink{syntaxi}{http://syntaxi.rubyforge.org/}.

El siguiente ejemplo muestra como funciona la gema \verb|syntaxi|:

\begin{verbatim}
[~/rubytesting/syntax_highlighting]$ cat ex_syntaxi.rb 
require 'syntaxi'
text = <<"EOF"
[code lang="ruby"]
  def foo
    puts 'bar'
  end
[/code]
EOF
formatted_text = Syntaxi.new(text).process
puts formatted_text
\end{verbatim}
Ejecución:
\begin{verbatim}
[~/rubytesting/syntax_highlighting]$ ruby ex_syntaxi.rb 
<pre>
<code>
<span class="line_number">1</span> <span class="keyword">def </span><span class="method">foo</span>
<span class="line_number">2</span> <span class="ident">puts</span> 
<span class="punct">'</span><span class="string">bar</span><span class="punct">'</span>
<span class="line_number">3</span> <span class="keyword">end</span>
</code>
</pre>
\end{verbatim}
La gema \verb|syntaxi| usa la gema \verb|syntax|:
\begin{verbatim}
[~/rubytesting/syntax_highlighting]$ gem which syntaxi/Users/casiano/.rvm/gems/ruby-1.9.2-head/gems/syntaxi-0.5.0/lib/syntaxi.rb
[~/rubytesting/syntax_highlighting]$ grep "require.*'" /Users/casiano/.rvm/gems/ruby-1.9.2-head/gems/syntaxi-0.5.0/lib/syntaxi.rb 
require 'syntax/convertors/html'
\end{verbatim}
Es en esta gema que se definen las hojas de estilo:
\begin{verbatim}
[~/rubytesting/syntax_highlighting]$ gem which syntax
/Users/casiano/.rvm/gems/ruby-1.9.2-head/gems/syntax-1.0.0/lib/syntax.rb
[~/rubytesting/syntax_highlighting]$ tree /Users/casiano/.rvm/gems/ruby-1.9.2-head/gems/syntax-1.0.0/
/Users/casiano/.rvm/gems/ruby-1.9.2-head/gems/syntax-1.0.0/
|-- data
|   |-- ruby.css
|   |-- xml.css
|   `-- yaml.css
|-- lib
|   |-- syntax
|   |   |-- common.rb
|   |   |-- convertors
|   |   |   |-- abstract.rb
|   |   |   `-- html.rb
|   |   |-- lang
|   |   |   |-- ruby.rb
|   |   |   |-- xml.rb
|   |   |   `-- yaml.rb
|   |   `-- version.rb
|   `-- syntax.rb
`-- test
    |-- ALL-TESTS.rb
    |-- syntax
    |   |-- tc_ruby.rb
    |   |-- tc_xml.rb
    |   |-- tc_yaml.rb
    |   `-- tokenizer_testcase.rb
    `-- tc_syntax.rb

7 directories, 17 files

\end{verbatim}

En el esquema incompleto que sigue se ha hecho para el lenguaje
Ruby. Añada que se pueda elegir el lenguaje a colorear (xml, yaml).

\begin{verbatim}
$ tree -A
.
|-- Gemfile
|-- Gemfile.lock
|-- toopaste.rb
`-- views
    |-- layout.erb
    |-- new.erb
    `-- show.erb
\end{verbatim}

\begin{verbatim}
$ cat Gemfile
source 'https://rubygems.org'

# Specify your gem's dependencies in my-gem.gemspec
# gemspec
# gem 'guard'
# gem 'guard-rspec'
# gem 'guard-bundler'
# gem 'rb-fsevent', '~> 0.9.1''

gem 'syntaxi'

\end{verbatim}

%\begin{verbatim}
%$ cat views/new.erb 
%<div class="snippet">
%  <form action="/" method="POST">
%    <textarea name="body" id="body" rows="20"></textarea>
%    <br/><input type="submit" value="Save"/>
%  </form>
%</div>
%\end{verbatim}
Este es un fragmento de la aplicación:
\begin{verbatim}
[~/srcSTW/syntax_highlighting(withoutdm)]$ cat  toopaste.rb 
require 'sinatra'
require 'syntaxi'

class String
  def formatted_body
    source = "[code lang='ruby']
                #{self}
              [/code]"
    html = Syntaxi.new(source).process
    %Q{
      <div class="syntax syntax_ruby">
        #{html}
      </div>
    }
  end
end

get '/' do
  erb :new
end

post '/' do
  .....
end

\end{verbatim}
Una versión simple de lo que puede ser \verb|new.erb|:
\begin{verbatim}
[~/srcSTW/syntax_highlighting(withoutdm)]$ cat views/new.erb 
<div class="snippet">
  <form action="/" method="POST">
    <textarea name="body" id="body" rows="20"></textarea>
    <br/><input type="submit" value="Save"/>
  </form>
</div>
\end{verbatim}

Véase la página  HTML generada por el programa para la entrada \verb|a = 5|:
\begin{verbatim}
<!DOCTYPE html PUBLIC "-//W3C//DTD XHTML 1.0 Transitional//EN" "http://www.w3.org/TR/xhtml1/DTD/xhtml1-transitional.dtd">
<html xmlns="http://www.w3.org/1999/xhtml">
<head>
  <title>Toopaste!</title>
  <style>
    html {
      background-color: #eee;
    }
    .snippet {
      margin: 5px;
    }
    .snippet textarea, .snippet .sbody {
      border: 5px dotted #eee;
      padding: 5px;
      width: 700px;
      color: #fff;
      background-color: #333;
    }
    .snippet textarea {
      padding: 20px;
    }
    .snippet input, .snippet .sdate {
      margin-top: 5px;
    }

    /* Syntax highlighting */
    #content .syntax_ruby .normal {}
    #content .syntax_ruby .comment { color: #CCC; font-style: italic; border: none; margin: none; }
    #content .syntax_ruby .keyword { color: #C60; font-weight: bold; }
    #content .syntax_ruby .method { color: #9FF; }
    #content .syntax_ruby .class { color: #074; }
    #content .syntax_ruby .module { color: #050; }
    #content .syntax_ruby .punct { color: #0D0; font-weight: bold; }
    #content .syntax_ruby .symbol { color: #099; }
    #content .syntax_ruby .string { color: #C03; }
    #content .syntax_ruby .char { color: #F07; }
    #content .syntax_ruby .ident { color: #0D0; }
    #content .syntax_ruby .constant { color: #07F; }
    #content .syntax_ruby .regex { color: #B66; }
    #content .syntax_ruby .number { color: #FF0; }
    #content .syntax_ruby .attribute { color: #7BB; }
    #content .syntax_ruby .global { color: #7FB; }
    #content .syntax_ruby .expr { color: #909; }
    #content .syntax_ruby .escape { color: #277; }
    #content .syntax {
      background-color: #333;
      padding: 2px;
      margin: 5px;
      margin-left: 1em;
      margin-bottom: 1em;
    }
    #content .syntax .line_number {
      text-align: right;
      font-family: monospace;
      padding-right: 1em;
      color: #999;
    }
  </style>
</head>
<body>
  <div class="snippet">
  <div class="snippet">
  <div class="sbody" id="content">
      <div class="syntax syntax_ruby">
        <pre>
           <code>
              <span class="line_number">1</span>  
              <span class="ident">a</span> 
              <span class="punct">=</span> 
              <span class="number">5</span>
            </code>
         </pre>
      </div>
  </div>
  <br/><a href="/">New Paste!</a>
</div>
</body>
</html>
\end{verbatim}

La gema

\begin{verbatim}
\end{verbatim}
Una versión resumida de \verb|layout.erb|:
\begin{verbatim}
[~/srcSTW/syntax_highlighting(withoutdm)]$ cat views/layout.erb 
<!DOCTYPE html PUBLIC "-//W3C//DTD XHTML 1.0 Transitional//EN" "http://www.w3.org/TR/xhtml1/DTD/xhtml1-transitional.dtd">
<html xmlns="http://www.w3.org/1999/xhtml">
<head>
  <title><%= @title || 'Toopaste!' %></title>
  <style>
    ..............
  </style>
</head>
<body>
  <%= yield %>
</body>
</html>
\end{verbatim}

