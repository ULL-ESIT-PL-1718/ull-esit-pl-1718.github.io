\section{Ejemplo Simple de uso de Sinatra}

\parrafo{Código}
\begin{verbatim}
[~/sinatra/sinatra-simple(master)]$ cat hi.rb 
require 'sinatra'

get '/hi' do
  "Hello World!"
end
\end{verbatim}

\parrafo{Opciones de Ejecución}
\begin{verbatim}
[~/sinatra/sinatra-simple(master)]$ ruby hi.rb --help
Usage: hi [options]
    -p port                          set the port (default is 4567)
    -o addr                          set the host (default is localhost)
    -e env                           set the environment (default is development)
    -s server                        specify rack server/handler (default is thin)
    -x                               turn on the mutex lock (default is off)
\end{verbatim}
Sinatra can be used in threaded environments where more than a
single request is processed at a time. 

However, not all applications
and libraries are thread-safe and may cause intermittent errors or
general weirdness. 

Enabling the \verb|-x| setting causes all requests
to synchronize on a mutex lock, ensuring that only a single request
is processed at a time.

The mutex lock setting is disabled by default.

\section{Rutas/Routes}\label{routes}

Repase la sección
{\it HTTP}
\ref{section:http}.

Vease el código de este ejemplo en 
\htmladdnormallink{GitHub}{https://github.com/crguezl/sinatra-simple}

\parrafo{Aplicacion}
\begin{verbatim}
[~/Dropbox/src/ruby/sinatra/sinatra-simple(master)]$ cat app.rb
require 'sinatra/base'

class App < Sinatra::Base
  get '/' do
    "hello get!"
  end

  post '/' do
    'hello post!'
  end

  put '/' do
    'hello put!'
  end

  delete '/' do
    'hello delete!'
  end

  get '/:name' do |name|
    "hello #{name}!"
  end

  get '/:name/?:apelido1?' do |name, apellido|
    "hello #{apellido}, #{name}!"
  end
end
\end{verbatim}


In Sinatra, a route is an HTTP method paired with a URL-matching
pattern. Each route is associated with a block

Routes are matched in the order they are defined. The first route
that matches the request is invoked.

Route patterns may include named parameters, accessible via the params hash:

\begin{verbatim}        get '/hello/:name' do
          # matches "GET /hello/foo" and "GET /hello/bar"
          # params[:name] is 'foo' or 'bar'
          "Hello #{params[:name]}!"
        end
\end{verbatim}

You can also access named parameters via block parameters:

\begin{verbatim}        get '/:name' do |name|
          "hello #{name}!"
        end
\end{verbatim}

Route patterns may also include splat (or wildcard) parameters,
accessible via the {\tt params[:splat]} array:

\begin{verbatim}        get '/say/*/to/*' do
          # matches /say/hello/to/world
          params[:splat] # => ["hello", "world"]
        end

        get '/download/*.*' do
          # matches /download/path/to/file.xml
          params[:splat] # => ["path/to/file", "xml"]
        end
\end{verbatim}

\parrafo{config.ru}
\begin{verbatim}
[~/sinatra/sinatra-simple(master)]$ cat config.ru 
require './app'

run App
\end{verbatim}

\parrafo{Rakefile}
\begin{verbatim}
[~/sinatra/sinatra-simple(master)]$ cat Rakefile 
task :default => :server

desc "run server"
task :server do
  sh "rackup"
end

desc "make a get / request via curl"
task :get do
  sh "curl -v localhost:9292"
end

desc "make a post / request via curl"
task :post do
  sh "curl -X POST -v -d 'ignored data' localhost:9292"
end

desc "make a put / request via curl"
task :put do
  sh "curl -X PUT -v localhost:9292"
end

desc "make a DELETE / request via curl"
task :delete do
  sh "curl -X DELETE -v localhost:9292"
end

desc "make a get /name request via curl"
task :getname, :name do |t,h|
  name = h[:name] or 'pepe'
  sh "curl -v localhost:9292/#{name}"
end

desc "make a get /name/appellido request via curl"
task :getfullname, :name, :apellido do |t,h|
  name = h[:name] or 'pepe'
  apellido = h[:apellido] or 'rodriguez'
  sh "curl -v localhost:9292/#{name}/#{apellido}"
end

task :html do
  sh "kramdown README.md > README.html"
end
\end{verbatim}

\parrafo{Ejecución del servidor}

\begin{verbatim}
[~/sinatra/sinatra-simple(master)]$ rake server
rackup
>> Thin web server (v1.5.1 codename Straight Razor)
>> Maximum connections set to 1024
>> Listening on 0.0.0.0:9292, CTRL+C to stop
127.0.0.1 - - [01/Jul/2013 20:25:16] "GET /juana HTTP/1.1" 200 12 0.0689
\end{verbatim}

\parrafo{Ejecución de los clientes}

\begin{verbatim}
[~/Dropbox/src/ruby/sinatra/sinatra-simple(master)]$ rake getname[juana]
{:name=>"juana"}
curl -v localhost:9292/juana
* About to connect() to localhost port 9292 (#0)
*   Trying ::1... Connection refused
*   Trying 127.0.0.1... connected
* Connected to localhost (127.0.0.1) port 9292 (#0)
> GET /juana HTTP/1.1
> User-Agent: curl/7.21.4 (universal-apple-darwin11.0) libcurl/7.21.4 OpenSSL/0.9.8x zlib/1.2.5
> Host: localhost:9292
> Accept: */*
> 
< HTTP/1.1 200 OK
< Content-Type: text/html;charset=utf-8
< Content-Length: 12
< X-XSS-Protection: 1; mode=block
< X-Content-Type-Options: nosniff
< X-Frame-Options: SAMEORIGIN
< Connection: keep-alive
< Server: thin 1.5.1 codename Straight Razor
< 
* Connection #0 to host localhost left intact
* Closing connection #0
hello juana!
\end{verbatim}

\begin{verbatim}
[~/Dropbox/src/ruby/sinatra/sinatra-simple(master)]$ rake getfullname[Ana,Hernandez]
curl -v localhost:9292/Ana/Hernandez
* About to connect() to localhost port 9292 (#0)
*   Trying ::1... Connection refused
*   Trying 127.0.0.1... connected
* Connected to localhost (127.0.0.1) port 9292 (#0)
> GET /Ana/Hernandez HTTP/1.1
> User-Agent: curl/7.21.4 (universal-apple-darwin11.0) libcurl/7.21.4 OpenSSL/0.9.8x zlib/1.2.5
> Host: localhost:9292
> Accept: */*
> 
< HTTP/1.1 200 OK
< Content-Type: text/html;charset=utf-8
< Content-Length: 21
< X-XSS-Protection: 1; mode=block
< X-Content-Type-Options: nosniff
< X-Frame-Options: SAMEORIGIN
< Connection: keep-alive
< Server: thin 1.5.1 codename Straight Razor
< 
* Connection #0 to host localhost left intact
* Closing connection #0
hello Hernandez, Ana!
\end{verbatim}

\subsection{Verbos HTTP en Sinatra/Base}

\htmladdnormallink{Método {\tt get}}{https://github.com/sinatra/sinatra/blob/master/lib/sinatra/base.rb\#L1351}
\begin{verbatim}
      def get(path, opts = {}, &block)
        conditions = @conditions.dup
        route('GET', path, opts, &block)

        @conditions = conditions
        route('HEAD', path, opts, &block)
      end
\end{verbatim}

\begin{verbatim}
      def put(path, opts = {}, &bk)     route 'PUT',     path, opts, &bk end
      def post(path, opts = {}, &bk)    route 'POST',    path, opts, &bk end
      def delete(path, opts = {}, &bk)  route 'DELETE',  path, opts, &bk end
      def head(path, opts = {}, &bk)    route 'HEAD',    path, opts, &bk end
      def options(path, opts = {}, &bk) route 'OPTIONS', path, opts, &bk end
      def patch(path, opts = {}, &bk)   route 'PATCH',   path, opts, &bk end
      def link(path, opts = {}, &bk)    route 'LINK',    path, opts, &bk end
      def unlink(path, opts = {}, &bk)  route 'UNLINK',  path, opts, &bk end
\end{verbatim}

\htmladdnormallink{Método {\tt route}}{https://github.com/sinatra/sinatra/blob/master/lib/sinatra/base.rb\#L1563}
\begin{verbatim}
      def route(verb, path, options = {}, &block)
        # Because of self.options.host
        host_name(options.delete(:host)) if options.key?(:host)
        enable :empty_path_info if path == "" and empty_path_info.nil?
        signature = compile!(verb, path, block, options)
        (@routes[verb] ||= []) << signature
        invoke_hook(:route_added, verb, path, block)
        signature
      end
\end{verbatim}


\section{Ficheros Estáticos}
\begin{enumerate}
\item 
Static files are served from the \verb|./public| directory. 
\item 
You can specify a different location by setting the \verb|:public_folder| option:
\begin{verbatim}
set :public_folder, File.dirname(__FILE__) + '/static'
\end{verbatim}
Put this code in a \verb|configure| block
\item 
Note that the public directory name is not included in the URL. 
\item 
A file \verb|./public/css/style.css| is made available as \verb|http://example.com/css/style.css|.
\item 
Use the \verb|:static_cache_control| setting to add Cache-Control header info.
Use an explicit array when setting multiple values:
\begin{verbatim}
 set :static_cache_control, [:public, :max_age => 300]
\end{verbatim}
\item 
What would be delivered in the event that a defined route conflicts with the name of the static resource?.
The answer is the static resource.
\end{enumerate}

\section{Vistas}

\begin{quote}
Writing a program that spits out HTML is often more difficult than
you might imagine. Although programming languages are better at
creating text than they used to be (some of us remember character
handling in Fortran and standard Pascal), creating and concatenating
string constructs is still painful. If there isn't much to do, it
isn't too bad, but a whole HTML page is a lot of text manipulation.

With static HTML pages - those that don't change from request to
request - you can use nice WYSIWG editors. Even those of us who
like raw text editors find it easier to just type in the text and
tags rather than fiddle with string concatenation in a programming
language.

Of course the issue is with dynamic Web pages - those that take the
results of something like database queries and embed them into the
HTML. The page looks different with each result, and as a result
regular HTML editors aren't up to the job.

The best way to work is to compose the dynamic Web page as you do
a static page but put in markers that can be resolved into calls
to gather dynamic information. Since the static part of the page
acts as a template for the particular response, I call this a
\cei{Template View}.

\flushright{Martin Fowler}

\end{quote}

Views in Sinatra  are \cei{HTML templates} that can optionally contain data
passed from the application.

There are two ways to work with views in Sinatra: \cei{inline templates} and
\cei{external templates}.

Véase en GitHub 
\htmladdnormallink{sinatra-up-and-running/tree/master/chapter2/views}{https://github.com/crguezl/sinatra-up-and-running/tree/master/chapter2/views}.

\subsection{Templates Inline}

Templates may be defined at the end of the source file.
En este ejemplo trabajamos con varios templates inline en diferentes ficheros:
\begin{verbatim}
[~/sinatra/sinatraupandrunning/chapter2/views(master)]$ cat example2-14.rb 
require 'sinatra/base'

class App < Sinatra::Base
  enable :inline_templates
  get '/index' do 
    puts "Visiting #{request.url}"
    erb :index
  end
end

require './another'
__END__
@@index
<!DOCTYPE html>
<html> 
  <head>
    <meta charset="UTF-8">
    <title>Inline template</title> 
  </head>
  <body> 
    <h1>Worked!</h1>
  </body> 
</html>
\end{verbatim}

En este fichero tenemos un segundo template inline:
\begin{verbatim}
[~/sinatra/sinatraupandrunning/chapter2/views(master)]$ cat another.rb 
class App 
  enable :inline_templates
  get '/' do
    erb :another
  end
end

__END__
@@another
<!DOCTYPE html>
<html> 
  <head>
    <meta charset="UTF-8">
    <title>Separated file</title> 
  </head>
  <body> 
    <h1>Inside another!</h1>
  </body> 
</html>
\end{verbatim}

Este es nuestro config.ru:

\begin{verbatim}
[~/sinatra/sinatraupandrunning/chapter2/views(master)]$ cat config.ru 
require './example2-14'

run App
\end{verbatim}
Para simplificar las cosas hemos hecho un \verb|Rakefile|:
\begin{verbatim}
[~/sinatra/sinatraupandrunning/chapter2/views(master)]$ cat Rakefile 
task :default => :server

desc "run server"
task :server do
  sh "rackup"
end

desc "make a get / request via curl"
task :root do
  sh "curl -v localhost:9292"
end

desc "make a get /index request via curl"
task :index do 
  sh "curl -v localhost:9292/index"
end
\end{verbatim}

El resultado de la ejecución es:
\begin{verbatim}
[~/sinatra/sinatra-up-and-running/chapter2/views/inline_templates(master)]$ curl http://localhost:9292/index
<!DOCTYPE html>
<html> 
  <head>
    <meta charset="UTF-8">
    <title>Inline template</title> 
  </head>
  <body> 
    <h1>Worked!</h1>
  </body> 
</html>
[~/sinatra/sinatra-up-and-running/chapter2/views/inline_templates(master)]$ curl http://localhost:9292/
<!DOCTYPE html>
<html> 
  <head>
    <meta charset="UTF-8">
    <title>Separated file</title> 
  </head>
  <body> 
    <h1>Inside another!</h1>
  </body> 
</html>
\end{verbatim}

\subsection{Named Templates}
Templates may also be defined using the top-level \verb|template| method:

\begin{verbatim}
template :layout do
  "%html\n  =yield\n"
end

template :index do
  '%div.title Hello World!'
end

get '/' do
  haml :index
end
\end{verbatim}
If a template named \verb|layout| exists, it will be used each time a template is rendered. 

You can individually disable layouts by passing \verb|:layout => false| or disable them by default via \verb|set :haml, :layout => false|:

\begin{verbatim}
get '/' do
  haml :index, :layout => !request.xhr?
end
\end{verbatim}

\subsection{Templates Externos}

\begin{verbatim}
$ ls
Rakefile               example2-16.rb         views
config.ru              
\end{verbatim}

\begin{verbatim}
$ cat example2-16.rb 
require 'sinatra/base'

class App < Sinatra::Base
  get '/index' do 
    puts "Visiting #{request.url}"
    erb :index
  end
end
\end{verbatim}

\begin{verbatim}
$ cat views/index.erb 
<!DOCTYPE html>
<html> 
  <head>
    <meta charset="UTF-8">
    <title>Inline template</title> 
  </head>
  <body> 
    <h1>Worked!</h1>
  </body> 
</html>
\end{verbatim}

\begin{verbatim}
$ cat config.ru 
require './example2-16'

run App
\end{verbatim}

\begin{verbatim}
$ cat Rakefile 
task :default => :server

desc "run server"
task :server do
  sh "rackup"
end

desc "make a get / request via curl"
task :root do
  sh "curl -v localhost:9292"
end

desc "make a get /index request via curl"
task :index do 
  sh "curl -v localhost:9292/index"
end
\end{verbatim}

\begin{verbatim}
$ rake server
rackup
>> Thin web server (v1.5.1 codename Straight Razor)
>> Maximum connections set to 1024
>> Listening on 0.0.0.0:9292, CTRL+C to stop
Visiting http://localhost:9292/index
127.0.0.1 - - [03/Jul/2013 22:30:16] "GET /index HTTP/1.1" 200 157 0.0774
\end{verbatim}

\begin{verbatim}
$ rake index
curl -v localhost:9292/index
* About to connect() to localhost port 9292 (#0)
*   Trying 127.0.0.1... connected
* Connected to localhost (127.0.0.1) port 9292 (#0)
> GET /index HTTP/1.1
> User-Agent: curl/7.21.4 (universal-apple-darwin11.0) libcurl/7.21.4 OpenSSL/0.9.8x zlib/1.2.5
> Host: localhost:9292
> Accept: */*
> 
< HTTP/1.1 200 OK
< Content-Type: text/html;charset=utf-8
< Content-Length: 157
< X-XSS-Protection: 1; mode=block
< X-Content-Type-Options: nosniff
< X-Frame-Options: SAMEORIGIN
< Connection: keep-alive
< Server: thin 1.5.1 codename Straight Razor
< 
<!DOCTYPE html>
<html> 
  <head>
    <meta charset="UTF-8">
    <title>Inline template</title> 
  </head>
  <body> 
    <h1>Worked!</h1>
  </body> 
</html>

* Connection #0 to host localhost left intact
* Closing connection #0
\end{verbatim}

\subsection{Templates Externos en Subcarpetas}

Véase en GitHub 
\htmladdnormallink{sinatra-up-and-running/tree/master/chapter2/views/external\_view\_files/external\_in\_subfolders}{https://github.com/crguezl/sinatra-up-and-running/tree/master/chapter2/views/external\_view\_files/external\_in\_subfolders}

\begin{verbatim}
$ ls
Rakefile  app.rb    config.ru views
\end{verbatim}

\begin{verbatim}
$ cat app.rb 
require 'sinatra/base'

class App < Sinatra::Base
  get '/:user/profile' do |user|
    @user = user
    erb '/user/profile'.to_sym
  end

  get '/:user/help' do |user|
    @user = user
    erb :'/user/help'
  end
end
\end{verbatim}

\begin{verbatim}
$ cat views/user/profile.erb 
<!DOCTYPE html>
<html> 
  <head>
    <meta charset="UTF-8">
    <title>Profile Template</title> 
  </head>
  <body> 
    <h1>Profile of <%= @user %></h1>
     <%= params %>
  </body> 
</html>
\end{verbatim}


\begin{verbatim}
$ cat views/user/help.erb 
<!DOCTYPE html>
<html> 
  <head>
    <meta charset="UTF-8">
    <title>HELP Template</title> 
  </head>
  <body> 
    <h1>Help for user <%= @user %></h1>
     <pre>
       <%= params %>
     </pre>
  </body> 
</html>
\end{verbatim}

\begin{verbatim}
$ cat config.ru 
require './app'

run App
\end{verbatim}

\begin{verbatim}
$ cat Rakefile 
PORT = 9292
task :default => :server

desc "run server"
task :server do
  sh "rackup"
end

desc "make a get /pepe/profile request via curl"
task :profile, :name do |t, h|
  user = h['name'] || 'pepe'
  sh "curl -v localhost:#{PORT}/#{user}/profile"
end

desc "make a get /pepe/help request via curl"
task :help do 
  sh "curl -v localhost:#{PORT}/pepe/help"
end
\end{verbatim}

\begin{verbatim}
$ rake server
rackup
>> Thin web server (v1.5.1 codename Straight Razor)
>> Maximum connections set to 1024
>> Listening on 0.0.0.0:9292, CTRL+C to stop
127.0.0.1 - - [03/Jul/2013 21:40:04] "GET /Pedro/profile HTTP/1.1" 200 227 0.1077
\end{verbatim}


\begin{verbatim}
$ rake profile[Pedro]
curl -v localhost:9292/Pedro/profile
* About to connect() to localhost port 9292 (#0)
*   Trying 127.0.0.1... connected
* Connected to localhost (127.0.0.1) port 9292 (#0)
> GET /Pedro/profile HTTP/1.1
> User-Agent: curl/7.21.4 (universal-apple-darwin11.0) libcurl/7.21.4 OpenSSL/0.9.8x zlib/1.2.5
> Host: localhost:9292
> Accept: */*
> 
< HTTP/1.1 200 OK
< Content-Type: text/html;charset=utf-8
< Content-Length: 227
< X-XSS-Protection: 1; mode=block
< X-Content-Type-Options: nosniff
< X-Frame-Options: SAMEORIGIN
< Connection: keep-alive
< Server: thin 1.5.1 codename Straight Razor
< 
<!DOCTYPE html>
<html> 
  <head>
    <meta charset="UTF-8">
    <title>Profile Template</title> 
  </head>
  <body> 
    <h1>Profile of Pedro</h1>
     {"splat"=>[], "captures"=>["Pedro"], "user"=>"Pedro"}
  </body> 
</html>


* Connection #0 to host localhost left intact
* Closing connection #0
\end{verbatim}

\subsection{Variables en las Vistas}

\parrafo{Comunicación vía variables de instancia}
Los templates se evaluan en el mismo contexto que los manejadores de las rutas.
Las variables de instancia son accesibles directamente en los templates.

\begin{verbatim}
get '/:id' do
  @foo = Foo.find(params[:id])
  haml '%h1= @foo.name'
end
\end{verbatim}
Veamos un ejemplo de comunicación via variables de instancia entre el manejador de la ruta
y el template:

\begin{verbatim}
[~/sinatra/sinatra-views/passing_data_into_views(master)]$ ls
Rakefile        config.ru       via_instance.rb
\end{verbatim}


\begin{verbatim}
[~/sinatra/sinatra-views/passing_data_into_views(master)]$ cat via_instance.rb 
require 'sinatra/base'


class App < Sinatra::Base
  get '/*' do |name|
    def some_template
       <<-'HAMLTEMP'
%ol
  - @foo.each do |item|
    %li 
      %i #{item}
HAMLTEMP
    end

    puts "*---***#{name}*---****"
    @foo = name.split('/')
    haml some_template
  end
end
\end{verbatim}

\begin{verbatim}
[~/sinatra/sinatra-views/passing_data_into_views(master)]$ cat config.ru 
require './via_instance'

run App
\end{verbatim}

\begin{verbatim}
[~/sinatra/sinatra-views/passing_data_into_views(master)]$ cat Rakefile 
task :default => :server

desc "run server"
task :server do
  sh "rackup"
end

desc "make a get /juan/leon/hernandez request via curl"
task :client do
  sh "curl -v localhost:9292/juan/leon/hernandez"
end
\end{verbatim}

\begin{verbatim}
[~/sinatra/sinatraupandrunning/chapter2/views/passing_data_into_views(master)]$ rake server
rackup
>> Thin web server (v1.5.1 codename Straight Razor)
>> Maximum connections set to 1024
>> Listening on 0.0.0.0:9292, CTRL+C to stop
*---***juan/leon/hernandez*---****
127.0.0.1 - - [05/Jul/2013 17:06:05] "GET /juan/leon/hernandez HTTP/1.1" 200 109 0.3502
\end{verbatim}

\begin{verbatim}
[~/sinatra/sinatra-views/passing_data_into_views(master)]$ rake client
curl -v localhost:9292/juan/leon/hernandez
* About to connect() to localhost port 9292 (#0)
*   Trying 127.0.0.1... connected
* Connected to localhost (127.0.0.1) port 9292 (#0)
> GET /juan/leon/hernandez HTTP/1.1
> User-Agent: curl/7.21.4 (universal-apple-darwin11.0) libcurl/7.21.4 OpenSSL/0.9.8x zlib/1.2.5
> Host: localhost:9292
> Accept: */*
> 
< HTTP/1.1 200 OK
< Content-Type: text/html;charset=utf-8
< Content-Length: 109
< X-XSS-Protection: 1; mode=block
< X-Content-Type-Options: nosniff
< X-Frame-Options: SAMEORIGIN
< Connection: keep-alive
< Server: thin 1.5.1 codename Straight Razor
< 
<ol>
  <li>
    <i>juan</i>
  </li>
  <li>
    <i>leon</i>
  </li>
  <li>
    <i>hernandez</i>
  </li>
</ol>
* Connection #0 to host localhost left intact
* Closing connection #0
\end{verbatim}

\subsection{Pasando variables a la vista explícitamente via un hash}

También es posible pasar en la llamada un hash especificando 
las variables locales:
\begin{verbatim}
get '/:id' do
  foo = Foo.find(params[:id])
  haml '%h1= bar.name', :locals => { :bar => foo }
end
\end{verbatim}
This is typically used when rendering templates as partials from within other templates.

Veamos un ejemplo:

\begin{verbatim}
$ ls
Rakefile    config.ru   via_hash.rb views
\end{verbatim}

\begin{verbatim}
$ cat via_hash.rb
require 'sinatra/base'

class App < Sinatra::Base
  get '/*' do |name|
    def some_template
       <<-'ERBTEMP'
<ul><% name.each do |item| %>
      <li> <i> <%= item %> </i> </li>
    <% end %>
</ul>
ERBTEMP
    end # method some_template

    puts "*---***#{name}*---****"
    erb some_template, :locals => { :name => name.split('/')}
  end
end
\end{verbatim}

\begin{verbatim}
$ cat views/layout.erb 
<!DOCTYPE html>
<html>
  <head>
        <title>Sinatra</title>
  </head>
  <body>
    <h1>Accesing variables in templates via a parameter hash</h1>
    <%= yield %>
  </body>
</html>
\end{verbatim}

\begin{verbatim}
$ cat config.ru 
require './via_hash'

run App
\end{verbatim}

\begin{verbatim}
$ cat Rakefile task :default => :server

desc "run server"
task :server do
  sh "rackup"
end

desc "make a get /juan/leon/hernandez request via curl"
task :client do
  sh "curl -v localhost:9292/juan/leon/hernandez"
end
\end{verbatim}

\begin{verbatim}
$ rake serverrackup
>> Thin web server (v1.5.1 codename Straight Razor)
>> Maximum connections set to 1024
>> Listening on 0.0.0.0:9292, CTRL+C to stop
*---***juan/leon/hernandez*---****
127.0.0.1 - - [05/Jul/2013 17:50:20] "GET /juan/leon/hernandez HTTP/1.1" 200 290 0.0352
\end{verbatim}

\begin{verbatim}
$ rake client
curl -v localhost:9292/juan/leon/hernandez
* About to connect() to localhost port 9292 (#0)
*   Trying 127.0.0.1... connected
* Connected to localhost (127.0.0.1) port 9292 (#0)
> GET /juan/leon/hernandez HTTP/1.1
> User-Agent: curl/7.21.4 (universal-apple-darwin11.0) libcurl/7.21.4 OpenSSL/0.9.8x zlib/1.2.5
> Host: localhost:9292
> Accept: */*
> 
< HTTP/1.1 200 OK
< Content-Type: text/html;charset=utf-8
< Content-Length: 290
< X-XSS-Protection: 1; mode=block
< X-Content-Type-Options: nosniff
< X-Frame-Options: SAMEORIGIN
< Connection: keep-alive
< Server: thin 1.5.1 codename Straight Razor
< 
<!DOCTYPE html>
<html>
  <head>
        <title>Sinatra</title>
  </head>
  <body>
    <h1>Accesing variables in templates via a parameter hash</h1>
    <ul>
      <li> <i> juan </i> </li>
    
      <li> <i> leon </i> </li>
    
      <li> <i> hernandez </i> </li>
    
</ul>

  </body>
</html>
* Connection #0 to host localhost left intact
* Closing connection #0
\end{verbatim}

\subsection{Opciones pasadas a los Métodos de los Templates}

Options passed to the render method override options set via \verb|set|.

Available Options:

\begin{enumerate}
\item \verb|locals|

List of locals passed to the document. Handy with partials. Example: 
\begin{verbatim}
erb "<%= foo %>", :locals => {:foo => "bar"}
\end{verbatim}
\item \verb|default_encoding|

String encoding to use if uncertain. Defaults to 
\begin{verbatim}
settings.default_encoding.
\end{verbatim}

\item 
\verb|views|

Views folder to load templates from. Defaults to \verb|settings.views|.

\item 
\verb|layout|

Whether to use a layout (true or false), if it's a Symbol, specifies what template to use. Example: 
\begin{verbatim}
erb :index, :layout => !request.xhr?
\end{verbatim}

\item 
\verb|content_type|

Content-Type the template produces, default depends on template language.

\item 
\verb|scope|

Scope to render template under. 

Defaults to the application instance. 

If you change this, instance variables and helper methods will not be available.

\item 
\verb|layout_engine|

Template engine to use for rendering the layout. 

Useful for languages
that do not support layouts otherwise. 

Defaults to the engine used
for the template. Example: 

\begin{verbatim}
set :rdoc, :layout_engine => :erb
\end{verbatim}

\item 
\verb|layout_options|

Special options only used for rendering the layout. Example: 

\begin{verbatim}
set :rdoc, :layout_options => { :views => 'views/layouts' }
\end{verbatim}

\item 
Templates are assumed to be located directly under the \verb|./views| directory. 

To use a different views directory: 

\begin{verbatim}
set :views, settings.root + '/templates'
\end{verbatim}

\item 
One important thing to remember is that you always have to reference
templates with symbols, even if they’re in a subdirectory (in this
case, use: 
\verb|:'subdir/template'| 
or 
\verb|'subdir/template'.to_sym|). 

You
must use a symbol because otherwise rendering methods will render
any strings passed to them directly.
\end{enumerate}

\section{Filtros}

\parrafo{Before Filters}
\cei{Before filters} are evaluated before each request within the
same context as the routes will be and can modify the request and
response. 

Instance variables set in filters are accessible by routes
and templates:

\begin{verbatim}
before do
  @note = 'Hi!'
  request.path_info = '/foo/bar/baz'
end

get '/foo/*' do
  @note #=> 'Hi!'
  params[:splat] #=> 'bar/baz'
end
\end{verbatim}

\parrafo{After Filters}
\cei{After filters} are evaluated after each request within the same
context and can also modify the request and response. 

Instance
variables set in before filters and routes are accessible by after
filters:

\begin{verbatim}
after do
  puts response.status
end
\end{verbatim}
Note: Unless you use the \verb|body| method rather than just returning a
String from the routes, the body will not yet be available in the
after filter, since it is generated later on.

\parrafo{Filters can take a Pattern}
Filters optionally take a pattern, causing them to be evaluated
only if the request path matches that pattern:
\begin{verbatim}
before '/protected/*' do
  authenticate!
end

after '/create/:slug' do |slug|
  session[:last_slug] = slug
end
\end{verbatim}

\parrafo{Filters can take a Condition}
Like routes, filters also take conditions:
\begin{verbatim}
before :agent => /Songbird/ do
  # ...
end

after '/blog/*', :host_name => 'example.com' do
  # ...
end
\end{verbatim}

\section{Manejo de Errores}

\begin{enumerate}
\item 
The HTTP specification states that response status in the range  200-299
indicate success in processing a request
\item 
500-599 is reserved for server errors
\item 
Sinatra  offers helpers for the 404 (Not Found) and 500 (Internal Server Error) status
\end{enumerate}

\parrafo{not\_found}
When a \verb|Sinatra::NotFound| exception is raised, or the response’s
status code is 404, the \verb|not_found| handler is invoked:

\begin{verbatim}
not_found do
  'This is nowhere to be found.'
end
\end{verbatim}

\parrafo{error}
The \verb|error| handler is invoked any time an exception is raised from
a route block or a filter. 

The exception object can be obtained
from the \verb|sinatra.error| Rack variable:

\begin{verbatim}
error do
  'Sorry there was a nasty error - ' + env['sinatra.error'].name
end
\end{verbatim}
Custom errors:

\begin{verbatim}
error MyCustomError do
  'So what happened was...' + env['sinatra.error'].message
end
\end{verbatim}
Then, if this happens:

\begin{verbatim}
get '/' do
  raise MyCustomError, 'something bad'
end
\end{verbatim}
You get this:

\begin{verbatim}
So what happened was... something bad
\end{verbatim}
Alternatively, you can install an error handler for a status code:

\begin{verbatim}
error 403 do
  'Access forbidden'
end

get '/secret' do
  403
end
\end{verbatim}
Or a range:

\begin{verbatim}
error 400..510 do
  'Boom'
end
\end{verbatim}
Sinatra installs special \verb|not_found| and \verb|error| handlers 
when running under the development environment to display nice stack
traces and additional debugging information in your browser (esto es,
en producción estos handlers son mucho mas "parcos").


\section{The methods body, status and headers}
\begin{enumerate}
\item 
It is possible and recommended to set the status code and response
body with the return value of the route block. 
\item 
However, in some
scenarios you might want to set the body at an arbitrary point in
the execution flow. 
\item 
You can do so with the \verb|body| helper method. 
\item 
If
you do so, you can use that method from there on to access the body:
\begin{verbatim}
get '/foo' do
  body "bar"
end

after do
  puts body
end
\end{verbatim}
It is also possible to pass a block to \verb|body|, which will be executed
by the Rack handler (this can be used to implement \cei{streaming}).

\item 
Similar to the \verb|body|, you can also set the \verb|status| code and \verb|headers|:

\begin{verbatim}
get '/foo' do
  status 418
  headers \
    "Allow"   => "BREW, POST, GET, PROPFIND, WHEN",
    "Refresh" => "Refresh: 20; http://www.ietf.org/rfc/rfc2324.txt"
  body "I'm a tea pot!"
end
\end{verbatim}
\item 
Like \verb|body|, \verb|headers| and \verb|status| with no arguments
can be used to access their current values.
\end{enumerate}


\section{Acceso al Objeto Request}
\label{section:requestobject}
El objeto que representa la solicitud {\it the request object}
es un hash con información de la solicitud: quien hizo la petición, 
que versión de HTTP usar, etc.

El objeto que representa la solicitud 
puede ser accedido desde el nivel de solicitud:
filtros, rutas y manejadores de error.

Véase \htmladdnormallink{https://github.com/crguezl/sinatra\_intro/blob/master/accesing\_the\_request\_object.rb}{https://github.com/crguezl/sinatra\_intro/blob/master/accesing\_the\_request\_object.rb}
% Dropbox/src/ruby/sinatra/intro/accesing_the_request_object.rb

\begin{rawhtml}
<img src="request_object.png" />
\end{rawhtml}


\section{Caching / Caches}
Mediante el uso del helper \verb|headers|podemos establecer los headers que queramos
para influir sobre la forma en la que ocurre el caching downstream.

\begin{enumerate}
\item \htmladdnormallink{Caching Tutorial}{http://www.mnot.net/cache\_docs/}
\end{enumerate}

\section{Sesiones y Cookies en Sinatra}

\parrafo{Introducción}
A session is used to keep state during requests. If activated, you
have one session hash per user session:

\begin{verbatim}
enable :sessions

get '/' do
  "value = " << session[:value].inspect
end

get '/:value' do
  session[:value] = params[:value]
end
\end{verbatim}
\begin{enumerate}
\item 
Note that \verb|enable :sessions| actually stores all data in a cookie
\item   This might not always be what you want (storing lots of data will increase your traffic, for instance)
\item  You can use any Rack session middleware: in order to do so, do not call \verb|enable :sessions|, but instead pull in your middleware of choice as you would any other middleware:

\begin{verbatim}
use Rack::Session::Pool, :expire_after => 2592000

get '/' do
  "value = " << session[:value].inspect
end

get '/:value' do
  session[:value] = params[:value]
end
\end{verbatim}
\item 
To improve security, the session data in the cookie is signed with a session secret
\item  A random secret is generated for you by Sinatra
\item  However, since this secret will change with every start of your application, you might want to set the secret yourself, so all your application instances share it:
\begin{verbatim}
set :session_secret, 'super secret'
\end{verbatim}
If you want to configure it further, you may also store a hash with options in the sessions setting:
\begin{verbatim}
set :sessions, :domain => 'foo.com'
\end{verbatim}
\item 
Just use
\verb|session.clear| to destroy the session.
\begin{verbatim}
get '/login' do
    session[:username] = params[:username]
    "logged in as #{session[:username]}"
  end

  get '/logout' do
    old_user = session[:username]
    session.clear
    "logged out #{old_user}"
  end
\end{verbatim}
\end{enumerate}


\parrafo{Cookies}
\begin{enumerate}
\item 
According to the Computer Science definition, a cookie, which is also known as an HTTP cookie, a tracking cookie, or a browser cookie, is a piece of text, no bigger than 4 kilobytes, which is stored on the user’s computer by a web server via a web browser
\item  It is a key-value pair structure, which is designed to retain specific information such as user preferences, user authentication, shopping carts, demographics, sessions, or any other data used by a website
\item  This mechanism, which was developed by Netscape in the distant 1994, provides a way to receive information from a web server and to send it back from the web browser absolutely unchanged
\item  This system complements the stateless nature of the HTTP protocol as it provides enough memory to store pieces of information during HTTP transactions

\item 
When you try to access a web site, your web browser connects to a web server and it sends a request for the respective page
\item  Then the web server replies by sending the requested content and it simultaneously stores a new cookie on your computer
\item  Every time the web browser requests web pages from the web server, it always sends the respective cookies back to the web server
\item  The process takes place as described, if the web browser supports cookies and the user allows their usage
\item  Only the web server can modify one or more of the cookie values
\item  Then it sends them to the web browser upon replying to a specific request

\item According to the RFC2965 specification, cookies are case insensitive
\item  A set of defined properties is inherent to the cookie structure  Those properties include:
 an expiration date, a path and a domain
\item  The first attribute requires a date defined in \verb|Wdy, DD-Mon-YYYY HH:MM:SS GMT| format
\item  The rest of the cookie characteristics require a \verb|path| and/or a \verb|domain| defined as a string
\item  Let’s take a look at this example:

\begin{verbatim}
Cookie: key0=value0; ...; keyX=valueX; expires=Wed, 23-Sep-2009 23:59:59 GMT; path=/; domain=.yoursite.com
\end{verbatim}

\item 
When the \verb|expiration| date is defined, your cookie will be \cei{persistent} as it will reoccur in different sessions until the set \verb|expiration| date has been reached
\item  If the \verb|expiration| date has not been defined in the cookie, it will occur until the end of your current session or when you close your web browser
\item  If the \verb|path| and/or the \verb|domain| attributes have been defined in your cookie, then the web server limits the scope of the cookie to that specific \verb|domain|, \verb|sub-domain| or \verb|path|
\end{enumerate}

\parrafo{Ejemplo con Sesiones}
\begin{verbatim}
require 'rubygems'
require 'sinatra'
require 'haml'

enable :sessions

get '/' do
  session["user"] ||= nil
  haml :index
end

get '/introduction' do
  haml :introduction
end

post '/introduction' do
  session["user"] = params[:name]
  redirect '/'
end

get '/bye' do
  session["user"] = nil
  haml :bye
end
\end{verbatim}

\parrafo{Ejemplo con Cookies}
\begin{enumerate}
\item 
The last example will demonstrate how to directly manage cookies through the 
\verb|request| and \verb|response| singletons provided by Sinatra
\item  You will see in the following example that the previously described process involving the use cookies is clearly implemented
\item  This technique is recommended when your application requires to use persistent and/or scoped cookies
\item  In this example, the application uses two persistent cookies, which expire at the same time, in order to store and manage different configuration data
\end{enumerate}

\begin{verbatim}
require 'sinatra'
require 'haml'

get '/' do
  @@expiration_date = Time.now + (60 * 2) \
  unless request.cookies.key?('some_options') && request.cookies.key?('other_options')
  haml :index
end

get '/some_options' do
  @some_cookie = request.cookies["some_options"]
  haml :some_options
end

post '/some_options' do  
  response.set_cookie('some_options', :value => cookie_values(params), :expires => @@expiration_date)
  redirect '/'
end

get '/other_options' do
  @other_cookie = request.cookies["other_options"]
  haml :other_options
end

post '/other_options' do
  response.set_cookie('other_options', :value => cookie_values(params),:expires => @@expiration_date)
  redirect '/'
end

helpers do
  def cookie_values(parameters)
    values = {}
    parameters.each do |key, value|
      case key
      when 'options'
        values[value] = true
      else
        values[key] = true
      end
    end
    values
  end
end
\end{verbatim}

\parrafo{Problemas}
\begin{enumerate}
\item 
\htmladdnormallink{I'm not sure why but my session gets wiped out every request?}{http://stackoverflow.com/questions/8957594/sinatra-session-gets-destroyed-automatically}
\item 
To keep sessions consistent you need to set a session secret, e.g.:

\begin{verbatim}
set :session_secret, 'super secret'
\end{verbatim}
When it's not set sinatra generates random one on application start
and shotgun restarts application before every request.
\end{enumerate}

\parrafo{Véanse}

\begin{enumerate}
\item 
\htmladdnormallink{Daily Ruby Tips \#60 – Simple Use of Sessions in Sinatra
May 6, 2013}{http://brettu.com/ruby-daily-ruby-tips-60-simple-use-of-sessions-in-sinatra/}
\item 
La sección {\it Cookies} en Rack \ref{section:cookies}.
\item 
\htmladdnormallink{Cookie-based Sessions in Sinatra
by JULIO JAVIER CICCHELLI on SEPTEMBER 30, 2009}{http://rubylearning.com/blog/2009/09/30/cookie-based-sessions-in-sinatra/}
RubyLearning Blog. El código está en
\htmladdnormallink{un Gist en GitHub}{http://gist.github.com/205962}
\end{enumerate}





\section{Downloads / Descargas / Attachments}


\parrafo{Usando attachment}

There is a built-in \verb|attachment| method
that optionally takes a filename parameter. If the filename has
an extension (\verb|.jpg|, etc.) that extension 
will be used to determine the \verb|Content-Type| header for the response.

The evaluation of the route will provide the contents of the attachment.

\begin{enumerate}
\item 
\htmladdnormallink{Documentación de attachment}{http://rubydoc.info/github/sinatra/sinatra/Sinatra/Helpers\#attachment-instance_method}
\item 
\htmladdnormallink{Código de attachment}{https://github.com/sinatra/sinatra/blob/master/lib/sinatra/base.rb\#L338-L348} en GitHub
\item 
\htmladdnormallink{Upload and download files in Sinatra}{http://alfuken.tumblr.com/post/874428235/upload-and-download-files-in-sinatra}
Random Ruby Thoughts
\end{enumerate}

\begin{verbatim}
[~/sinatra/sinatra-download(master)]$ cat app.rb
require 'sinatra'

before do 
  content_type :txt
end

get '/attachment?' do
  attachment 'file.txt'
  "Here's what will be sent downstream, in an attachment called 'file.txt'." 
end
\end{verbatim}
Cuando visitamos la página con el navegador se nos abre una ventana para la descarga de
un fichero que será guardado (por defecto) como \verb|file.txt|.

Los contenidos de ese fichero serán:
\begin{verbatim}
[~/sinatra/sinatra-download(master)]$ cat ~/Downloads/file.txt 
Here's what will be sent downstream, in an attachment called 'file.txt'.
\end{verbatim}

\parrafo{Usando send\_file}

Véase la documentación del módulo 
\sinatra{Sinatra::Streaming}{Sinatra/Streaming.html\#M000022}

\begin{verbatim}
[~/sinatra/sinatra-download(master)]$ cat sending_file.rb 
require 'sinatra'

get '/' do
   send_file 'foo.png', 
              :type => 'img/png', 
              :disposition => 'attachment', 
              :filename =>'tutu.png',
              :stream => false
end
\end{verbatim}

The options are:

\begin{enumerate}
\item 
\verb|filename|
file name, in response, defaults to the real file name.
\item 
\verb|last_modified|
value for Last-Modified header, defaults to the file's \verb|mtime|.
\item 
\verb|type|
content type to use, guessed from the file extension if missing.
\item 
\verb|disposition|
used for \verb|Content-Disposition|, possible value: \verb|nil| (default), \verb|:attachment| and 
\verb|:inline|
\item 
\verb|length|
\verb|Content-Length| header, defaults to file size.
\item 
\verb|status|
Status code to be send. 

Useful when sending a static file as an
error page. 
If supported by the Rack handler, other means than
streaming from the Ruby process will be used. If you use this helper
method, Sinatra will automatically handle range requests.
\end{enumerate}

\section{Uploads. Subida de Ficheros en Sinatra}

\parrafo{Véase}
\begin{enumerate}
\item 
\htmladdnormallink{El repositorio sinatra-upload en GitHub}{https://github.com/crguezl/sinatra-upload}
con el código de este ejemplo
\item 
\htmladdnormallink{FILE UPLOAD WITH SINATRA}{http://www.wooptoot.com/file-upload-with-sinatra} BY PANDAFOX POSTED IN RUBY, TUTORIALS
\item 
\htmladdnormallink{Multiple file uploads in Sinatra}{http://stackoverflow.com/questions/18860656/multiple-file-uploads-in-sinatra}
\end{enumerate}

\parrafo{Jerarquía de ficheros}
\begin{verbatim}
[~/sinatra/sinatra-upload]$ tree
.
|-- app.rb
|-- uploads
|   `--- README
`--- views
    `--- upload.haml

2 directories, 3 files
\end{verbatim}

\parrafo{upload.haml}

The important part is not to forget to set the \verb|enctype| 
in your form element, otherwise you will just get the filename instead of an object:

\begin{verbatim}
[~/sinatra/sinatra-upload(master)]$ cat views/upload.haml 
%html
  %body
    %h1 File uploader!
    %form(method="post" enctype='multipart/form-data')   
      %input(type='file' name='myfile')    
      %br
      %input(type='submit' value='Upload!')
\end{verbatim}

\parrafo{app.rb}
\begin{verbatim}
[~/sinatra/sinatra-upload(master)]$ cat app.rb 
require 'rubygems'
require 'sinatra'
require 'haml'
require 'pp'
 
# Handle GET-request (Show the upload form)
get "/upload?" do
  haml :upload
end      
    
# Handle POST-request (Receive and save the uploaded file)
post "/upload" do 
  pp params
  File.open('uploads/' + params['myfile'][:filename], "w") do |f|
    f.write(params['myfile'][:tempfile].read)
  end
  return "The file was successfully uploaded!"
end
[~/sinatra/sinatra-upload(master)]$ 
\end{verbatim}

As you can see, you don’t have to write much code to get this to
work. The \verb|params|-hash contains our uploaded element with data such
as filename, type and the actual datafile, which can be accessed
as a Tempfile-object. 

We read the contents from this file and store
it into a directory called uploads, which you will have to create
before running this script.

Here’s an example of what the \verb|params|-hash may look like when uploading
a picture of a cat:

\begin{verbatim}
{
  "myfile" => {
    :type => "image/png",
    :head =>  "Content-Disposition: form-data;
               name=\"myfile\";
               filename=\"cat.png\"\r\n
               Content-Type: image/png\r\n",
    :name => "myfile",
    :tempfile => #<File:/var/folders/3n/3asd/-Tmp-/RackMultipart201-1476-nfw2-0>,
    :filename=>"cat.png"
  }
}  
\end{verbatim}

\htmladdnormallink{File upload with sinatra. YouTube}{http://youtu.be/7tkQUTUxJWE}

\parrafo{BEWARE!}

This script offers little to no security at all. Clients will be able to overwrite old images, fill up your harddrive and so on. So just use some common sense and do some Ruby magic to patch up the security holes yourself.

\section{halt}

Sometimes we want to stop the program: maybe a critical error has ocurred.
To immediately stop a request within a filter or route use:

\begin{verbatim}
halt
\end{verbatim}
You can also specify the status when halting:

\begin{verbatim}
halt 410
\end{verbatim}
Or the body:

\begin{verbatim}
halt 'this will be the body'
\end{verbatim}
Or both:

\begin{verbatim}
halt 401, 'go away!'
\end{verbatim}
With headers:

\begin{verbatim}
halt 402, {'Content-Type' => 'text/plain'}, 'revenge'
\end{verbatim}
It is of course possible to combine a template with halt:

\begin{verbatim}
halt erb(:error)
\end{verbatim}

\section{Passing a Request}
When we want to pass processing 
to the next matching route we use
\verb|pass|:

\begin{verbatim}
get '/guess/:who' do
  pass unless params[:who] == 'Frank'
  'You got me!'
end

get '/guess/*' do
  'You missed!'
end
\end{verbatim}
The route block is immediately exited and control continues with
the next matching route. 

If no matching route is found, a 404 is
returned.

\section{Triggering Another Route: calling {\tt call}}
Sometimes \verb|pass| is not what you want, 
instead you would like to get the result of calling another route. 
Simply use \verb|call| to achieve this:

\begin{verbatim}
get '/foo' do
  status, headers, body = call env.merge("PATH_INFO" => '/bar')
  [status, headers, body.map(&:upcase)]
end

get '/bar' do
  "bar"
end
\end{verbatim}

Note that in the example above, you would ease testing and increase performance by simply moving \verb|"bar"| into a helper used by both \verb|/foo| and \verb|/bar|.

If you want the request to be sent to the same application instance rather than a duplicate, use \verb|call!| instead of \verb|call|.

Check out the Rack specification if you want to learn more about \verb|call|.

\section{Logging}
In the request scope, the \verb|logger| helper exposes a \verb|Logger| instance:

\begin{verbatim}
get '/' do
  logger.info "loading data"
  # ...
end
\end{verbatim}
\begin{enumerate}
\item 
This \verb|logger| will automatically take your Rack handler’s \verb|logging| settings into account
\item  If \verb|logging| is disabled, this method will return a dummy object, so you do not have to worry in your routes and filters about it
\end{enumerate}

Note that \verb|logging| is only enabled for \verb|Sinatra::Application|
 by default, so if you inherit from \sinatrabase{}, you probably want to enable it yourself:

\begin{verbatim}
class MyApp < Sinatra::Base
  configure :production, :development do
    enable :logging
  end
end
\end{verbatim}
\begin{enumerate}
\item 
To avoid any \verb|logging| middleware to be set up, set the \verb|logging| setting to \verb|nil|
\item  However, keep in mind that \verb|logger| will in that case return \verb|nil|
\item  A common use case is when you want to set your own \verb|logger|.
Sinatra will use whatever it will find in \verb|env['rack.logger']|
\end{enumerate}

\parrafo{Logging a stdout y a un fichero}

Véase 
\htmladdnormallink{Rack::CommonLogger}{http://recipes.sinatrarb.com/p/middleware/rack\_commonlogger}
en Sinatra Recipes.

Sinatra has logging support, but it's nearly impossible to log to a file and to the stdout (like Rails does).

However, there is a little trick you can use to log to stdout and to a file:

\begin{verbatim}
require 'sinatra'

configure do
  # logging is enabled by default in classic style applications,
  # so `enable :logging` is not needed
  file = File.new("#{settings.root}/log/#{settings.environment}.log", 'a+')
  file.sync = true
  use Rack::CommonLogger, file
end

get '/' do
  'Hello World'
end
\end{verbatim}
You can use the same configuration for modular style applications,
but you have to \verb|enable :logging first|:

\begin{verbatim}
require 'sinatra/base'

class SomeApp < Sinatra::Base
  configure do
    enable :logging
    file = File.new("#{settings.root}/log/#{settings.environment}.log", 'a+')
    file.sync = true
    use Rack::CommonLogger, file
  end

  get '/' do
    'Hello World'
  end

  run!
end
\end{verbatim}

\parrafo{Ejecución}
\begin{verbatim}
~/sinatra/sinatra-logging]$ tree
.
|-- app.rb
`-- log

1 directory, 1 file
[~/sinatra/sinatra-logging]$ ruby app.rb 
== Sinatra/1.4.4 has taken the stage on 4567 for development with backup from Thin
Thin web server (v1.6.1 codename Death Proof)
Maximum connections set to 1024
Listening on localhost:4567, CTRL+C to stop
\end{verbatim}
Consola después de visitar la página:
\begin{verbatim}
127.0.0.1 - - [19/Nov/2013 14:53:06] "GET / HTTP/1.1" 200 11 0.0041
\end{verbatim}

Fichero después de visitar la página:
\begin{verbatim}
[~/sinatra/sinatra-logging]$ cat log/development.log 
127.0.0.1 - - [19/Nov/2013 14:53:06] "GET / HTTP/1.1" 200 11 0.0038
\end{verbatim}

\parrafo{Véase}
\begin{enumerate}
\item 
 el código en GitHub de 
\htmladdnormallink{Rack::CommonLogger}{https://github.com/rack/rack/blob/master/lib/rack/commonlogger.rb}
\item Logging in Sinatra. StackOverflow.
\htmladdnormallink{Destination is set by  changing 
}{http://stackoverflow.com/questions/5995854/logging-in-sinatra/5995932\#5995932}
\verb|env['rack.errors']|.  Konstantin Haase May 13 '11 at 21:18'
\end{enumerate}

\section{Generating URLs}
For generating URLs you should use the \verb|url|
helper method, for instance, in Haml:

\begin{verbatim}
%a{:href => url('/foo')} foo
\end{verbatim}
It takes reverse proxies and Rack routers into account, if present.

This method is also aliased to \verb|to|.


\section{Redireccionamientos/Browser Redirect}

You can trigger a browser redirect with the \verb|redirect| helper method:

\begin{verbatim}
get '/foo' do
  redirect to('/bar')
end
\end{verbatim}
Any additional parameters are handled like arguments passed to \verb|halt|:

\begin{verbatim}
redirect to('/bar'), 303
redirect 'http://google.com', 'wrong place, buddy'
\end{verbatim}
You can also easily \verb|redirect| back to the page the user came from with \verb|redirect back|:

\begin{verbatim}
get '/foo' do
  "<a href='/bar'>do something</a>"
end

get '/bar' do
  do_something
  redirect back
end
\end{verbatim}
To pass arguments with a \verb|redirect|, either add them to the query:

\begin{verbatim}
redirect to('/bar?sum=42')
\end{verbatim}
Or use a session:

\begin{verbatim}
enable :sessions

get '/foo' do
  session[:secret] = 'foo'
  redirect to('/bar')
end

get '/bar' do
  session[:secret]
end
\end{verbatim}

\section{Configuration / Configuración}
Run once, at startup, in any environment:

\begin{verbatim}
configure do
  # setting one option
  set :option, 'value'

  # setting multiple options
  set :a => 1, :b => 2

  # same as `set :option, true`
  enable :option

  # same as `set :option, false`
  disable :option

  # you can also have dynamic settings with blocks
  set(:css_dir) { File.join(views, 'css') }
end
\end{verbatim}
Run only when the environment (\verb|RACK_ENV|
 environment variable) is set to :\verb|production|:

\begin{verbatim}
configure :production do
  ...
end
\end{verbatim}
Run when the environment is set to either \verb|:production| or \verb|:test|:

\begin{verbatim}
configure :production, :test do
  ...
end
\end{verbatim}
You can access those options via settings:

\begin{verbatim}
configure do
  set :foo, 'bar'
end

get '/' do
  settings.foo? # => true
  settings.foo  # => 'bar'
  ...
end
\end{verbatim}

\section{Configuring attack protection}
Sinatra is using \rackprotection{}
to defend your application against common, opportunistic attacks.

\begin{enumerate}
\item 
You can easily disable this behavior (which will open up your
application to tons of common vulnerabilities):

\verb|disable :protection|
\item 
To skip a single defense layer, set protection to an options hash:

\begin{verbatim}
set :protection, :except => :path_traversal
\end{verbatim}
\item 
You can also hand in an array in order to disable a list of protections:

\begin{verbatim}
set :protection, :except => [:path_traversal, :session_hijacking]
\end{verbatim}
\item 
By default, Sinatra will only set up session based protection if
\verb|:sessions| has been enabled. 

Sometimes you want to set up sessions
on your own, though. 

In that case you can get it to set up session
based protections by passing the \verb|:session option|:

\begin{verbatim}
use Rack::Session::Pool
set :protection, :session => true
\end{verbatim}
\end{enumerate}

\section{Settings disponibles/Available Settings}
\begin{enumerate}
\item 
\verb|absolute_redirects|
If disabled, Sinatra will allow relative redirects, however, Sinatra will no longer conform with RFC 2616 (HTTP 1.1), which only allows absolute redirects.

Enable if your app is running behind a reverse proxy that has not been set up properly. Note that the \verb|url| helper will still produce absolute URLs, unless you pass in \verb|false| as the second parameter.
Disabled by default.

\item 
\verb|add_charsets|
mime types the \verb|content_type| helper will automatically add the charset info to. 

You should add to it rather than overriding this option: 
\begin{verbatim}
settings.add_charsets << "application/foobar"
\end{verbatim}
\item 
\verb|app_file|
Path to the main application file, used to detect project root, views and public folder and inline templates.
\item 
\verb|bind|
IP address to bind to (default: 0.0.0.0 or localhost if your \verb|environment| is set to 
\verb|development|). Only used for built-in server.
\item 
\verb|default_encoding|
encoding to assume if unknown (defaults to "utf-8").
\item 
\verb|dump_errors|
display errors in the log.
\item 
\verb|environment|
current environment, defaults to \verb|ENV['RACK_ENV']|, 
or \verb"development" if not available.
\item 
\verb|logging|
use the logger.
\item 
\verb|lock|
Places a lock around every request, only running processing on request per Ruby process concurrently.
Enabled if your app is not thread-safe. Disabled per default.
\item 
\verb|method_override|
use \verb|_method magic| to allow put/delete forms in browsers that don't support it.
\item 
\verb|port|
Port to listen on. Only used for built-in server.
\item \verb|prefixed_redirects|
Whether or not to insert \verb|request.script_name| into redirects if no absolute path is given. That way redirect '/foo' would behave like redirect to('/foo'). Disabled per default.
\item \verb|protection|
Whether or not to enable web attack protections. See protection section above.
\item \verb|public_dir|
Alias for \verb|public_folder|. See below.
\item \verb|public_folder|
Path to the folder public files are served from. Only used if static file serving is enabled (see static setting below). Inferred from \verb|app_file| setting if not set.
\item \verb|reload_templates|
Whether or not to reload templates between requests. Enabled in development mode.
\item \verb|root|
Path to project root folder. Inferred from \verb|app_file| setting if not set.
\item \verb|raise_errors|
raise exceptions (will stop application). Enabled by default when environment is set to "test", disabled otherwise.
\item \verb|run|
if enabled, Sinatra will handle starting the web server, do not enable if using rackup or other means.
\item \verb|running|
is the built-in server running now? do not change this setting!
\item \verb|server|
Server or list of servers to use for built-in server. order indicates priority, default depends on Ruby implementation.
\item \verb|sessions|
Enable cookie-based sessions support using Rack::Session::Cookie. See 'Using Sessions' section for more information.
\item \verb|show_exceptions|
Show a stack trace in the browser when an exception happens. Enabled by default when environment is set to "development", disabled otherwise.
Can also be set to \verb|:after_handler| to trigger app-specified error handling before showing a stack trace in the browser.
\item \verb|static|
Whether Sinatra should handle serving static files.
Disable when using a server able to do this on its own.
Disabling will boost performance.
Enabled per default in classic style, disabled for modular apps.
\item \verb|static_cache_control|
When Sinatra is serving static files, set this to add Cache-Control headers to the responses. 
Uses the \verb|cache_control| helper. Disabled by default.
Use an explicit array when setting multiple values: 
\begin{verbatim}
set :static_cache_control, [:public, :max_age => 300]
\end{verbatim}
\item \verb|threaded|
If set to true, will tell Thin to use EventMachine.defer for processing the request.
\item \verb|views|
Path to the views folder. Inferred from \verb|app_file| setting if not set.
\item \verb|x_cascade|
Whether or not to set the X-Cascade header if no route matches. Defaults to true.
\end{enumerate}

\section{Environments}
\begin{enumerate}
\item 
There are three predefined environments:
\tei{development}, \tei{production} and \tei{test}. 

\item 
Environments can be set through the \verb|RACK_ENV| environment variable. 

\item 
The default value is \verb"development"
\item  In the \verb"development" environment all templates are
reloaded between requests, and special \verb|not_found| and 
\verb|error| handlers
display stack traces in your browser
\item  In the \verb"production" and \verb"test" environments, templates are cached by default
\item 
To run different environments, set the \verb|RACK_ENV| environment variable:

\begin{verbatim}
RACK_ENV=production ruby my_app.rb
\end{verbatim}
\item 
You can use predefined methods: \verb|development?|, \verb|test?| 
and \verb|production?| to check the current environment setting:

\begin{verbatim}
get '/' do
  if settings.development?
    "development!"
  else
    "not development!"
  end
end
\end{verbatim}
\end{enumerate}

\section{Correo}
\label{section:correo}
\begin{verbatim}
[~/srcSTW/sinatra-faq/mail(esau)]$ cat app.rb 
require 'sinatra'
require 'pony'
raise "Execute:\n\t#{$0} password email_to email_from" if ARGV.length.zero?
get '/' do
    email = ARGV.shift
    pass = ARGV.shift
    Pony.mail({
      :to => email,
      :body => "Hello Casiano",
      :subject => 'Howdy, Partna!',
      :via => :smtp,
      :via_options => {
          :address              => 'smtp.gmail.com',
          :port                 => '587',
          :enable_starttls_auto => true,
          :user_name            => ARGV.shift,
          :password             => pass,
          :authentication       => :plain, # :plain, :login, :cram_md5, no auth by default
          :domain               => "localhost.localdomain" # the HELO domain provided by the client to the server
        }
    })
    "Check your email at #{email}"
end

\end{verbatim}

\begin{enumerate}
\item 
\htmladdnormallink{Getting started with Sinatra}{http://www.millwoodonline.co.uk/blog/getting-started-with-sinatra}
\end{enumerate}

\section{Ambito}
The scope you are currently in determines what methods and variables are available.

\parrafo{Ámbito de Clase/Class Scope}

\begin{enumerate}
\item 
Every Sinatra application corresponds to a subclass of \sinatrabase{Sinatra::Base}
\item  If you are using the top-level DSL (\verb|require 'sinatra'|), then this class is \sinatraapplication{Sinatra::Application}, otherwise it is the subclass you created explicitly
\item  At class level you have methods like \verb|get| or \verb|before|, 
but you cannot access the \verb|request| or \verb|session| objects, 
as there is only a single application class for all requests
\end{enumerate}

Options created via \verb|set| are methods at class level:

\begin{verbatim}
class MyApp < Sinatra::Base
  # Hey, I'm in the application scope!
  set :foo, 42
  foo # => 42

  get '/foo' do
    # Hey, I'm no longer in the application scope!
  end
end
\end{verbatim}

You have the application scope binding inside:

\begin{enumerate}
\item 
Your application class body
\item 
Methods defined by extensions
\item 
The block passed to \verb|helpers|
\item 
Procs/blocks used as value for \verb|set|
\item 
The block passed to \verb|Sinatra.new|
\end{enumerate}

You can reach the scope object (the class) like this:

\begin{enumerate}
\item 
Via the object passed to configure blocks (\verb:configure { |c| ... }:)
\item 
\verb|settings| from within the request scope
\end{enumerate}

\parrafo{Ámbito de Instancia/Instance Scope}

For every incoming request, a new instance of your application class is created and all handler blocks run in that scope
\begin{enumerate}
\item  From within this scope you can access the \verb|request| and \verb|session| objects or 
\item 
call rendering methods like \verb|erb| or \verb|haml|
\item  You can access the application scope from within the \verb|request| scope via the \verb|settings| helper:
\end{enumerate}

\begin{verbatim}
[~/sinatra/sinatra-scope]$ cat app.rb 
require 'sinatra'

class MyApp < Sinatra::Base
  # Hey, I'm in the application scope!
  get '/define_route/:name' do
    # Request scope for '/define_route/:name'
    @value = 42
    puts "Inside /define_route/:name @value = #{@value}"
    puts self.class

    settings.get("/#{params[:name]}") do
      # Request scope for "/#{params[:name]}"
      puts "@value = <#{@value}>"
      "Inside defined route #{params[:name]}"
    end

    "Route #{params[:name]} defined!"
  end
  
  run! if __FILE__ == $0
end
\end{verbatim}

\parrafo{Ejecución en el servidor}
\begin{verbatim}
[~/sinatra/sinatra-scope]$ ruby app.rb 
== Sinatra/1.4.4 has taken the stage on 4567 for development with backup from Thin
Thin web server (v1.6.1 codename Death Proof)
Maximum connections set to 1024
Listening on localhost:4567, CTRL+C to stop
Inside /define_route/:name @value = 42
MyApp
@value = <>

\end{verbatim}

\parrafo{Ejecución en el cliente. Ruta: /define\_route/juan}
\begin{verbatim}
[~/sinatra/sinatra-scope]$ curl -v 'http://localhost:4567/define_route/juan'
* Adding handle: conn: 0x7fbacb004000
* Adding handle: send: 0
* Adding handle: recv: 0
* Curl_addHandleToPipeline: length: 1
* - Conn 0 (0x7fbacb004000) send_pipe: 1, recv_pipe: 0
* About to connect() to localhost port 4567 (#0)
*   Trying 127.0.0.1...
* Connected to localhost (127.0.0.1) port 4567 (#0)
> GET /define_route/juan HTTP/1.1
> User-Agent: curl/7.30.0
> Host: localhost:4567
> Accept: */*
> 
< HTTP/1.1 200 OK
< Content-Type: text/html;charset=utf-8
< Content-Length: 19
< X-XSS-Protection: 1; mode=block
< X-Content-Type-Options: nosniff
< X-Frame-Options: SAMEORIGIN
< Connection: keep-alive
* Server thin 1.6.1 codename Death Proof is not blacklisted
< Server: thin 1.6.1 codename Death Proof
< 
* Connection #0 to host localhost left intact
Route juan defined!
[~/sinatra/sinatra-scope]$
\end{verbatim}

\parrafo{Ejecución en el cliente. Ruta: /juan}
\begin{verbatim}
[~/sinatra/sinatra-scope]$ curl -v 'http://localhost:4567/juan'
* Adding handle: conn: 0x7fbdd1800000
* Adding handle: send: 0
* Adding handle: recv: 0
* Curl_addHandleToPipeline: length: 1
* - Conn 0 (0x7fbdd1800000) send_pipe: 1, recv_pipe: 0
* About to connect() to localhost port 4567 (#0)
*   Trying ::1...
*   Trying 127.0.0.1...
* Connected to localhost (127.0.0.1) port 4567 (#0)
> GET /juan HTTP/1.1
> User-Agent: curl/7.30.0
> Host: localhost:4567
> Accept: */*
> 
< HTTP/1.1 200 OK
< Content-Type: text/html;charset=utf-8
< Content-Length: 21
< X-XSS-Protection: 1; mode=block
< X-Content-Type-Options: nosniff
< X-Frame-Options: SAMEORIGIN
< Connection: keep-alive
* Server thin 1.6.1 codename Death Proof is not blacklisted
< Server: thin 1.6.1 codename Death Proof
< 
* Connection #0 to host localhost left intact
Inside defined route
\end{verbatim}
You have the request scope binding inside:

\begin{enumerate}
\item 
\verb|get|, \verb|head|, \verb|post|, \verb|put|, \verb|delete|, \verb|options|, \verb|patch|, \verb|link|, and \verb|unlink| blocks
\item 
\verb|before| and \verb|after| filters
\item 
\verb|helper| methods
\item 
templates/views
\end{enumerate}

\section{Sinatra Authentication}
\label{section:authentication}


\subsection{Referencias}
\begin{enumerate}
\item \htmladdnormallink{Ejemplo de uso de sinatra-authentication}{https://github.com/maxjustus/sinatra-authentication/tree/master/example}
\end{enumerate}


\section{Autentificación Básica}
\label{section:basicauthentication}
\begin{verbatim}
[~/srcSTW/sinatra-faq/authentication/basic(esau)]$ cat app.rb 
require 'rubygems'
require 'sinatra'

use Rack::Auth::Basic, "Restricted Area" do |username, password|
  [username, password] == ['admin', 'admin']
end

get '/' do
  "You're welcome"
end

get '/foo' do
  "You're also welcome"
end

\end{verbatim}

\section{Sinatra como Middleware}

Not only is Sinatra able to use other Rack middleware, any Sinatra
application can in turn be added in front of any Rack endpoint as
middleware itself. 

This endpoint could be another Sinatra application,
or any other Rack-based application (Rails/Ramaze/Camping/…):


\begin{enumerate}
\item 
When a request comes in, all \verb|before| filters are triggered
\item 
Then, if a route matches, the corresponding block will be executed
\item 
If no route matches, the request is handed off to the wrapped application
\item 
The \verb|after| filters are executed after we've got a response back from the
route or  the wrapped app
\end{enumerate}
Thus, our Sinatra app is a middleware.

\begin{rawhtml}
<img src="rack_stack.png" />
\end{rawhtml}

\begin{verbatim}
[~/sinatra/sinatra-as-middleware]$ cat app.rb
require 'sinatra/base'
require 'haml'
require 'pp'

class LoginScreen < Sinatra::Base
  enable :sessions
  enable :inline_templates

  get('/login') { haml :login }

  post('/login') do
    if params[:name] == 'admin' && params[:password] == 'admin'
      puts "params = "
      pp params
      session['user_name'] = params[:name]
      redirect '/'
    else
      redirect '/login'
    end
  end
end

class MyApp < Sinatra::Base
  enable :inline_templates
  # middleware will run before filters
  use LoginScreen

  before do
    unless session['user_name']
      halt haml :denied
    end
  end

  get('/') do
    haml :cheer, :locals => { :name => session['user_name'] }
  end

  run!
end

__END__

@@ layout
!!!
%html
  %head
    %title Sinatra as Middleware
  %body
    %h1 Sinatra as Middleware
    = yield

@@ login
%form{:action=>'/login', :method=>'post'}
  %label{:for=>'name'} Name
  %input#name{:type=>"text",     :name=>"name", :autofocus => true }
  %br
  %label{:for=>'password'} Password
  %input#password{:type=>"password", :name=>"password"}
  %br
  %button#go{:type=>"submit",  :name=>"submit", :value=>"submit"} Click me!

@@ cheer
%h1
  Hello #{name}
  %br

@@ denied
%h1 
  Access denied, please 
  %a{:href=>'/login'}login.
\end{verbatim}
