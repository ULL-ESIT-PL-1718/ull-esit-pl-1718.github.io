
\parrafo{Ejemplo usando Ajax con jQuery y Express.js}

\htmladdnormallink{Código del server}{https://github.com/crguezl/how-jquery-works-tutorial/tree/getallparams}:

\begin{verbatim}
[~/javascript/jquery/how-jquery-works-tutorial(getallparams)]$ cat app.js
var express = require('express');
var app = express();
var path = require('path');

app.use(express.static('public'));

// view engine setup
app.set('views', path.join(__dirname, 'views'));
app.set('view engine', 'ejs');

app.get('/', function (req, res) {
  res.render('index', { title: 'Express' });
})

app.get('/chuchu', function (req, res) {
  var isAjaxRequest = req.xhr;
  console.log(isAjaxRequest);
  if (isAjaxRequest) {
    console.log(req.query);
    res.send('{"answer": "Server responds: hello world!"}')
  }
  else {
    res.send('not an ajax request');
  }
});

var server = app.listen(3000, function () {

  var host = server.address().address
  var port = server.address().port

  console.log('Example app listening at http://%s:%s', host, port)

});
\end{verbatim}

\begin{itemize}
\item \verb|jQuery.get( url [, data ] [, success(data, textStatus, jqXHR) ] [, dataType ] )|
load data from the server using a HTTP GET request.

\item \verb|url|

Type: String

A string containing the URL to which the request is sent.
\item \verb|data|

Type: PlainObject or String

A plain object or string that is sent to the server with the request.
\item \verb|success(data, textStatus, jqXHR)|

Type: Function()

A callback function that is executed if the request succeeds.
\item \verb|dataType|

Type: String

The type of data expected from the server. Default: Intelligent Guess 
(\verb|xml|, \verb|json|, \verb|script|, \verb|or| \verb|html|).
\end{itemize}

To use callbacks, it is important to know how to pass them into their parent function.

En el directorio \verb|views| hemos puesto el template:
\begin{verbatim}
[~/javascript/jquery/how-jquery-works-tutorial(getallparams)]$ cat views/index.ejs 
<!doctype html>
<html>
  <head>
    <title><%- title %></title>
  </head>
  <body>
    <h1><%- title  %></h1>
    <ul>
      <li><a href="http://jquery.com/" id="jq">jQuery</a>
      <li><a href="/chuchu">Visit chuchu!</a>
    </ul>
    <div class="result"></div>
    <script src="https://code.jquery.com/jquery-2.1.3.js"></script>
    <script>
      $( document ).ready(function() {
          $( "#jq" ).click(function( event ) {
              event.preventDefault();
              $.get( "/chuchu", {nombres: ["juan", "pedro"]}, function( data ) {
                $( ".result" ).html( data["answer"]);
                console.log(data);
              }, 'json');
          });
      });
    </script>
  </body>
</html>
\end{verbatim}
\begin{itemize}
\item
\verb|req.query|

An object containing a property for each query string parameter in the route. If there is no query string, it is the empty object, \verb|{}|.

\begin{verbatim}
// GET /search?q=tobi+ferret
req.query.q
// => "tobi ferret"

// GET /shoes?order=desc&shoe[color]=blue&shoe[type]=converse
req.query.order
// => "desc"

req.query.shoe.color
// => "blue"

req.query.shoe.type
// => "converse"
\end{verbatim}
\end{itemize}

Estas son las dependencias:
\begin{verbatim}
[~/javascript/jquery/how-jquery-works-tutorial(getallparams)]$ cat package.json 
{
  "name": "ajaxjquery",
  "version": "0.0.0",
  "description": "",
  "main": "hello.js",
  "dependencies": {
    "express": "*",
    "ejs": "*",
    "gulp-shell": "*",
    "body-parser": "~1.12.0"
  },
  "devDependencies": {},
  "scripts": {
    "test": "node hello.js"
  },
  "author": "",
  "license": "ISC"
}
\end{verbatim}
Además hemos instalado a nivel global \verb|gulp| y \verb|node-supervisor|.

Podemos arrancar el servidor usando este \verb|gulpfile|:

\begin{verbatim}
[~/javascript/jquery/how-jquery-works-tutorial(getallparams)]$ cat gulpfile.js 
var gulp    = require('gulp');
var shell = require('gulp-shell');

gulp.task('default', ['server']);

// npm install supervisor -g
gulp.task('server', function () {
  return gulp.src('').pipe(shell([ 'node-supervisor app.js' ]));
});

gulp.task('open', function() {
  return gulp.src('').
           pipe(shell("open https://github.com/crguezl/how-jquery-works-tutorial/tree/getallparams"));
});
\end{verbatim}

\parrafo{Ejemplo de como Desplegar una Aplicación  Express sobre Node.JS en Heroku}

Véase:
\begin{itemize}
\item
La rama heroku del repo \htmladdnormallink{how-jquery-works-tutorial}{https://github.com/crguezl/how-jquery-works-tutorial/tree/heroku}
\item
El tutorial de Heroku 
\htmladdnormallink{Getting Started with Node.js on Heroku}{https://devcenter.heroku.com/articles/getting-started-with-nodejs}
\item
El capítulo sobre Heroku en los apuntes de LPP
\end{itemize}

