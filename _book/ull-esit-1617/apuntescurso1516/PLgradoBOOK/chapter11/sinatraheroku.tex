\parrafo{Ajax, jQuery y Sinatra}

JavaScript enables you to freely
pass functions around to be executed at a later time. A \cei{callback} is a
function that is passed as an argument to another function and is usually 
executed
after its parent function has completed. 

Callbacks are special because
they wait to execute until their parent finishes or some event occurs. 

Meanwhile, the
browser can be executing other functions or doing all sorts of other work.
\begin{verbatim}
[~/javascript/jquery(master)]$ cat app.rb
require 'sinatra'

set :public_folder, File.dirname(__FILE__) + '/starterkit'

get '/' do
  erb :index
end

get '/chuchu' do
  if request.xhr?
    "hello world!"
  else 
    erb :tutu
  end
end

__END__

@@layout
  <!DOCTYPE html>
  <html>
    <head>
        <meta charset="utf-8" />
        <title>Demo</title>
    </head>
    <body>
        <a href="http://jquery.com/">jQuery</a>
        <div class="result"></div>
        <script src="jquery.js"></script>
        <%= yield %>
    </body>
  </html>

@@index
  <script>
  $( document ).ready(function() {
      $( "a" ).click(function( event ) {
          event.preventDefault();
          $.get( "/chuchu", function( data ) {
            $( ".result" ).html( data );
            alert( "Load was performed." );
          });
      });
  });
  </script>

@@tutu
  <h1>Not an Ajax Request!</h1>
\end{verbatim}

\begin{itemize}
\item \verb|jQuery.get( url [, data ] [, success(data, textStatus, jqXHR) ] [, dataType ] )|
load data from the server using a HTTP GET request.

\item \verb|url|

Type: String

A string containing the URL to which the request is sent.
\item \verb|data|

Type: PlainObject or String

A plain object or string that is sent to the server with the request.
\item \verb|success(data, textStatus, jqXHR)|

Type: Function()

A callback function that is executed if the request succeeds.
\item \verb|dataType|

Type: String

The type of data expected from the server. Default: Intelligent Guess 
(\verb|xml|, \verb|json|, \verb|script|, \verb|or| \verb|html|).
\end{itemize}

To use callbacks, it is important to know how to pass them into their parent function.


Executing callbacks with arguments can be tricky.

This code example will not work:

\begin{verbatim}
$.get( "myhtmlpage.html", myCallBack( param1, param2 ) );
\end{verbatim}
The reason this fails is that the code executes 

\begin{verbatim}
myCallBack( param1, param2) 
\end{verbatim}

immediately and then passes \verb|myCallBack()|'s return value as the second
parameter to \verb|$.get()|. 

We actually want to pass the function \verb|myCallBack|,
not \verb|myCallBack( param1, param2 )|'s return value (which might or might not
be a function). 

So, how to pass in \verb|myCallBack()| and include arguments?

To defer executing \verb|myCallBack()| with its parameters, you can use
an anonymous function as a wrapper.


\begin{verbatim}
[~/javascript/jquery(master)]$ cat app2.rb
require 'sinatra'

set :public_folder, File.dirname(__FILE__) + '/starterkit'

get '/' do
  erb :index
end

get '/chuchu' do
  if request.xhr? # is an ajax request
    "hello world!"
  else 
    erb :tutu
  end
end

__END__

@@layout
  <!DOCTYPE html>
  <html>
    <head>
        <meta charset="utf-8" />
        <title>Demo</title>
    </head>
    <body>
        <a href="http://jquery.com/">jQuery</a>
        <div class="result"></div>
        <script src="jquery.js"></script>
        <%= yield %>
    </body>
  </html>

@@tutu
  <h1>Not an Ajax Request!</h1>

@@index
  <script>
    var param = "chuchu param";
    var handler = function( data, textStatus, jqXHR, param ) {
      $( ".result" ).html( data );
      alert( "Load was performed.\n"+
             "$data = "+data+
             "\ntextStatus = "+textStatus+
             "\njqXHR = "+JSON.stringify(jqXHR)+
             "\nparam = "+param );
    };
    $( document ).ready(function() {
        $( "a" ).click(function( event ) {
            event.preventDefault();
            $.get( "/chuchu", function(data, textStatus, jqXHR ) {
              handler( data, textStatus, jqXHR, param);
            });
        });
    });
  </script>
\end{verbatim}
El ejemplo en \verb|app2.rb| puede verse desplegado en Heroku:
\htmladdnormallink{http://jquery-tutorial.herokuapp.com/}{http://jquery-tutorial.herokuapp.com/}



