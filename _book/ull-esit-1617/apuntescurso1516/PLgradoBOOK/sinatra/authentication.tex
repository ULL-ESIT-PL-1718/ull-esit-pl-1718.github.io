
OpenID provides sites and services with a decentralized protocol for
authenticating users through a wide variety of providers. What this means
is that a site integrating OpenID can allow its users to log in using,
for example, their Yahoo!, Google, or AOL accounts. Not only can the
consuming site avoid having to create a login system itself, but it
can also take advantage of the accounts that its users already have,
thereby in- creasing user registration and login rates.


In addition to simple authentication, OpenID also offers a series of
extensions through which an OpenID provider can allow sites to obtain
a user’s profile information or integrate additional layers
of security for the login procedure.


What makes OpenID so intriguing is the fact that it offers a standard
that is fully decentralized from the providers and consumers. This
aspect is what allows a single consuming site to allow its
users to log in via Yahoo! and Google, while another site may want to
allow logins via Blogger or WordPress. Ultimately, it is up to the OpenID
consumer (your site or service) to choose what login
methods it would like to offer its user base.

\section{Referencias. Véase Tambien}

\begin{itemize}
\item
\htmladdnormallink{GitHub ahx/sinatra-openid-consumer-example}{https://github.com/ahx/sinatra-openid-consumer-example}
\item
\htmladdnormallink{Google Offers Named OpenIDs}{http://blog.stackoverflow.com/2009/11/google-offers-named-openids/} por  Jeff Atwood
\item
\htmladdnormallink{How do I log in with OpenID?}{http://openid.net/get-an-openid/start-using-your-openid/}
\item
Programming Social Applications por Jonathan Leblanc. O'Reilly. 2011.
\end{itemize}


