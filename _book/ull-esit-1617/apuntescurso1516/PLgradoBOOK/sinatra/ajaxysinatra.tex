
\section{Un Programa para Experimentar con las Expresiones Regulares Usando Ajax}

\htmladdnormallink{regular expression testing playground with ruby and sinatra}{https://gist.github.com/blazeeboy/9259204}

\begin{verbatim}
[~/sinatra/regexp(master)]$ cat regex-tester.rb 
require 'sinatra' # gem install sinatra --no-ri --no-rdoc
set :port, 3000
html = <<-EOT
<html><head><style>
#regex,#text{ width:100%; font-size:15px; display:block; margin-bottom:5px; }
#text{ height: 200px; }
span{ background:rgb(230,191,161); display:inline-block; border-radius:3px;}
</style></head><body>
 
  <input id="regex" placeholder="Regex"/>
  <textarea id="text" placeholder="Text"></textarea>
  <div id="result"></div>
 
  <script src="http://code.jquery.com/jquery-1.11.0.min.js"></script>
  <script>
  $('#regex,#text').keyup(function(){
      $.get('/preview',{
        reg:$('#regex').val(),
        text:$('#text').val()
      },function(r){
        $('#result').html(r);
      });
  });
  </script>
 
</body></html>
EOT
 
get('/'){ html }
get '/preview' do 
  begin
    params['text'].gsub(/(#{params['reg']})/,'<span>\1</span>')
  rescue
    'Your regex is invalid'
  end
end
\end{verbatim}

\section{Un Ejemplo Simple}

Véase 
\htmladdnormallink{sinatra-jquery-ajax}{https://github.com/crguezl/sinatra-jquery-ajax}
en GitHub.

\begin{verbatim}
~/sinatra/sinatra-jquery-ajax(master)]$ tree
.
|--- Gemfile
|--- Gemfile.lock
|--- README
|--- config.ru
|--- public
|   |--- css
|   |   `--- style.css
|   `--- js
|       `--- app.js
|--- sinatra_jquery_test.rb
`--- views
    |--- app.erb
    `--- layout.erb
\end{verbatim}

\begin{verbatim}
[~/sinatra/sinatra-jquery-ajax(master)]$ cat sinatra_jquery_test.rb 
require 'sinatra'

get '/' do
  erb :app
end

get '/play' do 
  if request.xhr?
    %q{<h1 class="blue">Hello! <a href="/">back</a></h1>}
  else
    "<h1>Not an Ajax request!</h1>"
  end
end
\end{verbatim}

\parrafo{request.xhr?}

El predicado
\verb|request.xhr?| nos permite saber si este es una request ajax.

\begin{verbatim}
[~/sinatra/sinatra-jquery-ajax(master)]$ cat views/layout.erb 
<!DOCTYPE html>
<html>
  <head>
    <link rel="stylesheet" type="text/css" href="css/style.css">
    <script src="http://ajax.googleapis.com/ajax/libs/jquery/1.10.2/jquery.min.js"></script>
    <script src="js/app.js"></script>
  </head>
  <body>
    <%= yield %>
  </body>
</html>
\end{verbatim}

\parrafo{Content Delivery Network (CDN)}
A \cei{Content Delivery Network} (\cei{CDN}) is a distributed system of web servers 
that aim to deliver content to users quickly and efficiently. 

A large number of CDNs are available for JavaScript libraries. 

The idea is that if lots of sites use the same CDN or \cei{hotlink}
(véase
\htmladdnormallink{Hotlinking}{http://simple.wikipedia.org/wiki/Hotlinking}), 
it will be cached locally on the user’s machine, saving the user 
from an extra download across all those sites. 

The downside is a loss of control and the chance (however small) 
that the CDN might go down and be unavailable.

\begin{enumerate}
\item 
\htmladdnormallink{Google Hosted Libraries - Developer's Guide}{https://developers.google.com/speed/libraries/devguide}
\item 
\htmladdnormallink{jQuery download}{http://jquery.com/download/}
\end{enumerate}

\parrafo{views/app.erb}
\begin{verbatim}
[~/sinatra/sinatra-jquery-ajax(master)]$ cat views/app.erb 
<div id="div1">
  <h2 class="pink">Let jQuery AJAX Change This Text</h2>
  <button type="button">Get External Content</button>
</div>
\end{verbatim}
Inside a \verb|<button>| element you can put content, like text or images. This is the difference between this element and buttons created with the \verb|<input>| element.

Always specify the type attribute for a \verb|<button>| element. Different browsers use different default types for the \verb|<button>| element.

HTML5 has the following new attributes: \verb|autofocus|, \verb|form|, \verb|formaction|, \verb|formenctype|, \verb|formmethod|, \verb|formnovalidate|, and \verb|formtarget|.


\parrafo{public/js/app.js}

\begin{verbatim}
[~/sinatra/sinatra-jquery-ajax(master)]$ cat public/js/app.js 
$(document).ready(function(){
  $("button").click(function(){
    $("#div1").load("/play",function(responseTxt,statusTxt,xhr){
      /* if(statusTxt=="success") alert("External content loaded successfully!"); */
      if(statusTxt=="error")
        alert("Error: "+xhr.status+": "+xhr.statusText);
    });
  });
});
\end{verbatim}

\parrafo{\$(document).ready(function()\{ ... \})}
No es posible interactuar de forma segura con el contenido de una página hasta que el documento no se encuentre preparado para su manipulación. 

jQuery permite detectar dicho estado a través de la declaración 
\begin{verbatim}
$(document).ready() 
\end{verbatim}
de forma tal que el bloque se ejecutará sólo una vez que la página este disponible.

\parrafo{selectores CSS}
El concepto más básico de jQuery es el de 
\emph{seleccionar algunos elementos y realizar acciones con ellos}.

La biblioteca soporta gran parte de los selectores CSS3 y varios más no estandarizados. 

\begin{verbatim}
    $("#div1").load("/play",function(responseTxt,statusTxt,xhr){ ... }
\end{verbatim}

En 
\htmladdnormallink{http://api.jquery.com/category/selectors/ }{http://api.jquery.com/category/selectors/ }
se puede encontrar una completa referencia sobre los selectores de la biblioteca.

\parrafo{Controladores de Eventos (Event Handlers)}

jQuery provee métodos para asociar \cei{controladores de eventos} (en inglés \cei{event handlers}) 
a selectores. 

Cuando un evento ocurre, la función provista es ejecutada. 
\begin{verbatim}
  $("button").click(function(){ ... })
\end{verbatim}

Dentro de la función, la palabra clave \verb|this| hace referencia al elemento en que el evento ocurre.

Para más detalles sobre los eventos en jQuery, puede consultar 
\htmladdnormallink{http://api.jquery.com/category/events/.}{http://api.jquery.com/category/events/}

\parrafo{El Evento}
La función del controlador de eventos puede recibir un objeto. 

Este objeto puede ser utilizado para determinar la naturaleza del
evento o, por ejemplo, prevenir el comportamiento predeterminado
de éste.

Para más detalles sobre el objeto del evento, visite 

\parrafo{XMLHttpRequest(XHR)}

\htmladdnormallink{http://api.jquery.com/category/events/event-object/}{http://api.jquery.com/category/events/event-object/}

El método \verb|XMLHttpRequest(XHR)| permite a los navegadores comunicarse con el servidor sin la necesidad de recargar la página. 

Este método, también conocido como \cei{Ajax} (\cei{Asynchronous JavaScript and XML}), permite la creación de aplicaciones ricas en interactividad.

Las peticiones Ajax son ejecutadas por el código JavaScript, el cual 

\begin{enumerate}
\item 
envía una petición a una URL y 
\item 
cuando recibe una respuesta, una \cei{función de devolución} (\cei{callback}) 
puede ser ejecutada 
\item 
Esta función recibe como argumento la respuesta del servidor y realiza algo con ella. 
\item 
Debido a que la respuesta es asíncrona, 
el resto del código de la aplicación continua ejecutándose, por lo cual, es imperativo que una función de devolución sea ejecutada para manejar la respuesta.
\end{enumerate}

\parrafo{Métodos jQuery para Ajax}
A través de varios métodos, jQuery provee soporte para Ajax, 
permitiendo abstraer las diferencias que pueden existir entre navegadores. 

Los métodos en cuestión son 

\begin{enumerate}
\item \verb|$.get()|
\htmladdnormallink{.get()}{http://api.jquery.com/jQuery.get/}
\item \verb|$.getScript()|
\htmladdnormallink{.getScript()}{http://api.jquery.com/jQuery.getScript/}
\item \verb|$.getJSON()|
\htmladdnormallink{.getJSON()}{http://api.jquery.com/jQuery.getJSON/}
\item \verb|$.post()|
\htmladdnormallink{.post()}{http://api.jquery.com/jQuery.post/}
\item \verb|$().load()|
\htmladdnormallink{.load()}{http://api.jquery.com/load/}
\item 
\item \verb|$().ajax()|
\htmladdnormallink{jQuery.ajax()}{http://api.jquery.com/jQuery.ajax/}
\end{enumerate}

A pesar que la definición de Ajax posee la palabra \verb|XML|, la
mayoría de las aplicaciones no utilizan dicho formato para el
transporte de datos, sino que en su lugar se utiliza HTML plano o
información en formato \cei{JSON} (\wikip{JSON}, \cei{JavaScript Object Notation}).

Generalmente, jQuery necesita algunas instrucciones sobre el tipo de información 
que se espera recibir cuando se realiza una petición Ajax. 

En algunos casos, el tipo de dato es especificado por el nombre del método, 
pero en otros casos se lo debe detallar como parte de la configuración del método:

\begin{enumerate}
\item 
\verb|text| Para el transporte de cadenas de caracteres simples.
\item 
\verb|html| Para el transporte de bloques de código HTML que serán ubicados en la página.
\item 
\verb|script| Para añadir un nuevo script con código JavaScript a la página.
\item 
\verb|json| Para transportar información en formato JSON, el cual puede incluir cadenas de caracteres, arreglos y objetos.

Es recomendable utilizar los mecanismos que posea el lenguaje del
lado de servidor para la generación de información en JSON.
\item 
\verb|jsonp| Para transportar información JSON de un dominio a otro. 
\item 
\verb|xml| Para transportar información en formato XML.
\end{enumerate}

\parrafo{.load()}
La sintáxis de 
\htmladdnormallink{.load()}{http://api.jquery.com/load/}
es:
\begin{verbatim}
.load( url [, data ] [, complete(responseText, textStatus, XMLHttpRequest) ] )
\end{verbatim}

En nuestro ejemplo lo hemos usado así:

\begin{verbatim}
    $("#div1").load("/play",function(responseTxt,statusTxt,xhr){
      /* if(statusTxt=="success") alert("External content loaded successfully!"); */
      if(statusTxt=="error")
        alert("Error: "+xhr.status+": "+xhr.statusText);
    });
\end{verbatim}
Load data from the server and place the returned HTML into the matched element.
Por ejemplo:
\begin{verbatim}
$( "#result" ).load( "ajax/test.html" );
\end{verbatim}
If no element is matched by the selector, in this case, 
\begin{verbatim}
.load( url [, data ] [, complete(responseText, textStatus, XMLHttpRequest) ] )
\end{verbatim}
if the document does not contain an element with \verb|id="result"|,
the Ajax request will not be sent.

\parrafo{.load() usa GET si no se especifica {\tt data}}
The \verb|POST| method is used if \verb|data| \emph{is provided as an
object}; otherwise, \verb|GET| is assumed.

\parrafo{Especificando un Objeto del Documento Remoto via un Selector}
The \verb|.load()| method, unlike \verb|$.get()|, 
allows us to specify a portion of the remote document to be inserted. 

This is achieved with a special syntax for the \verb|url| 
parameter. 

If one or more space characters are included in the string, 
the portion of the string following the first space is assumed 
to be a jQuery selector that determines the content to be loaded.

\begin{verbatim}
$( "#result" ).load( "ajax/test.html #container" );
\end{verbatim}

When this method executes, it retrieves the content of \verb|ajax/test.html|,
but then jQuery parses the returned document to find the element
with an ID of \verb|container|. 

This element, along with its contents, is
inserted into the element with an ID of \verb|result|, and the rest of the
retrieved document is discarded.

We could modify the example above to use only part of the document that is fetched:

\parrafo{.innerHTML y .load}

jQuery uses the browser's 
\htmladdnormallink{.innerHTML}{http://www.w3schools.com/jsref/tryit.asp?filename=tryjsref_elmnt_innerhtml} property to parse 
the retrieved document and insert it into the current document. 

During this process, browsers often filter elements from the document such as 
\verb|<html>|, 
\verb|<title>|, or 
\verb|<head>| elements. As a result, the elements retrieved by 
\verb|.load()| may not be exactly the same as if the 
document were retrieved directly by the browser.

\section{Ajax, Sinatra y RightJS}
Ajax has been around for a while now but that doesn’t mean it is any
less fun. The nice thing about Sinatra is you are left to do as much or
little JavaScript as you like and you can do it in any way
that you want as well. 

In this 
\htmladdnormallink{ditty }{http://ididitmyway.herokuapp.com/past/2011/2/27/ajax_in_sinatra/}
I hope to show that it’s easy
to add some Ajax magic to a Sinatra app (with a little help from a
JavaScript framework).

\begin{enumerate}
\item 
\htmladdnormallink{An example of Sinatra working with Ajaxified JQuery based on some pieces of code published by Rafael George on the Sinatra Google Group}{https://gist.github.com/mr-rock/205948}
\item 
\htmladdnormallink{Ajax, Sinatra y RightJS}{http://ididitmyway.herokuapp.com/past/2011/2/27/ajax_in_sinatra/}
Darren Jones
\end{enumerate}

\sectionpractica{TicTactoe Usando Ajax}
\label{practica:tictactoeajax}

Extienda la práctica del TicTacToe enunciada e la sección 
\ref{practica:tictactoedatamapper}
para que la página no se recarge cada vez que el jugador hace click 
en una de las casillas.

El código Javascript se encargará de que el navegador envíe la jugada 
elegida por el usuario ""\verb|b2|". Si la jugada es correcta (la casilla 
\verb|b2| no está ocupada) el servidor retornará al navegador la información necesaria
para que pueda proceder a mostrar los movimientos elegidos por el jugador y el computador.
En caso contrario el servidor envía un código de jugada ilegal.
El código javascript es el que modifica la clase de la casilla a \verb|cross| o 
\verb|circle| de manera adecuada.

Mejore las hojas de estilo usando SAAS \ref{chapter:sass}.
Despliegue la aplicación en \Heroku{}.

\begin{enumerate}
\item 
jQuery \htmladdnormallink{.click()}{http://api.jquery.com/click/}
\item 
jQuery \htmladdnormallink{.get()}{http://api.jquery.com/jQuery.get/}
\item 
jQuery \htmladdnormallink{.addClass()}{http://api.jquery.com/addClass/}
\item 
jQuery \htmladdnormallink{.data()}{http://api.jquery.com/data/}
\item 
Como obtener en JS el valor de un elemento sobre el que se ha hecho click?.
Véase una solución en 
\htmladdnormallink{jsfiddle}{http://jsfiddle.net/UcHuD/}
\item 
\htmladdnormallink{HTML5 Custom Data Attributes (data-*)}{http://html5doctor.com/html5-custom-data-attributes/}
\item Puede usar plain text para la comunicación. Opcionalmente si se quiere usar JSON,
la gema \htmladdnormallink{sinatra-json}{http://www.sinatrarb.com/contrib/json.html}
añade un JSON helper que permite retornar documentos JSON.
Se pueden consultar estas fuentes:
  \begin{enumerate}
  \item 
  \htmladdnormallink{json}{http://www.sinatrarb.com/contrib/json.html}
  \item 
  \htmladdnormallink{How to: Return JSON from Sinatra}{http://nathanhoad.net/how-to-return-json-from-sinatra}
  \item \htmladdnormallink{A very small example app showing how to accept and return JSON as an API}{https://github.com/sklise/sinatra-api-example}
  \end{enumerate}
\end{enumerate}


