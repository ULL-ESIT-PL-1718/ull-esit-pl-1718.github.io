\documentclass[]{article}
\usepackage{lmodern}
\usepackage{longtable}
\usepackage{booktabs}
\usepackage{amssymb,amsmath}
\usepackage{ifxetex,ifluatex}
\usepackage{fixltx2e} % provides \textsubscript
\ifnum 0\ifxetex 1\fi\ifluatex 1\fi=0 % if pdftex
  \usepackage[T1]{fontenc}
  \usepackage[utf8]{inputenc}
\else % if luatex or xelatex
  \ifxetex
    \usepackage{mathspec}
    \usepackage{xltxtra,xunicode}
  \else
    \usepackage{fontspec}
  \fi
  \defaultfontfeatures{Mapping=tex-text,Scale=MatchLowercase}
  \newcommand{\euro}{€}
\fi
% use upquote if available, for straight quotes in verbatim environments
\IfFileExists{upquote.sty}{\usepackage{upquote}}{}
% use microtype if available
\IfFileExists{microtype.sty}{%
\usepackage{microtype}
\UseMicrotypeSet[protrusion]{basicmath} % disable protrusion for tt fonts
}{}
\usepackage{color}
\usepackage{fancyvrb}
\newcommand{\VerbBar}{|}
\newcommand{\VERB}{\Verb[commandchars=\\\{\}]}
\DefineVerbatimEnvironment{Highlighting}{Verbatim}{commandchars=\\\{\}}
% Add ',fontsize=\small' for more characters per line
\newenvironment{Shaded}{}{}
\newcommand{\KeywordTok}[1]{\textcolor[rgb]{0.00,0.44,0.13}{\textbf{{#1}}}}
\newcommand{\DataTypeTok}[1]{\textcolor[rgb]{0.56,0.13,0.00}{{#1}}}
\newcommand{\DecValTok}[1]{\textcolor[rgb]{0.25,0.63,0.44}{{#1}}}
\newcommand{\BaseNTok}[1]{\textcolor[rgb]{0.25,0.63,0.44}{{#1}}}
\newcommand{\FloatTok}[1]{\textcolor[rgb]{0.25,0.63,0.44}{{#1}}}
\newcommand{\CharTok}[1]{\textcolor[rgb]{0.25,0.44,0.63}{{#1}}}
\newcommand{\StringTok}[1]{\textcolor[rgb]{0.25,0.44,0.63}{{#1}}}
\newcommand{\CommentTok}[1]{\textcolor[rgb]{0.38,0.63,0.69}{\textit{{#1}}}}
\newcommand{\OtherTok}[1]{\textcolor[rgb]{0.00,0.44,0.13}{{#1}}}
\newcommand{\AlertTok}[1]{\textcolor[rgb]{1.00,0.00,0.00}{\textbf{{#1}}}}
\newcommand{\FunctionTok}[1]{\textcolor[rgb]{0.02,0.16,0.49}{{#1}}}
\newcommand{\RegionMarkerTok}[1]{{#1}}
\newcommand{\ErrorTok}[1]{\textcolor[rgb]{1.00,0.00,0.00}{\textbf{{#1}}}}
\newcommand{\NormalTok}[1]{{#1}}
\ifxetex
  \usepackage[setpagesize=false, % page size defined by xetex
              unicode=false, % unicode breaks when used with xetex
              xetex]{hyperref}
\else
  \usepackage[unicode=true]{hyperref}
\fi
\hypersetup{breaklinks=true,
            bookmarks=true,
            pdfauthor={},
            pdftitle={},
            colorlinks=true,
            citecolor=blue,
            urlcolor=blue,
            linkcolor=magenta,
            pdfborder={0 0 0}}
\urlstyle{same}  % don't use monospace font for urls
\setlength{\parindent}{0pt}
\setlength{\parskip}{6pt plus 2pt minus 1pt}
\setlength{\emergencystretch}{3em}  % prevent overfull lines
\setcounter{secnumdepth}{0}

\usepackage{lastpage}

\date{}

\begin{document}

\thispagestyle{empty}
%begin{tabular}{lcc}
%%%%
% \begin{tabular}{c}
%   \epsfig{file=/tmp/ullesc.eps,width=1.5cm}  
% \end{tabular}                      &
%%%%
  \begin{tabular}{c}
   {\bf Universidad de La Laguna.  Escuela Técnica Superior de Ingeniería Informática}     \\
   {\bf Tercero del Grado de Informática}\\
   {\bf PROCESADORES DE LENGUAJES. CONVOCATORIA DE JUNIO}\\
   26/05/2017  \pageref*{LastPage} páginas         \\   
  \end{tabular}                     % &
%%%%
%%%%
%end{tabular}

\bigskip

%\hrulefill
Nombre:  \underline{\hspace{10.5cm}} 
Alu: \underline{\hspace{2cm}} 
GitHub Id: \underline{\hspace{2cm}} 
%GitHub Team: \underline{\hspace{2cm}} 
\bigskip

%\begin{footnotesize}
%%Notas:
%\begin{itemize}
%  \item
%%  La duración del examen completo es de 2 horas.
%   Respete el uso de mayúsuclas y minúsculas en los comandos y programas
%  \item Escriba con letra clara. Use también el reverso de las hojas 
%  \item Los ejercicios deben realizarse con bolígrafo.
%  \item Al finalizar el exámen, ENTREGAR TODOS LOS FOLIOS utilizados, incluyendo éste.
%  \item Las calificaciones del exámen estarán disponibles unos días antes de la fecha límite de entrega de las actas.
%%  \item Si esta es su 5ª ó 6ª convocatoria, escriba “Xª CONVOCATORIA” en el encabezado de esta hoja.
%\end{itemize}
%\end{footnotesize}



%\tableofcontents

\begin{enumerate}
\def\labelenumi{\arabic{enumi}.}
\item
  Escriba un PEGJS que recibe como entrada una gramática escrita en un
  lenguaje como este:

\begin{verbatim}
  a -> b 'c' | c 'b';
  b -> 'b' b | '' ;
  c -> 'c' c | '';
\end{verbatim}
\end{enumerate}

\begin{itemize}
\item
  Las variables sintácticas como \texttt{a}, \texttt{b} y \texttt{c}
  figuran en la parte izquierda,
\item
  la flecha \texttt{-\textgreater{}} indica \emph{produce} y
\item
  las alternativas de la parte derecha van separadas por barras
  \texttt{\textbar{}}.
\item
  Las partes derechas son concatenaciones de cadenas como
  \texttt{\textquotesingle{}c\textquotesingle{}},
  \texttt{\textquotesingle{}b\textquotesingle{}} o la cadena vacía
  \texttt{\textquotesingle{}\textquotesingle{}} y de variables
  sintácticas como \texttt{a}, \texttt{b} y \texttt{c}
\item
  Las reglas se terminan con un punto y coma \texttt{;}
\item
  El programa debe retornar como salida el programa PEG.js equivalente a
  la gramática de entrada. Para el ejemplo arriba, la salida sería:

\begin{verbatim}
  a = b 'c' / c 'b';
  b = 'b' b / '' ;
  c = 'c' c / '';
\end{verbatim}
\item
  Los alumnos que no hayan superado la parte práctica deben explicar
  como se compila la gramática y escribir un programa que haga uso del
  parser generado
\end{itemize}

\begin{enumerate}
\def\labelenumi{\arabic{enumi}.}
\setcounter{enumi}{1}
\itemsep1pt\parskip0pt\parsep0pt
\item
  Escriba un analizador descendente recursivo predictivo que resuelva el
  problema anterior.

  \begin{itemize}
  \itemsep1pt\parskip0pt\parsep0pt
  \item
    Los alumnos que no hayan superado la 2ª parte o la parte práctica
    deberán escribir el analizador léxico
  \item
    Si tiene pendientes las prácticas procure especialmente la precisión
    y claridad en todo el código de esta prueba.
  \end{itemize}
\end{enumerate}


\end{document}

