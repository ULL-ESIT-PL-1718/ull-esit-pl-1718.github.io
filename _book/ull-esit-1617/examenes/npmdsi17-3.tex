\subsection{Preguntas de Como crear y publicar un paquete
npm}\label{preguntas-de-como-crear-y-publicar-un-paquete-npm}

\begin{enumerate}
\def\labelenumi{\arabic{enumi}.}
\setcounter{enumi}{9}
\item
  ¿Cuales son los pasos para escribir y publicar un paquete
  \texttt{npm}?
\item
  ¿Cómo instalo una versión anterior de un package npm?

\begin{verbatim}
npm install <package>@<version>
\end{verbatim}
\item
  \href{http://stackoverflow.com/questions/10972176/find-the-version-of-an-installed-npm-package}{¿Cómo
  encuentro la versión de un paquete NodeJs instalado?}
\item
  ¿Que se entiende por \texttt{scope} de un paquete npm?

  \begin{enumerate}
  \def\labelenumii{\arabic{enumii}.}
  \item
    ¿Cual es la notación para referenciar un paquete
    \texttt{somepackage} que se aloja en el ámbito \texttt{somescope}?
  \item
    ¿Como se hace un \texttt{require}de un paquete \texttt{somepackage}
    que se aloja en el ámbito \texttt{somescope}?
  \item
    Cualquier usario npm puede publicar sus módulos en cualquier ámbito:
    ¿Verdadero o falso?
  \item
    ¿Cómo puedes hacer para tener varios ámbitos npm?
  \item
    ¿Cual es el tipo de acceso por defecto de mi paquete con ámbito:
    público o privado? 
  \item
    ¿Cómo indico a la hora de publicar mi paquete con ámbito que quiero
    que sea de acceso público? 
  \item
    ¿Que hace este comando?

\begin{verbatim}
 npm login --registry=http://reg.example.com --scope=@myco
\end{verbatim}
  \item
    ¿Que hace este comando?

\begin{verbatim}
 npm init --scope=username
\end{verbatim}
  \item
    ¿Que hace este comando?

\begin{verbatim}
 npm config set scope username
\end{verbatim}
  \end{enumerate}
\item
  ¿Que hace este comando?

\begin{verbatim}
 npm version patch -m "Upgrade to %s for reasons"
\end{verbatim}

  \begin{enumerate}
  \def\labelenumii{\arabic{enumii}.}
  \item
    Cuando el comando anterior se ejecuta en un git repo, ¿Crea un
    commit? ¿Crea un tag?
  \item
    If \texttt{preversion}, \texttt{version}, or \texttt{postversion}
    are in the scripts property of the package.json, they will be
    executed as part of running npm \texttt{version}. Take the following
    example:

\begin{verbatim}
 "scripts": {
   "preversion": "npm test",
   "version": "npm run build && git add -A dist",
   "postversion": "git push && git push --tags && rm -rf build/temp"
 }
\end{verbatim}

    Describe the set of actions that will happen when running
    \texttt{npm version}
  \end{enumerate}
\end{enumerate}

\subsection{JsDoc}\label{jsdoc}

\begin{enumerate}
\def\labelenumi{\arabic{enumi}.}
\itemsep1pt\parskip0pt\parsep0pt
\item
  ¿Cual es la sintáxis de los comentarios de documentación en JSDoc?
  ¿Como empiezan?
\item
  ¿Donde debo ubicar un comentario de documentación JSDoc para
  documentar un cierto código?
\item
  Special \emph{JSDoc tags} can be used to give more information. For
  example,

  \begin{enumerate}
  \def\labelenumii{\arabic{enumii}.}
  \item
    If the function is a constructor for a class, you can indicate this
    by adding a \_\_\_\_\_\_\_\_\_\_\_\_ tag (fill the gap)
  \item
    ¿Cómo se documenta un parámetro de una función?
  \item
    ¿Que hace esta documentación JSDoc?

\begin{verbatim}
/**
 * See {@tutorial gettingstarted} 
 */
function myFunction() {}
\end{verbatim}
  \item
    ¿Que hace esta documentación JSDoc?

\begin{verbatim}
/**
 * See {@link MyClass} and [MyClass's foo property]{@link MyClass#foo}.
 * Also, check out {@link http://www.google.com|Google} and
 * {@link https://github.com GitHub}.
 */
function myFunction() {}
\end{verbatim}
  \end{enumerate}
\end{enumerate}
