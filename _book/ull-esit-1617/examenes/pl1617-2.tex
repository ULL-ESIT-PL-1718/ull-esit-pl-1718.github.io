\documentclass[]{article}
\usepackage{lmodern}
\usepackage{longtable}
\usepackage{booktabs}
\usepackage{amssymb,amsmath}
\usepackage{ifxetex,ifluatex}
\usepackage{fixltx2e} % provides \textsubscript
\ifnum 0\ifxetex 1\fi\ifluatex 1\fi=0 % if pdftex
  \usepackage[T1]{fontenc}
  \usepackage[utf8]{inputenc}
\else % if luatex or xelatex
  \ifxetex
    \usepackage{mathspec}
    \usepackage{xltxtra,xunicode}
  \else
    \usepackage{fontspec}
  \fi
  \defaultfontfeatures{Mapping=tex-text,Scale=MatchLowercase}
  \newcommand{\euro}{€}
\fi
% use upquote if available, for straight quotes in verbatim environments
\IfFileExists{upquote.sty}{\usepackage{upquote}}{}
% use microtype if available
\IfFileExists{microtype.sty}{%
\usepackage{microtype}
\UseMicrotypeSet[protrusion]{basicmath} % disable protrusion for tt fonts
}{}
\usepackage{color}
\usepackage{fancyvrb}
\newcommand{\VerbBar}{|}
\newcommand{\VERB}{\Verb[commandchars=\\\{\}]}
\DefineVerbatimEnvironment{Highlighting}{Verbatim}{commandchars=\\\{\}}
% Add ',fontsize=\small' for more characters per line
\newenvironment{Shaded}{}{}
\newcommand{\KeywordTok}[1]{\textcolor[rgb]{0.00,0.44,0.13}{\textbf{{#1}}}}
\newcommand{\DataTypeTok}[1]{\textcolor[rgb]{0.56,0.13,0.00}{{#1}}}
\newcommand{\DecValTok}[1]{\textcolor[rgb]{0.25,0.63,0.44}{{#1}}}
\newcommand{\BaseNTok}[1]{\textcolor[rgb]{0.25,0.63,0.44}{{#1}}}
\newcommand{\FloatTok}[1]{\textcolor[rgb]{0.25,0.63,0.44}{{#1}}}
\newcommand{\CharTok}[1]{\textcolor[rgb]{0.25,0.44,0.63}{{#1}}}
\newcommand{\StringTok}[1]{\textcolor[rgb]{0.25,0.44,0.63}{{#1}}}
\newcommand{\CommentTok}[1]{\textcolor[rgb]{0.38,0.63,0.69}{\textit{{#1}}}}
\newcommand{\OtherTok}[1]{\textcolor[rgb]{0.00,0.44,0.13}{{#1}}}
\newcommand{\AlertTok}[1]{\textcolor[rgb]{1.00,0.00,0.00}{\textbf{{#1}}}}
\newcommand{\FunctionTok}[1]{\textcolor[rgb]{0.02,0.16,0.49}{{#1}}}
\newcommand{\RegionMarkerTok}[1]{{#1}}
\newcommand{\ErrorTok}[1]{\textcolor[rgb]{1.00,0.00,0.00}{\textbf{{#1}}}}
\newcommand{\NormalTok}[1]{{#1}}
\ifxetex
  \usepackage[setpagesize=false, % page size defined by xetex
              unicode=false, % unicode breaks when used with xetex
              xetex]{hyperref}
\else
  \usepackage[unicode=true]{hyperref}
\fi
\hypersetup{breaklinks=true,
            bookmarks=true,
            pdfauthor={},
            pdftitle={},
            colorlinks=true,
            citecolor=blue,
            urlcolor=blue,
            linkcolor=magenta,
            pdfborder={0 0 0}}
\urlstyle{same}  % don't use monospace font for urls
\setlength{\parindent}{0pt}
\setlength{\parskip}{6pt plus 2pt minus 1pt}
\setlength{\emergencystretch}{3em}  % prevent overfull lines
\setcounter{secnumdepth}{0}

\usepackage{lastpage}

\date{}

\begin{document}

\thispagestyle{empty}
%begin{tabular}{lcc}
%%%%
% \begin{tabular}{c}
%   \epsfig{file=/tmp/ullesc.eps,width=1.5cm}  
% \end{tabular}                      &
%%%%
  \begin{tabular}{c}
   {\bf Universidad de La Laguna.  Escuela Técnica Superior de Ingeniería Informática}     \\
   {\bf Tercero del Grado de Informática}\\
   {\bf PROCESADORES DE LENGUAJES: 2ª PARTE}\\
   05/04/2017  \pageref*{LastPage} páginas         \\   
  \end{tabular}                     % &
%%%%
%%%%
%end{tabular}

\bigskip

%\hrulefill
Nombre:  \underline{\hspace{10.5cm}} 
Alu: \underline{\hspace{2cm}} 
GitHub Id: \underline{\hspace{2cm}} 
GitHub Team: \underline{\hspace{2cm}} 
\bigskip

%\begin{footnotesize}
%%Notas:
%\begin{itemize}
%  \item
%%  La duración del examen completo es de 2 horas.
%   Respete el uso de mayúsuclas y minúsculas en los comandos y programas
%  \item Escriba con letra clara. Use también el reverso de las hojas 
%  \item Los ejercicios deben realizarse con bolígrafo.
%  \item Al finalizar el exámen, ENTREGAR TODOS LOS FOLIOS utilizados, incluyendo éste.
%  \item Las calificaciones del exámen estarán disponibles unos días antes de la fecha límite de entrega de las actas.
%%  \item Si esta es su 5ª ó 6ª convocatoria, escriba “Xª CONVOCATORIA” en el encabezado de esta hoja.
%\end{itemize}
%\end{footnotesize}



%\tableofcontents

\subsection{Preguntas de Repaso de Expresiones
Regulares}\label{preguntas-de-repaso-de-expresiones-regulares}

\begin{enumerate}
\def\labelenumi{\arabic{enumi}.}
\item
  Escriba expresiones regulares que casen con las siguientes
  especificaciones:

  \begin{enumerate}
  \def\labelenumii{\arabic{enumii}.}
  \itemsep1pt\parskip0pt\parsep0pt
  \item
    car and cat
  \item
    pop and prop
  \item
    ferret, ferry, and ferrari
  \item
    Any word ending in ious
  \item
    A whitespace character followed by a dot, comma, colon, or semicolon
  \item
    A word longer than six letters
  \item
    A word without the letter e
  \end{enumerate}
\item
  Escriba una expresión regular que reconozca las cadenas de doble
  comillas. Debe permitir la presencia de comillas y caracteres
  escapados.
\item
  Escriba una expresión regular que reconozca los números en punto
  flotante (por ejemplo \texttt{-2.3e-1}, \texttt{-3e2}, \texttt{23},
  \texttt{3.2}). numbers = /\^{} \ldots{} \$/, matching exacto
\item
  Escriba una expresión regular que case con los números no primos
  expresados en unario. Pruebe con \texttt{1111}, \texttt{111},
  \texttt{111111}, \texttt{1111111}, \ldots{}
\item
  Escriba una expresión regular que case con los comentarios JavaScript.
\item
  Escriba una expresión JavaScript que permita reemplazar todas las
  apariciones de palabras consecutivas repetidas (como
  \texttt{hello hello}) por una sóla aparición de la misma
\item
  ¿Cual es la salida?

\begin{verbatim}
> "bb".match(/b|bb/)

> "bb".match(/bb|b/)
\end{verbatim}

  Justifique su respuesta.
\item
  El siguiente fragmento de código tiene por objetivo escapar las
  entidades HTML para que no sean intérpretadas como código HTML.
  Rellene las partes que faltan.
\end{enumerate}

\begin{Shaded}
\begin{Highlighting}[]
\KeywordTok{var} \NormalTok{entityMap = \{}
    \StringTok{"&"}\NormalTok{: }\StringTok{"&___;"}\NormalTok{,}
    \StringTok{"<"}\NormalTok{: }\StringTok{"&__;"}\NormalTok{,}
    \StringTok{">"}\NormalTok{: }\StringTok{"&__;"}\NormalTok{,}
    \StringTok{'"'}\NormalTok{: }\StringTok{'&quot;'}\NormalTok{,}
    \StringTok{"'"}\NormalTok{: }\StringTok{'&#39;'}\NormalTok{,}
    \StringTok{"/"}\NormalTok{: }\StringTok{'&#x2F;'}
  \NormalTok{\};}

\KeywordTok{function} \FunctionTok{escapeHtml}\NormalTok{(string) \{}
  \KeywordTok{return} \FunctionTok{String}\NormalTok{(string).}\FunctionTok{replace}\NormalTok{(}\OtherTok{/_________/g}\NormalTok{, }\KeywordTok{function} \NormalTok{(s) \{}
    \KeywordTok{return} \NormalTok{____________;}
  \NormalTok{\});}
\end{Highlighting}
\end{Shaded}

\begin{enumerate}
\def\labelenumi{\arabic{enumi}.}
\setcounter{enumi}{1}
\item
  Se quiere poner un espacio en blanco después de la aparición de cada
  coma:

\begin{verbatim}
> x = "a,b,c,1,2,d, e,f"
'a,b,c,1,2,d, e,f'
> x.replace(/,/g,", ")
'a, b, c, 1, 2, d,  e, f'
\end{verbatim}

  Se pide que si hay ya un espacio después de la coma, no se duplique.
\item
  Se pide una expresión regular que case con expresiones del tipo
  \texttt{identifier = number} y retorne con cada paréntesis el
  identificador y el número. Pruebe con \texttt{h     = 4},
  \texttt{temp = 5.6}, \texttt{x23= -2.3e1} y \texttt{z += 3}
\item
  Imagine you have written a story and used single quotation marks
  throughout to mark pieces of dialogue. Now you want to replace all the
  dialogue quotes with double quotes, while keeping the single quotes
  used in contractions like \emph{aren't}. Think of a pattern that
  distinguishes these two kinds of quote usage and craft a call to the
  replace method that does the proper replacement.

\begin{verbatim}
var text = "'I'm the cook,' he said, 'it's my job.'";
// Change this call.
var result = text.replace(/.../g, '...');
console.log(result);
var expected = `"I'm the cook," he said, "it's my job."`;
if (expected === result) console.log("OK")
else console.log("ERROR!");
\end{verbatim}
\end{enumerate}


\subsection{Preguntas de Repaso de Análisis
Léxico}\label{preguntas-de-repaso-de-anuxe1lisis-luxe9xico}

\begin{enumerate}
\def\labelenumi{\arabic{enumi}.}
\itemsep1pt\parskip0pt\parsep0pt
\item
  Escriba un método \texttt{bexec} para los objetos \texttt{RegExp}que
  funcione como \texttt{exec} pero sólo case en la posición
  \texttt{lastIndex}:
\end{enumerate}

\begin{verbatim}
        RegExp.prototype.bexec = function(str) {
          __________________________
          __________________________
          __________________________
          __________________________
          __________________________
          __________________________
          __________________________
        };
\end{verbatim}

\begin{enumerate}
\def\labelenumi{\arabic{enumi}.}
\setcounter{enumi}{1}
\itemsep1pt\parskip0pt\parsep0pt
\item
  Basándose en el método \texttt{bexec} del ejercicio anterior escriba
  las partes que faltan del método \texttt{tokens} que implementa un
  analizador léxico para un lenguaje de tipo JavaScript en el que
  \texttt{if} y \texttt{then} son palabras reservadas:
\end{enumerate}

\begin{verbatim}
  String.prototype.tokens = function() {
    var RESERVED_WORD, from, getTok, i, key, m, make, n, result, rw, tokens, value;
    from = 0;
    i = 0;
    n = 0;
    m = 0;
    result = [];
    tokens = {
      WHITES: ____________________________________________,
      ID: ____________________________________________,
      NUM: ____________________________________________,
      STRING: ____________________________________________,
      ONELINECOMMENT: ____________________________________________,
      MULTIPLELINECOMMENT: ____________________________________________,
      COMPARISONOPERATOR: ____________________________________________,
      ONECHAROPERATORS: /([=()&|;:,{}[\]])/g,
      ADDOP: /[+-]/g,
      MULTOP: /[*\/]/g
    };
    RESERVED_WORD = {
      "if": "IF",
      then: "THEN"
    };
\end{verbatim}

Explique como se usa en el análisis el objeto \texttt{RESERVED\_WORD} y
para que sirve. Observe su uso en el bucle

\begin{verbatim}
    make = function(type, value) {
      return {
        type: type,
        value: value,
        from: from,
        to: i
      };
    };
    getTok = function() {
      var str;
      str = m[0];
      i += _____________________;
      return str;
    };
    if (!this) {
      return;
    }
    while (i < this.length) {
      for (key in tokens) {
        value = tokens[key];
        ________________________
      }
      from = __________;
      if (m = tokens.WHITES.bexec(this) || 
         (m = tokens.ONELINECOMMENT.bexec(this)) || 
         (m = tokens.MULTIPLELINECOMMENT.bexec(this))) {
        _________________________________;
      } else if (m = tokens.ID.bexec(this)) {
        rw = RESERVED_WORD[m[0]];
        if (rw) {
          result.push(make(rw, getTok()));
        } else {
          result.push(_____________________________________);
        }
      } else if (m = tokens.NUM.bexec(this)) {
        n = +getTok();
        if (isFinite(n)) {
          result.push(____________________________________);
        } else {
          make("NUM", m[0]).error("Bad number");
        }
      } else if (m = tokens.STRING.bexec(this)) {
        result.push(make("STRING", getTok().replace(_______________________/g, "")));
      } else if (m = tokens.COMPARISONOPERATOR.bexec(this)) {
        result.push(make("COMPARISON", getTok()));
      } else if (m = tokens.ADDOP.bexec(this)) {
        result.push(make("ADDOP", getTok()));
      } else if (m = tokens.MULTOP.bexec(this)) {
        result.push(make("MULTOP", getTok()));
      } else if (m = tokens.ONECHAROPERATORS.bexec(this)) {
        result.push(make(m[0], getTok()));
      } else {
        throw "Syntax error near '" + (this.substr(i)) + "'";
      }
    }
    return result;
  };
\end{verbatim}


\end{document}
