\subsection{Repaso de MarkDown}\label{repaso-de-markdown}

\begin{itemize}
\item
  ¿Como se escribe en MarkDown un link de tipo referencia?

\begin{verbatim}
This is [an example][id] reference-style link.

Then, anywhere in the document, you define your link label like this, on a line by itself:

[id]: http://example.com/  "Optional Title Here"
\end{verbatim}
\item
  ¿Cómo se pone una imagen?

\begin{verbatim}
![Screenshot of the toolbar](http://so.mrozekma.com/editor-bar-help-button.png)
\end{verbatim}
\item
  ¿Cómo se anidan listas?

\begin{verbatim}
 To put other Markdown blocks in a list; just indent four spaces for each nesting level
  For example 

  1. Dog
      1. German Shepherd
      2. Belgian Shepherd
          1. Malinois
          2. Groenendael
          3. Tervuren
  2. Cat
      1. Siberian
      2. Siamese
\end{verbatim}
\item
  ¿Cómo se inserta una línea horizontal?
\end{itemize}
