\subsection{Preguntas de Repaso de Expresiones
Regulares}\label{preguntas-de-repaso-de-expresiones-regulares}

\begin{enumerate}
\def\labelenumi{\arabic{enumi}.}
\itemsep1pt\parskip0pt\parsep0pt
\item
  Rellene las partes que faltan:
\end{enumerate}

\begin{Shaded}
\begin{Highlighting}[]
\NormalTok{> re = }\OtherTok{/d}\FloatTok{(}\OtherTok{b}\FloatTok{+)(}\OtherTok{d}\FloatTok{)}\OtherTok{/ig}
\OtherTok{/d}\FloatTok{(}\OtherTok{b}\FloatTok{+)(}\OtherTok{d}\FloatTok{)}\OtherTok{/gi}
\NormalTok{> z = }\StringTok{"dBdxdbbdzdbd"}
\StringTok{'dBdxdbbdzdbd'}
\NormalTok{> result = }\OtherTok{re}\NormalTok{.}\FunctionTok{exec}\NormalTok{(z)}
\NormalTok{[ ______, _____, ______, index: __, input: }\StringTok{'dBdxdbbdzdbd'} \NormalTok{]}
\NormalTok{> }\OtherTok{re}\NormalTok{.}\FunctionTok{lastIndex}
\NormalTok{______}
\NormalTok{> result = }\OtherTok{re}\NormalTok{.}\FunctionTok{exec}\NormalTok{(z)}
\NormalTok{[ ______, _____, ______, index: __, input: }\StringTok{'dBdxdbbdzdbd'} \NormalTok{]}
\NormalTok{> }\OtherTok{re}\NormalTok{.}\FunctionTok{lastIndex}
\NormalTok{______}
\NormalTok{> result = }\OtherTok{re}\NormalTok{.}\FunctionTok{exec}\NormalTok{(z)}
\NormalTok{[ ______, _____, ______, index: __, input: }\StringTok{'dBdxdbbdzdbd'} \NormalTok{]}
\NormalTok{> }\OtherTok{re}\NormalTok{.}\FunctionTok{lastIndex}
\NormalTok{______}
\NormalTok{> result = }\OtherTok{re}\NormalTok{.}\FunctionTok{exec}\NormalTok{(z)}
\NormalTok{_____}
\end{Highlighting}
\end{Shaded}

\begin{enumerate}
\def\labelenumi{\arabic{enumi}.}
\setcounter{enumi}{1}
\itemsep1pt\parskip0pt\parsep0pt
\item
  Escriba la expresión regular \texttt{r} para que produzca el resultado
  final:
\end{enumerate}

\begin{Shaded}
\begin{Highlighting}[]
\NormalTok{> x = }\StringTok{"hello"}
\NormalTok{> r = }\OtherTok{/l}\FloatTok{(}\OtherTok{___}\FloatTok{)}\OtherTok{/}
\NormalTok{> z = }\OtherTok{r}\NormalTok{.}\FunctionTok{exec}\NormalTok{(x)}
\NormalTok{[ }\StringTok{'l'}\NormalTok{, index: }\DecValTok{3}\NormalTok{, input: }\StringTok{'hello'} \NormalTok{]}
\end{Highlighting}
\end{Shaded}

\begin{enumerate}
\def\labelenumi{\arabic{enumi}.}
\setcounter{enumi}{2}
\itemsep1pt\parskip0pt\parsep0pt
\item
  Rellene el valor que falta:
\end{enumerate}

\begin{Shaded}
\begin{Highlighting}[]
\NormalTok{> z = }\StringTok{"dBdDBBD"}
\NormalTok{> re = }\OtherTok{/d}\FloatTok{(}\OtherTok{b}\FloatTok{+)(}\OtherTok{d}\FloatTok{)}\OtherTok{/ig}
\NormalTok{> }\OtherTok{re}\NormalTok{.}\FunctionTok{lastIndex} \NormalTok{= ________}
\NormalTok{> result = }\OtherTok{re}\NormalTok{.}\FunctionTok{exec}\NormalTok{(z)}
\NormalTok{[ }\StringTok{'DBBD'}\NormalTok{,}
  \StringTok{'BB'}\NormalTok{,}
  \StringTok{'D'}\NormalTok{,}
  \NormalTok{index: }\DecValTok{3}\NormalTok{,}
  \NormalTok{input: }\StringTok{'dBdDBBD'} \NormalTok{]}
\end{Highlighting}
\end{Shaded}

\begin{enumerate}
\def\labelenumi{\arabic{enumi}.}
\setcounter{enumi}{3}
\item
  Conteste:

  \begin{enumerate}
  \def\labelenumii{\arabic{enumii}.}
  \item
    Explique que hace el siguiente fragmento de código:

\begin{verbatim}
> RegExp.prototype.bexec = function(str) {
...   var i = this.lastIndex;
...   var m = this.exec(str);
...   if (m && m.index == i) return m;
...   return null;
... }
[Function]
\end{verbatim}
  \item
    Rellene las salidas que faltan:
  \end{enumerate}
\end{enumerate}

\begin{Shaded}
\begin{Highlighting}[]
\NormalTok{> re = }\OtherTok{/d}\FloatTok{(}\OtherTok{b}\FloatTok{+)(}\OtherTok{d}\FloatTok{)}\OtherTok{/ig}
\OtherTok{/d}\FloatTok{(}\OtherTok{b}\FloatTok{+)(}\OtherTok{d}\FloatTok{)}\OtherTok{/gi}
\NormalTok{> z = }\StringTok{"dBdXXXXDBBD"}
\StringTok{'dBdXXXXDBBD'}
\NormalTok{> }\OtherTok{re}\NormalTok{.}\FunctionTok{lastIndex} \NormalTok{= }\DecValTok{3}
\NormalTok{> }\OtherTok{re}\NormalTok{.}\FunctionTok{bexec}\NormalTok{(z)}
\NormalTok{_____________________________________________________}
\NormalTok{> }\OtherTok{re}\NormalTok{.}\FunctionTok{lastIndex} \NormalTok{= }\DecValTok{7}
\NormalTok{> }\OtherTok{re}\NormalTok{.}\FunctionTok{bexec}\NormalTok{(z)}
\NormalTok{_____________________________________________________}
\end{Highlighting}
\end{Shaded}

\begin{enumerate}
\def\labelenumi{\arabic{enumi}.}
\setcounter{enumi}{4}
\item
  Escriba una expresión JavaScript que permita reemplazar todas las
  apariciones de palabras consecutivas repetidas (como
  \texttt{hello hello}) por una sóla aparición de la misma
\item
  ¿Cual es la salida?

\begin{verbatim}
> "bb".match(/b|bb/)

> "bb".match(/bb|b/)
\end{verbatim}

  Justifique su respuesta.
\item
  El siguiente fragmento de código tiene por objetivo escapar las
  entidades HTML para que no sean intérpretadas como código HTML.
  Rellene las partes que faltan.
\end{enumerate}

\begin{Shaded}
\begin{Highlighting}[]
\KeywordTok{var} \NormalTok{entityMap = \{}
    \StringTok{"&"}\NormalTok{: }\StringTok{"&___;"}\NormalTok{,}
    \StringTok{"<"}\NormalTok{: }\StringTok{"&__;"}\NormalTok{,}
    \StringTok{">"}\NormalTok{: }\StringTok{"&__;"}\NormalTok{,}
    \StringTok{'"'}\NormalTok{: }\StringTok{'&quot;'}\NormalTok{,}
    \StringTok{"'"}\NormalTok{: }\StringTok{'&#39;'}\NormalTok{,}
    \StringTok{"/"}\NormalTok{: }\StringTok{'&#x2F;'}
  \NormalTok{\};}

\KeywordTok{function} \FunctionTok{escapeHtml}\NormalTok{(string) \{}
  \KeywordTok{return} \FunctionTok{String}\NormalTok{(string).}\FunctionTok{replace}\NormalTok{(}\OtherTok{/_________/g}\NormalTok{, }\KeywordTok{function} \NormalTok{(s) \{}
    \KeywordTok{return} \NormalTok{____________;}
  \NormalTok{\});}
\end{Highlighting}
\end{Shaded}

