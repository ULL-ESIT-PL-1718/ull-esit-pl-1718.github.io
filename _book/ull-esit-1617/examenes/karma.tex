\subsection{Preguntas de Repaso de
Karma}\label{preguntas-de-repaso-de-karma}

\begin{enumerate}
\def\labelenumi{\arabic{enumi}.}
\item
  Explique como funciona Karma
\item
  ¿Con que comando puedo crear el fichero de configuración de Karma?
\item
  ¿Que debemos poner en la entrada \texttt{frameworks} de karma para el
  ejemplo de la temperatura?

\begin{verbatim}
    frameworks: ['_____'],
\end{verbatim}
\item
  La librería/plugin \texttt{karma-mocha} provee el adapter de Karma
  para Mocha. ¿Como le pasamos opciones para configurar Mocha desde
  Karma? Rellene las partes que faltan:
\end{enumerate}

\begin{Shaded}
\begin{Highlighting}[]
\NormalTok{client: \{}
  \DataTypeTok{args}\NormalTok{: [}\StringTok{'--grep'}\NormalTok{, }\StringTok{'pattern'}\NormalTok{], }\CommentTok{// solo pruebas que casan con pattern}
  \DataTypeTok{mocha}\NormalTok{: \{}
    \DataTypeTok{__}\NormalTok{: }\StringTok{'___'}
  \NormalTok{\}}
\NormalTok{\},}
\end{Highlighting}
\end{Shaded}

\begin{enumerate}
\def\labelenumi{\arabic{enumi}.}
\setcounter{enumi}{4}
\item
  Explique que debe ponerse (y que no) en la sección \texttt{files} del
  fichero de configuración de Karma ¿Donde son cargados dichos
  ficheros?:

\begin{verbatim}
    files: [ ... ],
\end{verbatim}
\item
  Los preprocesadores en Karma nos permiten procesar los ficheros en
  \texttt{files} antes de que sean cargados en el navegador.
\end{enumerate}

\begin{Shaded}
\begin{Highlighting}[]
          \NormalTok{preprocessors = \{}
            \StringTok{'**/*.coffee'}\NormalTok{: }\StringTok{'coffee'}\NormalTok{,}
            \StringTok{'**/*.html'}\NormalTok{: }\StringTok{'html2js'}
          \NormalTok{\};}
\end{Highlighting}
\end{Shaded}

\begin{verbatim}
¿Que hace el preprocesador `html2js`? ¿Que hace el preprocesador
`coffee`?
\end{verbatim}

\begin{enumerate}
\def\labelenumi{\arabic{enumi}.}
\setcounter{enumi}{6}
\itemsep1pt\parskip0pt\parsep0pt
\item
  Complete la función \texttt{setup} de las pruebas de la práctica de la
  temperatura con Mocha, Chai y Karma:
\end{enumerate}

\begin{Shaded}
\begin{Highlighting}[]
\FunctionTok{setup}\NormalTok{(}\KeywordTok{function}\NormalTok{()\{}
  \KeywordTok{if} \NormalTok{(}\KeywordTok{typeof} \NormalTok{________ !== }\StringTok{'undefined'}\NormalTok{) \{}
      \OtherTok{document}\NormalTok{.}\OtherTok{body}\NormalTok{.}\FunctionTok{innerHTML} \NormalTok{= ________[}\StringTok{'tests/test.html'}\NormalTok{];}
      \NormalTok{original = }\OtherTok{document}\NormalTok{.}\FunctionTok{______________}\NormalTok{(}\StringTok{'original'}\NormalTok{);}
      \NormalTok{converted = }\OtherTok{document}\NormalTok{.}\FunctionTok{______________}\NormalTok{(}\StringTok{'converted'}\NormalTok{);}
  \NormalTok{\}}
\NormalTok{\});}
\end{Highlighting}
\end{Shaded}

¿Como se llama la variable en la que se guardan los HTML de los ficheros
cargados en los navegadores?

\begin{enumerate}
\def\labelenumi{\arabic{enumi}.}
\setcounter{enumi}{7}
\itemsep1pt\parskip0pt\parsep0pt
\item
  ¿Que es PhantomJS? ¿Como funciona?
\end{enumerate}
