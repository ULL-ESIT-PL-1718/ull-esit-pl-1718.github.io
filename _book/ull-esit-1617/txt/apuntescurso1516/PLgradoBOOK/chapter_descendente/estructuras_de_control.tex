\section{Condicionales}

\section{Bucles}

\section{iteradores y Objetos Enumerables}

\section{Bloques}

\section{Alterando el Flujo de Control}

\section{Manejo de Excepciones}

\section{BEGIN y END}

\section{Threads, Fibras y Continuaciones}

\sectionpractica{Reto Dropbox. El Problema de la Dieta}
En el sitio web de Dropbox en
\htmladdnormallink{http://www.dropbox.com/jobs/challenges}{http://www.dropbox.com/jobs/challenges}
pueden encontrarse algunos retos de programación.

\begin{it}
\begin{quote}
We get a lot of interest in working at Dropbox but it's not always easy to tell how a person's brain works from a resume. If you're like us, you love a good puzzle. That's why this page is here! It's a great way to share why we love working at Dropbox: innovative thinking, a bit of elbow grease, and some good fun. Who knows? If you knock these puzzles out of the park, we'll have something to talk about when you come in.


Send all submissions to challenges@dropbox.com. The subject line should match the puzzle name. Please send your code (and any associated makefiles), not the executable.
\end{quote}
\end{it}

En esta práctica se trata de resolver {\it el problema de la dieta}.
En el sitio de Dropbox se describe como sigue:

\begin{it}
\begin{quote}
{\bf The Dropbox Diet}

Of the boatload of perks Dropbox offers, the ones most threatening
to our engineers' waistlines are the daily lunches, the fully-stocked
drink fridge, and a full-length bar covered with every snack you
could want. All of those calories add up. Luckily, the office is
also well-equipped with ping-pong, a \wikip{DDR}{Dance\_Dance\_Revolution} machine, and a subsidized
gym right across the street that can burn those calories right back
off. Although we often don't, Dropboxers should choose the food
they eat to counterbalance the activities they perform so that they
don't end up with caloric deficit or excess.

Help us keep our caloric intake in check. You'll be given a list
of activities and their caloric impact. Write a program that outputs
the names of activities a Dropboxer should choose to partake in so
that the sum of their caloric impact is zero. Once an activity is
selected, it cannot be chosen again.

\begin{itemize}

\item Input

Your program reads an integer \verb|N (1 <= N <= 50)| from \verb|STDIN|
representing
the number of list items in the test input. The list is comprised
of activities or food items and its respective calorie impact
separated by a space, one pair per line. Activity names will use
only lowercase ASCII letters and the dash (-) character.


\item Output

Output should be sent to stdout, one activity name per line,
alphabetized. If there is no possible solution, the output should
be \verb|no solution|. If there are multiple solutions, your program can
output {\it any one of them}. Solutions should be non-trivial, so don't
send us \verb|cat > /dev/null|, you smart aleck.

\item  Examples:

\begin{enumerate}
\item
\begin{tabular}{|p{7cm}|p{5cm}|}
 Sample Input & Sample Output\\ \hline
\begin{verbatim}
2
red-bull 140
coke 110
\end{verbatim}
&
\begin{verbatim}
no solution
\end{verbatim}\\ \hline
\end{tabular}

\item

\begin{tabular}{|p{7cm}|p{5cm}|}
 Sample Input & Sample Output\\ \hline
\begin{verbatim}
12
free-lunch 802
mixed-nuts 421
orange-juice 143
heavy-ddr-session -302
cheese-snacks 137
cookies 316
mexican-coke 150
dropballers-basketball -611
coding-six-hours -466
riding-scooter -42
rock-band -195
playing-drums -295
\end{verbatim}
&
\begin{verbatim}
coding-six-hours
cookies
mexican-coke
\end{verbatim}\\
\hline
\end{tabular}
\end{enumerate}
\item
Puede encontrar dos soluciones al problema escritas en Ruby
en el blog de \htmladdnormallink{Alan Skorkin}{http://www.skorks.com/2011/02/algorithms-a-dropbox-challenge-and-dynamic-programming/}. La primera es una solución exhaustiva. La segunda hace uso 
de Programación Dinámica. Asegúrese que entiende ambas soluciones e intente adaptarlas a su estilo y
mejorarlas.

\end{itemize}
\end{quote}
\end{it}

