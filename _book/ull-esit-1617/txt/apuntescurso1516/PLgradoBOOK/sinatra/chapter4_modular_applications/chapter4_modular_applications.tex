
Las aplicaciones normales Sinatra se denominan \cei{alicaciones clásicas sinatra}
y viven en \sinatrabase{Sinatra::Application},
que es una subclase de \sinatrabase{Sinatra::Base}.

En las aplicaciones clásicas Sinatra extiende la clase \verb|Object| en el momento 
de cargarse lo que, en cierto modo, contamina el espacio de nombres global. Eso dificulta que nuestra aplicación 
pueda ser distribuída como una gema y que se puedan tener varias aplicaciones clásicas 
en un único proceso.

Una aplicación Sinatra se dice una \cei{aplicación modular sinatra}
si no hace uso de \sinatrabase{Sinatra::Application},
renunciando al DSL de alto nivel proveído por Sinatra,
sino que hereda de \sinatrabase{Sinatra::Base}.

Podemos combinar una aplicación clásica  con una modular, pero sólo puede haber una aplicación clásica por proceso.

