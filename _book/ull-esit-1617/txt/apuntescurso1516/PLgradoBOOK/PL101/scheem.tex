\htmladdnormallink{http://nathansuniversity.com/scheem.html}{http://nathansuniversity.com/scheem.html}

\section{Scheem Interpreter}

\section{Variables}


\section{Setting Values}


\section{Putting Things Together}

\subsection{Unit Testing: Mocha}
\label{subsection:mocha}
\parrafo{Introducción}

Mocha is a feature-rich JavaScript test framework running on node.js
and the browser, making asynchronous testing simple and fun. Mocha tests
run serially, allowing for flexible and accurate
reporting, while mapping uncaught exceptions to the correct test cases.

\begin{itemize}
\item
\htmladdnormallink{http://mochajs.org/}{http://mochajs.org/}
\item
\htmladdnormallink{https://github.com/visionmedia/mocha}{https://github.com/visionmedia/mocha}
\item
\begin{itemize}
\item
An example setup for unit testing JavaScript in the browser with the Mocha testing framework and Chai assertions:
\htmladdnormallink{https://github.com/ludovicofischer/mocha-chai-browser-demo}{https://github.com/ludovicofischer/mocha-chai-browser-demo}
\item
\htmladdnormallink{Karma - a test runner}{http://karma-runner.github.io/0.10/index.html}
\end{itemize}
\end{itemize}

\parrafo{mocha init}

\begin{verbatim}
[~/srcPLgrado/mocha-chai-browser-demo(master)]$ mocha --help

  Usage: _mocha [debug] [options] [files]

  Commands:

    init <path>            initialize a client-side mocha setup at <path>

  Options:

    -h, --help                      output usage information
    -V, --version                   output the version number
    -r, --require <name>            require the given module
    -R, --reporter <name>           specify the reporter to use
    -u, --ui <name>                 specify user-interface (bdd|tdd|exports)
    -g, --grep <pattern>            only run tests matching <pattern>
    -i, --invert                    inverts --grep matches
    -t, --timeout <ms>              set test-case timeout in milliseconds [2000]
    -s, --slow <ms>                 "slow" test threshold in milliseconds [75]
    -w, --watch                     watch files for changes
    -c, --colors                    force enabling of colors
    -C, --no-colors                 force disabling of colors
    -G, --growl                     enable growl notification support
    -d, --debug                     enable node's debugger, synonym for node --debug
    -b, --bail                      bail after first test failure
    -A, --async-only                force all tests to take a callback (async)
    -S, --sort                      sort test files
    --recursive                     include sub directories
    --debug-brk                     enable node's debugger breaking on the first line
    --globals <names>               allow the given comma-delimited global [names]
    --check-leaks                   check for global variable leaks
    --interfaces                    display available interfaces
    --reporters                     display available reporters
    --compilers <ext>:<module>,...  use the given module(s) to compile files
    --inline-diffs                  display actual/expected differences inline within each string
    --no-exit                       require a clean shutdown of the event loop: mocha will not call process.exit
\end{verbatim}

\begin{verbatim}
[~/srcPLgrado]$ mocha init chuchu
[~/srcPLgrado]$ ls -ltr
total 16
....
drwxr-xr-x   6 casiano  staff  204 20 ene 11:16 chuchu
[~/srcPLgrado]$ tree chuchu/
chuchu/
|-- index.html
|-- mocha.css
|-- mocha.js
`-- tests.js
\end{verbatim}

\begin{verbatim}
[~/srcPLgrado/mocha-tutorial]$ cat test/test.js 
var assert = require("assert")
describe('Array', function(){
  describe('#indexOf()', function(){
    it('should return -1 when the value is not present', function(){
      assert.equal(-1, [1,2,3].indexOf(5));
      assert.equal(-1, [1,2,3].indexOf(0));
      assert.equal( 0, [1,2,3].indexOf(99));
    })
  })
})
\end{verbatim}

\begin{verbatim}
[~/srcPLgrado/mocha-tutorial]$ mocha
  .
  0 passing (5ms)
  1 failing

  1) Array #indexOf() should return -1 when the value is not present:
     AssertionError: 0 == -1
      at Context.<anonymous> (/Users/casiano/local/src/javascript/PLgrado/mocha-tutorial/test/test.js:7:14)
\end{verbatim}

Mocha allows you to use any assertion library you want, if it throws
an error, it will work! This means you can utilize libraries such as
should.js, node's regular assert module, or others. 


\parrafo{Browser support}

Mocha runs in the browser. 

\begin{itemize}
\item
Every release of Mocha will have new builds
of \verb|./mocha.js| and \verb|./mocha.css| for use in the browser. 

\item
To setup Mocha
for browser use all you have to do is include the script, stylesheet,
\item
Tell Mocha which interface you wish to use, and then 
\item
Run the tests. 
\end{itemize}
A
typical setup might look something like the following, where we call
\verb|mocha.setup('bdd')| to use the BDD interface before loading the test
scripts, running them onload with \verb|mocha.run()|.

\begin{verbatim}
<html>
<head>
  <meta charset="utf-8">
  <title>Mocha Tests</title>
  <link rel="stylesheet" href="mocha.css" />
</head>
<body>
  <div id="mocha"></div>
  <script src="jquery.js"></script>
  <script src="expect.js"></script>
  <script src="mocha.js"></script>

  <script>mocha.setup('bdd')</script>

  <script src="test.array.js"></script>
  <script src="test.object.js"></script>
  <script src="test.xhr.js"></script>

  <script>
    mocha.checkLeaks();
    mocha.globals(['jQuery']);
    mocha.run();
  </script>

</body>
</html>
\end{verbatim}

\begin{itemize}
\item
Mocha "interface" system allows developers to choose their style of
DSL. Shipping with BDD, TDD, and exports flavoured interfaces.
\item
\verb|mocha.globals([names ...])|

A list of accepted global variable names. For example, suppose your app deliberately exposes a global named app and YUI

\item
\verb|mocha.checkLeaks()|

By default Mocha will not check for global variables leaked while running tests
\end{itemize}


\parrafo{TDD}

The \cei{Mocha TDD interface} provides \verb|suite()|, \verb|test()|, \verb|setup()|, and \verb|teardown()|.

\begin{verbatim}
suite('Array', function(){
  setup(function(){
    // ...
  });

  suite('#indexOf()', function(){
    test('should return -1 when not present', function(){
      assert.equal(-1, [1,2,3].indexOf(4));
    });
  });
});
\end{verbatim}




\parrafo{Véase}


\begin{itemize}
\item
\htmladdnormallink{https://github.com/crguezl/nathanuniversityexercisesPL/tree/master/scheem8}{https://github.com/crguezl/nathanuniversityexercisesPL/tree/master/scheem8}
\end{itemize}


\subsection{Grunt}

\htmladdnormallink{http://gruntjs.com/getting-started}{http://gruntjs.com/getting-started}

\begin{verbatim}
npm install -g grunt-cli
\end{verbatim}

A typical setup will involve adding two files to your project: \verb|package.json| and the 
\verb|Gruntfile|.

\begin{itemize}
\item
\verb|package.json|: This file is used by \verb|npm| 
to store metadata for projects published as \verb|npm| modules. 

You will list grunt and the Grunt plugins your project needs as \cei{devDependencies} 
in this file.

\item
\verb|Gruntfile|: This file is named \verb|Gruntfile.js| or \verb|Gruntfile.coffee| and is used to configure or define tasks and load Grunt plugins.
\end{itemize}

\subsubsection{package.json}

\begin{itemize}
\item
The package.json file belongs in the root directory of your project,
next to the Gruntfile, and should be committed with your project
source. 

\item
Running npm install in the same folder as a package.json file
will install the correct version of each dependency listed therein.

\item
There are a few ways to create a package.json file for your project:

  \begin{itemize}
  \item
  Most grunt-init templates will automatically create a project-specific package.json file.
  \item
  The npm init command will create a basic package.json file.
  \item
  Start with the example below, and expand as needed, following this specification.
  \begin{verbatim}
  {
    "name": "my-project-name",
    "version": "0.1.0",
    "devDependencies": {
      "grunt": "~0.4.2",
      "grunt-contrib-jshint": "~0.6.3",
      "grunt-contrib-nodeunit": "~0.2.0",
      "grunt-contrib-uglify": "~0.2.2"
    }
  }
  \end{verbatim}
  \end{itemize}
\end{itemize}

\subsubsection{Gruntfile}

The Gruntfile.js or Gruntfile.coffee file is a valid JavaScript or
CoffeeScript file that belongs in the root directory of your project,
next to the package.json file, and should be committed with your project
source.

A Gruntfile is comprised of the following parts:

\begin{itemize}
\item
The "wrapper" function
\item
Project and task configuration
\item
Loading Grunt plugins and tasks
\item
Custom tasks
\end{itemize}


\parrafo{An example Gruntfile}

In the following Gruntfile, project metadata is imported into the Grunt
config from the project's \verb|package.json| file and the 

grunt-contrib-uglify

plugin's uglify task is configured to minify a source file and generate
a banner comment dynamically using that metadata. 

When grunt is run on
the command line, the uglify task will be run by default.

\begin{verbatim}
module.exports = function(grunt) {

  // Project configuration.
  grunt.initConfig({
    pkg: grunt.file.readJSON('package.json'),
    uglify: {
      options: {
        banner: '/*! <%= pkg.name %> <%= grunt.template.today("yyyy-mm-dd") %> */\n'
      },
      build: {
        src: 'src/<%= pkg.name %>.js',
        dest: 'build/<%= pkg.name %>.min.js'
      }
    }
  });

  // Load the plugin that provides the "uglify" task.
  grunt.loadNpmTasks('grunt-contrib-uglify');

  // Default task(s).
  grunt.registerTask('default', ['uglify']);

};
\end{verbatim}
Now that you've seen the whole Gruntfile, let's look at its component parts.

\parrafo{The "wrapper" function}

Every Gruntfile (and gruntplugin) uses this basic format, and all of your Grunt code must be specified inside this function:

\begin{verbatim}
module.exports = function(grunt) {
  // Do grunt-related things in here
};
\end{verbatim}

\parrafo{Project and task configuration}

Most Grunt tasks rely on configuration data defined in an object passed to the \verb|grunt.initConfig| method.

In this example, 
\verb|grunt.file.readJSON('package.json')| 
imports the JSON metadata stored in 
\verb|package.json| into the grunt config. 
Because \verb|<% %>| template strings may reference any config properties, configuration data like filepaths and file lists may be specified this way to reduce repetition.

You may store any arbitrary data inside of the configuration object,
and as long as it doesn't conflict with properties your tasks require,
it will be otherwise ignored. Also, because this is JavaScript, you're
not limited to JSON; you may use any valid JS here. You can even
programmatically generate the configuration if necessary.

Like most tasks, the 
\verb|grunt-contrib-uglify| plugin's uglify task expects its configuration 
to be specified in a property of the same name. 
Here, the \verb|banner| option is specified, along with a single uglify target 
named \verb|build| that minifies a single source file to a single destination file.

\begin{verbatim}
// Project configuration.
grunt.initConfig({
  pkg: grunt.file.readJSON('package.json'),
  uglify: {
    options: {
      banner: '/*! <%= pkg.name %> <%= grunt.template.today("yyyy-mm-dd") %> */\n'
    },
    build: {
      src: 'src/<%= pkg.name %>.js',
      dest: 'build/<%= pkg.name %>.min.js'
    }
  }
});
\end{verbatim}

\parrafo{A simple Grunt.js example}

\htmladdnormallink{https://github.com/UWMadisonUcomm/grunt-simple-example}{https://github.com/UWMadisonUcomm/grunt-simple-example}

\begin{verbatim}
[~/srcPLgrado/grunt-simple-example(master)]$ pwd
/Users/casiano/srcPLgrado/grunt-simple-example
[~/srcPLgrado/grunt-simple-example(master)]$ git remote -v
origin  git@github.com:UWMadisonUcomm/grunt-simple-example.git (fetch)
origin  git@github.com:UWMadisonUcomm/grunt-simple-example.git (push)
[~/srcPLgrado/grunt-simple-example(master)]$ ls
Gruntfile.js Readme.md    assets       index.html   node_modules package.json src
\end{verbatim}

\begin{verbatim}
[~/srcPLgrado/grunt-simple-example(master)]$ cat Gruntfile.js 
module.exports = function(grunt){
  grunt.initConfig({
    uglify: {
      main: {
        files: {
          'assets/app.min.js': [
            'src/javascripts/jquery-1.10.2.min.js',
            'src/javascripts/bootstrap.js',
            'src/javascripts/application.js'
          ]
        }
      }
    },
    less: {
      application: {
        options: {
          yuicompress: true
        },
        files: {
          "assets/app.min.css": "src/stylesheets/application.less"
        }
      }
    },
    watch: {
      javascripts: {
        files: ['src/javascripts/**/*'],
        tasks: ['uglify']
      },
      stylesheets: {
        files: ['src/stylesheets/**/*'],
        tasks: ['less']
      }
    }
  });

  // Load plugins
  grunt.loadNpmTasks('grunt-contrib-less');
  grunt.loadNpmTasks('grunt-contrib-uglify');
  grunt.loadNpmTasks('grunt-contrib-watch');

  // Register tasks
  grunt.registerTask('default', ['uglify', 'less']);
}
\end{verbatim}

\begin{verbatim}
[~/srcPLgrado/grunt-simple-example(master)]$ cat package.json 
{
  "name": "grunt-simple-example",
  "version": "0.0.1",
  "main": "index.js",
  "devDependencies": {
    "grunt": "~0.4.1",
    "grunt-contrib-cssmin": "~0.6.2",
    "grunt-contrib-less": "~0.7.0",
    "grunt-contrib-uglify": "~0.2.4",
    "grunt-contrib-watch": "~0.5.3"
  },
  "author": "Bryan Shelton",
  "license": "BSD-2-Clause"
}

\end{verbatim}

\begin{verbatim}
[~/srcPLgrado/grunt-simple-example(master)]$ npm install
npm WARN package.json grunt-simple-example@0.0.1 No repository field.
[~/srcPLgrado/grunt-simple-example(master)]$ 
\end{verbatim}


\begin{verbatim}
[~/srcPLgrado/grunt-simple-example(master)]$ grunt watch
Running "watch" task
Waiting...OK
>> File "src/javascripts/application.js" changed.

Running "uglify:main" (uglify) task
File "assets/app.min.js" created.

Done, without errors.
Completed in 3.897s at Mon Jan 20 2014 19:02:03 GMT+0000 (WET) - Waiting...
\end{verbatim}

\subsection{GitHub Project Pages}
\label{subsection:githubprojectpages}

Project Pages are kept in the same repository as the project they are for. 

These pages are similar to User and Org Pages, with a few slight differences:

\begin{itemize}
\item
The \verb|gh-pages| branch is used to build and publish from.
\item
A custom domain on user/org pages will apply the same domain redirect to all project pages hosted under that account, unless the project pages use their own custom domain.
\item
If no custom domain is used, the project pages are served under a subpath of the user pages: 
\begin{verbatim}
username.github.io/projectname
\end{verbatim}
Por ejemplo, mi usuario es \verb|crguezl|. Si el proyecto se llama \verb|nathanuniversityexercisesPL|, la dirección será:

\begin{center}
\htmladdnormallink{http://crguezl.github.io/nathanuniversityexercisesPL/}{http://crguezl.github.io/nathanuniversityexercisesPL/}
\end{center}
\item
Custom 404s will only work if a custom domain is used, otherwise the User Pages 404 is used.
\end{itemize}


\begin{itemize}
\item
\htmladdnormallink{Creating Project Pages manually}{https://help.github.com/articles/creating-project-pages-manually}
\end{itemize}

\begin{enumerate}
\item
Setting up Pages on a project requires a new "orphan" branch in your repository. The safest way to do this is to start with a fresh clone.

\begin{verbatim}
git clone https://github.com/user/repository.git
# Clone our repository
# Cloning into 'repository'...
remote: Counting objects: 2791, done.
remote: Compressing objects: 100% (1225/1225), done.
remote: Total 2791 (delta 1722), reused 2513 (delta 1493)
Receiving objects: 100% (2791/2791), 3.77 MiB | 969 KiB/s, done.
Resolving deltas: 100% (1722/1722), done.
\end{verbatim}
\item
Now that we have a clean repository, we need to create the new branch and remove all content from the working directory and index.

\begin{verbatim}
cd repository

git checkout --orphan gh-pages
# Creates our branch, without any parents (it's an orphan!)
# Switched to a new branch 'gh-pages'

git rm -rf .
# Remove all files from the old working tree
# rm '.gitignore'
\end{verbatim}
\item
Now we have an empty working directory. We can create some content in this branch and push it to GitHub. For example:
\begin{verbatim}
echo "My GitHub Page" > index.html
git add index.html
git commit -a -m "First pages commit"
git push origin gh-pages
\end{verbatim}
\end{enumerate}
