
\item
Escriba un analizador léxico en Jison 
(que suponemos se guardará en un fichero \verb|calc.l|) 
para una calculadora con números, restas, productos y menos unario

\item
\label{item:grammar}
Escriba el correspondiente analizador sintáctico en Jison 
(que suponemos se guardará en un fichero \verb|calc.jison|) 
para una calculadora con números, productos, restas y menos unario.
Deberá aceptar frases como \verb|-2.5|, \verb|-3.1e2-5e3|, \verb|-2*3|,
etc.


\item
Añada acciones semánticas a 
la gramática del ejercicio 
\ref{item:grammar}
para la evaluación de las expresiones
aritméticas. Puede contestar conjuntamente a este ejercicio 
y al ejercicio \ref{item:grammar} 

\item 
Explique que conflictos aparecen en el 
ejercicio
\ref{item:grammar} y como los ha resuelto
¿Cómo se resuelven los conflictos en yacc?
¿Cómo se da precedencia a las reglas y a los terminales?

\item 
Escriba el comando para generar el código javascript de la mini
calculadora a partir de los fuentes \verb|calc.jison| y
\verb|calc.l|

\item 
Calcule los FIRST para las variables sintácticas de dicha gramática
(Véase 
\ref{subsection:first})
\item 
Calcule los FOLLOW  para las variables sintácticas de dicha gramática
(Véase la sección
\ref{subsection:first})

\item
\label{item:dfa}
Calcule el DFA que reconoce los prefijos viables de dicha gramática
(Véase la sección 
\ref{section:conceptosbasicos}
y la sección
\ref{subsection:nfa2dfa})
\item 
Simule la antiderivacion a derechas/construción del árbol de análisis sintáctico
realizada usando el DFA construído en el ejercicio 
\ref{item:dfa}
sobre la 
entrada
'\verb|-1-2|'.

En cada momento de la simulación indique cual es la forma sentencial derecha actual y 
la posición en la misma (algo como $-e_\uparrow - NUM$),
en que posición de lectura de la entrada estamos 
(esto es, quién es el token lookahead que se está viendo), en que estado
del DFA estamos y cual es la acción tomada (desplazar o reducir).

