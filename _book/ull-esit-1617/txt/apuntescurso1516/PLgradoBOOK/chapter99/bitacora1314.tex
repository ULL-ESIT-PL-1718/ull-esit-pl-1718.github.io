\chapter{2014}

\section{01}
\subsection{Semana del 27/01/14 al 01/02/2014}
\begin{itemize}
\item Presentación de la Asignatura
\item Ejercicio: \htmladdnormallink{Darse de alta en la comunidad de google plus
PL Grado ULL 13/14}{https://plus.google.com/u/0/communities/107031495100582318205}
%{https://plus.google.com/u/0/communities/100856772699690495413"}
%\item Ejercicio: Indicar un mail en gmail para compartir recursos
%\item Ejercicio: Indicar página en GitHub
\item
\htmladdnormallink{JavaScript Review}{http://nathansuniversity.com/jsreview.html}
\item {\it Expresiones Regulares y Análisis Léxico en JavaScript} \ref{chapter:expresionesregularesyanalisslexico}
\item {\it Conversor de Temperaturas} \ref{sectionpractica:conversordetemperaturas}
\item {\it GitHub Project Pages} \ref{subsection:githubprojectpages}
\end{itemize}

\section{02}
\subsection{Semana del 4/02/14 al 7/02/2014}

\begin{itemize}
\item Martes 4/02.
{\it Comma Separated Values. CSV} Sección \ref{sectionpractica:csv}.
Secciones: Donde, Introducción al formato CSV, Ejemplo de ejecución,
Aproximación al análisis mediante expresiones regulares de CSV.
\end{itemize}

\subsection{Semana del 24/02/14 al 02/03/14. Repaso para el micro-examen del 05/03/14}

\begin{enumerate}

\item
¿Que retorna?
\begin{verbatim}
"hello small world and blue sky".match(/(\S+)\s+(\S+)/);
\end{verbatim}

\item
Indique que casa con el primer paréntesis y que con el segundo en las siguientes expresiones regulares:
\begin{verbatim}
> x = "I have 2 numbers: 53147"
> pats = [ /(.*)(\d*)/, 
           /(.*)(\d+)/, 
           /(.*?)(\d*)/, 
           /(.*?)(\d+)/, 
           /(.*)(\d+)$/, 
           /(.*?)(\d+)$/, 
           /(.*)\b(\d+)$/, 
           /(.*\D)(\d+)$/ ]
\end{verbatim}
Es decir, compute la salida de:
\begin{verbatim}
   pats.map( function(r) { return r.exec(x).slice(1); })
\end{verbatim}
\item
¿Que retorna el matching?:
\begin{verbatim}
>  a = "hola juan"
 => "hola juan" 
> a.match(/(?:hola )*(juan)/)
\end{verbatim}
\item ¿Que salidas se obtienen?
\begin{verbatim}
> "a\na".match(/a$/)
________________________________
> "a\na".match(/a$/m)
________________________________
> "a\na".match(/^a/gm)
____________
> "a\na".match(/^a/g)
_______
\end{verbatim}
\item
Escriba  la expresión regular que da lugar a este resultado (enumerar las líneas):
\begin{verbatim}
> x = "one\ntwo\nthree\nfour"
'one\ntwo\nthree\nfour'
> a = (c = 1, x.replace(_____, function(t) { return  c++ + ' ' + t; }))
'1 one\n2 two\n3 three\n4 four'
> console.log(a)
1 one
2 two
3 three
4 four
undefined
\end{verbatim}
\item
Supongamos dado el método
\begin{verbatim}
String.prototype.repeat = function( num ) {
    return new Array( num + 1 ).join( this );
}
\end{verbatim}
de manera que podamos escribir expresiones como:
\begin{verbatim}
> x = 'a'.repeat(40)
'aaaaaaaaaaaaaaaaaaaaaaaaaaaaaaaaaaaaaaaa'
\end{verbatim}
Encontremos una solución de la ecuación diofántica \verb|3x + 2y + 5z = 40|
\begin{verbatim}
> m = x.match(/^_______________________________$/).slice(1)
[ 'aaaaaaaaaaaaaaaaaaaaaaaaaaaaaaaaa',
  'aa',
  'aaaaa' ]
\end{verbatim}
Calculemos las longitudes de las tres cadenas:
\begin{verbatim}
> r = m.map(function(s) { return s.length; })
[ 33, 2, 5 ]
\end{verbatim}
Dividamos por los coeficientes para obtener la solución:
\begin{verbatim}
> coef = [3, 2, 5]
> i = 0; w = r.map(function(x) { return x/coef[i++]; }
[ 11, 1, 1 ]
\end{verbatim}
Encuentre la expresión regular usada.
\item 
Escriba una expresión regular que reconozca cadenas de dobles comillas como \verb|"hello world"|
y en las que las comillas puedan aparecer escapadas como en \verb|"Hello \"Jane\" and Jakes"|

\item
Escriba una expresión regular que reconozca los números en punto flotante como
\verb|2.34|, \verb|-5.2e-1| y \verb|0.9e3|

\item
\label{item:ccomments}
¿Que queda en \verb|m[0]|?
\begin{verbatim}
m = 'main() /* 1c */ { /* 2c */ return; /* 3c */ }'.match(new RegExp('/\\*.*\\*/'))
\end{verbatim}
¿Por qué?
\item 
¿Por qué debemos duplicar el carácter de escape \verb|\| en  la expresión regular \verb|new RegExp('/\\*.*\\*/')| de la pregunta anterior \ref{item:ccomments}?
\item
Se quiere poner un espacio en blanco después de la aparición de cada coma:
\begin{verbatim}
> 'ab,cd,4,3,   de,   fg'.replace(/,/, ', ')
=> "ab, cd, 4, 3,    de,    fg" 
\end{verbatim}
pero se quiere que la sustitución no tenga lugar si la coma esta incrustada entre
dos dígitos. Además se pide que si hay ya un espacio después de la coma,
no se duplique

Como función de reemplazo use:
\begin{verbatim}
f = function(match, p1, p2, offset, string) { return (p1 || p2 + " "); }
\end{verbatim}

\item
Escribe un patrón regular
que reconozca las cadenas  que representan números no primos en unario
de manera que el primer paréntesis case con el divisor mas grande del número.

\item
Escribe un patrón regular
que reconozca las cadenas  que representan números no primos en unario
de manera que el primer paréntesis case con el divisor mas pequeño del número.

\item Escriba una expresión regular que reconozca los comentarios del lenguaje JavaScript de la forma
\verb|// ...  |

\item Escriba una expresión regular que reconozca los comentarios del lenguaje JavaScript de la forma
\verb|/* ...  */|


\item Rellene lo que falta para que la salida sea la que aparece en la sesión de node:
\begin{verbatim}
> re = __________
> str = "John Smith"
'John Smith'
> newstr = str.replace(re, "______")
'Smith, John'
\end{verbatim}
\item  Rellene las partes que faltan:
\begin{verbatim}
> re = /d(b+)(d)/ig
/d(b+)(d)/gi
> z = "dBdxdbbdzdbd"
'dBdxdbbdzdbd'
> result = re.exec(z)
[ ______, _____, ______, index: __, input: 'dBdxdbbdzdbd' ]
> re.lastIndex
______
> result = re.exec(z)
[ ______, _____, ______, index: __, input: 'dBdxdbbdzdbd' ]
> re.lastIndex
______
> result = re.exec(z)
[ ______, _____, ______, index: __, input: 'dBdxdbbdzdbd' ]
> re.lastIndex
______
> result = re.exec(z)
_____
\end{verbatim}
\item Escriba la expresión regular \verb|r| para que produzca el resultado final:
\begin{verbatim}
> x = "hello"
> r = /l(___)/
> z = r.exec(x)
[ 'l', index: 3, input: 'hello' ]
\end{verbatim}
\item 
\begin{verbatim}
> z = "dBdDBBD"
> re = /d(b+)(d)/ig
> re.lastIndex = ________
> result = re.exec(z)
[ 'DBBD',
  'BB',
  'D',
  index: 3,
  input: 'dBdDBBD' ]
\end{verbatim}
\item  Conteste:
\begin{enumerate}
\item Explique que hace el siguiente fragmento de código:
\begin{verbatim}
> RegExp.prototype.bexec = function(str) {
...   var i = this.lastIndex;
...   var m = this.exec(str);
...   if (m && m.index == i) return m;
...   return null;
... }
[Function]
\end{verbatim}
\item Rellene las salidas que faltan:
\begin{verbatim}
> re = /d(b+)(d)/ig
/d(b+)(d)/gi
> z = "dBdXXXXDBBD"
'dBdXXXXDBBD'
> re.lastIndex = 3
> re.bexec(z)
_____________________________________________________
> re.lastIndex = 7
> re.bexec(z)
_____________________________________________________
\end{verbatim}
\end{enumerate}
\item 
Escriba una expresión JavaScript que permita reemplazar todas las apariciones de palabras repetidas en una String por una sóla aparición de la misma
\item 
Supongamos que se usa una función como segundo argumento de \verb|replace|.
¿Que argumentos recibe?
\item 
¿Cual es la salida?
\begin{verbatim}
> "bb".match(/b|bb/)

> "bb".match(/bb|b/)

\end{verbatim}

\item  El siguiente fragmento de código tiene por objetivo
escapar las entidades HTML para que no sean intérpretadas como código HTML.
Rellene las partes que faltan.
\begin{verbatim}
var entityMap = {
    "&": "&___;",
    "<": "&__;",
    ">": "&__;",
    '"': '&quot;',
    "'": '&#39;',
    "/": '&#x2F;'
  };

function escapeHtml(string) {
  return String(string).replace(/_________/g, function (s) {
    return ____________;
  });
\end{verbatim}
\item ¿Cual es la salida?
\begin{verbatim}
> a = [1,2,3]
[ 1, 2, 3 ]
> b = [1,2,3]
[ 1, 2, 3 ]
> a == b
________
\end{verbatim}
\item
¿Como se llama el método que permite obtener una representación como cadena de un objeto?
¿Que parámetros espera? ¿Como afectan dichos parámetros?
\item ¿Cual debe ser el valor del atributo \verb|rel| para usar la imagen como favicon?
\begin{verbatim}
<link rel="_____________" href="etsiiull.png" type="image/x-icon"> 
\end{verbatim}
\item
Escriba un código JavaScript que defina una clase \verb|Persona| con atributos \verb|nombre|
y \verb|apellidos| y que disponga de un método \verb|saluda|.
\item
Reescriba la solución al problema anterior haciendo uso del método \verb|template|
de  \verb|underscore| y ubicando el template dentro de un tag \verb1script1.
\item Rellene lo que falta:
\begin{verbatim}
[~/srcPLgrado/temperature/tests(master)]$ cat tests.js 
var assert = chai.______;

suite('temperature', function() {
    test('[1,{a:2}] == [1,2]', function() {
      assert._________([1, {a:2}], [1, {a:2}]);
    });
    test('5X = error', function() {
        original.value = "5X";
        calculate();
        assert._____(converted.innerHTML, /ERROR/);
    });
});
\end{verbatim}
% \item añadir sinatra app
\item
¿Cómo se llama el directorio por defecto desde el que una aplicación sinatra sirve los ficheros estáticos?
\item
Explique la línea:
\begin{verbatim}
set :public_folder, File.dirname(__FILE__) + '/starterkit'
\end{verbatim}
¿Que es \verb|__FILE__|? ¿Que es \verb|File.dirname(__FILE__)|?
¿Que hace el método \verb|set|? (Véase 
\htmladdnormallink{http://www.sinatrarb.com/configuration.html}{http://www.sinatrarb.com/configuration.html})
% takes a setting name and value and creates an attribute on the application object
\item Escriba un programa sinatra que cuando se visite la URI \verb|/chuchu| muestre
una página que diga \verb|"hello world!"|
\item
¿Cual es el signifcado de \verb|__END__| en un programa Ruby?
\item  
\label{sinatalayout}
Esta y las preguntas 
\ref{sinatraindex} y
\ref{sinatrachuchu} se refieren al mismo programa ruby sinatra.
Explique este fragmento de dicho programa ruby sinatra. 
\begin{verbatim}
@@layout
  <!DOCTYPE html>
  <html>
    <head>
        <meta charset="utf-8" />
        <title>Demo</title>
    </head>
    <body>
        <a href="http://jquery.com/">jQuery</a>
        <div class="result"></div>
        <script src="jquery.js"></script>
        <%= yield %>
    </body>
  </html>
\end{verbatim}
\begin{enumerate}
\item ¿En que lugar del fichero que contiene el programa está ubicada esta sección? 
\item ¿Cómo se llama el lenguaje en el que esta escrita esta sección?
\item ¿Para que sirve la sección \verb|layout|?
\item ¿Cual es la función del \verb|<div class="result"></div>|?
\item ¿Para que sirve el \verb|<%= yield %>|?
\end{enumerate}
\item 
\label{sinatraindex}
Explique este fragmento de un programa ruby sinatra.
\begin{verbatim}
@@index
  <script>
  $( document ).ready(function() {
      $( "a" ).click(function( event ) {
          event.preventDefault();
          $.get( "/chuchu", function( data ) {
            $( ".result" ).html( data );
            alert( "Load was performed." );
          });
      });
  });
  </script>
\end{verbatim}
\begin{enumerate}
\item ¿Cuando ocurre el evento \verb|ready|?
\item ¿Que hace \verb|event.preventDefault()|?
\item ¿Que hace \verb|$.get( "/chuchu", function( data ) { ... }|?
¿Cuando se dispara la callback?
\item ¿que hace la línea \verb|$( ".result" ).html( data )|?
\end{enumerate}
% \item añadir jquery ajax
\item Explique este fragmento de código ruby-sinatra:
\label{sinatrachuchu}
\begin{verbatim}
get '/chuchu' do
  if request.xhr? 
    "hello world!"
  else 
    erb :tutu
  end
end
\end{verbatim}
\item  En el siguiente programa - que calcula la conversión
de temperaturas entre grados Farenheit y Celsius - rellene las partes que faltan:
\begin{enumerate}
\item  index.html:
\begin{verbatim}
<html>
  <head>
      <meta http-equiv="Content-Type" content="text/html; charset=_____">
      <title>JavaScript Temperature Converter</title>
      <link ____="global.css" ___="stylesheet" ____="text/css">

     <script type="_______________" src="temperature.js"></script>
  </head>
  <____>
    <h1>Temperature Converter</h1>
    <table>
      <tr>
        <th>Enter  Temperature (examples: 32F, 45C, -2.5f):</th>
        <td><input id="________" ________="calculate();"></td>
      </tr>
      <tr>
        <th>Converted Temperature:</th>
        <td><span class="output" id="_________"></span></td>
      </tr>
    </table>
  </____>
</html>
\end{verbatim}

\item Rellene las partes del código JavaScript que faltan en \verb|temperature.js|:
\begin{verbatim}
"use strict"; // Use ECMAScript 5 strict mode in browsers that support it
function calculate() {
  var result;
  var original       = document.getElementById("________");
  var temp = original.value;
  var regexp = /_______________________________/;
  
  var m = temp.match(______);
  
  if (m) {
    var num = ____;  // paréntesis correspondiente
    var type = ____;
    num = parseFloat(num);
    if (type == 'c' || type == 'C') {
      result = (num * 9/5)+32;
      result = ______________________________ // 1 sólo decimal y el tipo
    }
    else {
      result = (num - 32)*5/9;
      result = ____________________________ // 1 sólo decimal y el tipo
    }
    converted._________ = result; // Insertar "result" en la página
  }
  else {
    converted._________ = "ERROR! Try something like '-4.2C' instead";
  }
}
\end{verbatim}
\end{enumerate}
\item  ¿Que hace \verb|autofocus|?
\begin{verbatim}
<td><textarea autofocus cols = "80" rows = "5" id="original"></textarea></td> 
\end{verbatim}
\item  ¿Que hacen las siguientes pseudo-clases estructurales CSS3?
\begin{verbatim}
tr:nth-child(odd)    { background-color:#eee; }
tr:nth-child(even)    { background-color:#00FF66; }
\end{verbatim}
\item ¿Que contiene el objeto \verb|window| en un programa JavaScript que se ejecuta en un navegador?

\item 
\begin{enumerate}
\item 
¿Que es \htmladdnormallink{Local Storage}{http://diveinto.html5doctor.com/storage.html}? ¿Que hace la siguiente línea?
\begin{verbatim}
  if (window.localStorage) localStorage.original  = temp;
\end{verbatim}
\item  ¿Cuando se ejecutará esta callback? ¿Que hace?
\begin{verbatim}
window.onload = function() {
  // If the browser supports localStorage and we have some stored data
  if (window.localStorage && localStorage.original) {
    document.getElementById("original").value = localStorage.original;
  }
};
\end{verbatim}
\end{enumerate}

\item  ¿Cómo se hace para que elementos de la página web permanezcan ocultos para 
posteriormente mostrarlos? ¿Que hay que hacer en el HTML, en la hoja de estilo y en el JavaScript?
\item Rellene los estilos para los elementos de las clases para que su visibilidad
case con la que su nombre indica:
\begin{verbatim}
.hidden      { display: ____; }
.unhidden    { display: _____; }
\end{verbatim}
\item 
Los siguientes textos corresponden  a los ficheros de 
la práctica 
de construcción de un analizador léxico de los ficheros de configuración INI. 
Rellena las partes que faltan.
\begin{enumerate}
\item  Rellena las partes que faltan en el contenido del fichero \verb|index.html|. 
Comenta que hace el tag \verb|<input>|.
Comenta que hace el tag \verb|<pre>|.
\begin{verbatim}
<html>
  <head>
     <meta http-equiv="Content-Type" content="text/html; charset=UTF-8">
     <title>INI files</title>
     <link href="global.css" rel="__________" type="text/css">

     <script type="_______________" src="underscore.js"></script>
     <script type="_______________" src="jquery.js"></script>
     <script type="_______________" src="______"></script>
  </head>
  <body>
    <h1>INI files</h1>
    <input type="file" id="_________" />
    <div id="out" class="hidden">
    <table>
      <tr><th>Original</th><th>Tokens</th></tr>
      <tr>
        <td>
          <pre class="input" id="____________"></pre>
        </td>
        <td>
          <pre class="output" id="___________"></pre>
        </td>
      </tr>
    </table>
    </div>
  </body>
</html>
\end{verbatim}

\item 
A continuación siguen los contenidos del fichero \verb|ini.js| conteniendo el JavaScript.
\begin{enumerate}
\item 
Rellena las partes que faltan. 
El siguiente ejemplo de fichero \verb|.ini| le puede ayudar
a recordar la parte de las expresiones regulares 
\begin{verbatim}
; last modified 1 April 2001 by John Doe
[owner]
name=John Doe
organization=Acme Widgets Inc.
\end{verbatim}
\item 
Explica 
el uso del template.
\item 
Explica el uso de JSON.stringify
\end{enumerate}
\begin{verbatim}
"use ______"; // Use ECMAScript 5 strict mode in browsers that support it

$(document)._____(function() {
   $("#fileinput").______(calculate);
});

function calculate(evt) {
  var f = evt.target.files[0]; 

  if (f) {
    var r = new __________();
    r.onload = function(e) { 
      var contents = e.target.______;
      
      var tokens = lexer(contents);
      var pretty = tokensToString(tokens);
      
      out.className = 'unhidden';
      initialinput._________ = contents;
      finaloutput._________ = pretty;
    }
    r.__________(f); // Leer como texto
  } else { 
    alert("Failed to load file");
  }
}

var temp = '<li> <span class = "<%= ______ %>"> <%= _ %> </span>\n';

function tokensToString(tokens) {
   var r = '';
   for(var i in tokens) {
     var t = tokens[i];
     var s = JSON.stringify(t, undefined, 2); //______________________________
     s = _.template(temp, {t: t, s: s});
     r += s;
   }
   return '<ol>\n'+r+'</ol>';
}

function lexer(input) {
  var blanks         = /^___/;
  var iniheader      = /^________________/;
  var comments       = /^________/;
  var nameEqualValue = /^________________________/;
  var any            = /^_______/;

  var out = [];
  var m = null;

  while (input != '') {
    if (m = blanks.____(input)) {
      input = input.substr(m.index+___________);
      out.push({ type : ________, match: _ });
    }
    else if (m = iniheader.exec(input)) {
      input = input.substr(___________________);
      _______________________________________ // avanzemos en input
    }
    else if (m = comments.exec(input)) {
      input = input.substr(___________________);
      _________________________________________
    }
    else if (m = nameEqualValue.exec(input)) {
      input = input.substr(___________________);
      _______________________________________________
    }
    else if (m = any.exec(input)) {
      _______________________________________
      input = '';
    }
    else {
      alert("Fatal Error!"+substr(input,0,20));
      input = '';
    }
  }
  return out;
}
\end{verbatim}
\end{enumerate}


\end{enumerate}


\section{Proyecto: Diseña e Implementa un Lenguaje de Dominio Específico}

Se trata de realizar un proyecto relacionado con el procesamiento de lenguajes.
El objetivo puede ser:


\begin{enumerate}
\item
Diseñar un lenguaje de dominio específico para simplificar cualquier tarea en la que estés interesado:

\begin{itemize}
\item
Para escribir exámenes, 
\begin{itemize}
\item Por ejemplo se puede escribir un traductor para el formato Moodle gift que traduzca a javascript + HTML + css y que evalúe al usuario
\item Por ejemplo se puede escribir un traductor para el formato Moodle XML que traduzca a javascript + HTML + css y que evalúe al usuario
\item
\end{itemize}
\item
Para dibujar árboles, 
\item
Para calcular fechas,
\item
Para generar emails
\item
Para escribir música
\item
Para escribir autómatas finitos
\item 
Para procesar 
\htmladdnormallink{CSS}{http://www.w3.org/TR/CSS21/grammar.html\#grammar}
\item
etc.
\end{itemize}
\item
Estudiar un traductor existente en profundidad  como:
\begin{itemize}
\item 
ECMAscript 5.1: 
\htmladdnormallink{Creating a JavaScript Parser}{http://cjihrig.com/blog/creating-a-javascript-parser/} Una implementación de ECMAScript 5.1 usando Jison 
disponible 
en GitHub
 en
\htmladdnormallink{https://github.com/cjihrig/jsparser}{https://github.com/cjihrig/jsparser}.
Puede probarse en:
\htmladdnormallink{http://www.cjihrig.com/development/jsparser/}{http://www.cjihrig.com/development/jsparser/}
\item Roy
\item CoffeScript
\item Jison
\item 
\htmladdnormallink{Javascript 1.4}{http://www-archive.mozilla.org/js/language/grammar14.html}
\item etc.
\end{itemize}
\item
También puedes proponer tu propio tema relacionado al profesor
\end{enumerate}

Se recomienda para ello organizar equipos de no menos de dos y no mas de cuatro.

Las presentaciones de los proyectos tendrán lugar el último día de clase Martes 21 de Mayo.

\section{03}

\section{04}

\subsection{Semana del 07/04/14 al 11/04/14. Repaso para el micro-examen del 09/04/14}

\begin{rawhtml}

<html>
  <head>
    <meta charset="utf-8">
    <meta http-equiv="X-UA-Compatible" content="IE=edge">
    <meta name="viewport" content="width=device-width, initial-scale=1">
    <title>REPASO. 2ª PARTE. <br />
ANÁLISIS SINTÁCTICO</title>
    <!-- Latest compiled and minified CSS -->
    <link rel="stylesheet" href="http://netdna.bootstrapcdn.com/bootstrap/3.1.1/css/bootstrap.min.css">
    <!-- -->
    <style type="text/css" media="all">
      div.quiz, p.comment, div.explanation {display:inline;}
      strong.correct {color:#14B63F;}
      strong.incorrect {color:rgb(255,0,0);}
      strong.mark {color:rgb(255,128,0);}
      input.correct {color:#14B63F; font-weight: bold;}
      input.incorrect {color:rgb(255,0,0); font-weight: bold;}
      div.custom-header {background-color:#7a3b7a; color:#FFFFFF; font-weight: bold;}
      body {padding-top: 20px; padding-bottom: 20px; margin-bottom: 20px;}
      html {position: relative; min-height: 100%;}
      .navbar {margin-bottom: 20px;}
      .btn-footer {text-align: center;}
      #footer {
        position: absolute;
        bottom: 0;
        height: 30px;
        width: inherit;
      }
      div.container {background-color:#EAEAEA;}
      div.explanation {font-style: italic;}
      .nav>li>a.links {
        display: inline; 
        padding: 0; 
        color: #FFFFFF;
      }
      .nav>li>a.links:hover {color: #FFCCFF;}
      /* Inputs size */
      input.size-3 { width: 3em; }

input.size-4 { width: 4em; }

input.size-5 { width: 5em; }

input.size-6 { width: 6em; }

input.size-7 { width: 7em; }

input.size-8 { width: 8em; }

input.size-9 { width: 9em; }

input.size-10 { width: 10em; }

input.size-11 { width: 11em; }

input.size-12 { width: 12em; }

input.size-13 { width: 13em; }

input.size-14 { width: 14em; }

input.size-15 { width: 15em; }

    </style>
    <!-- Any CSS included by the user -->
    
    <!-- Mathjax -->
    
        <script type=text/javascript src=http://cdn.mathjax.org/mathjax/latest/MathJax.js?config=TeX-AMS-MML_HTMLorMML></script>
        <script type=text/javascript>
          MathJax.Hub.Config({tex2jax: {inlineMath: [['$','$'], ['\\(','\\)']]}});
        </script>
      
    <!--         -->
  </head>
  <body>
    <div class="container">
      
        <div class="navbar navbar-default custom-header">
          <div class="navbar-header">
            <h3>REPASO. 2ª PARTE. <br />
ANÁLISIS SINTÁCTICO</h3>
          </div>
          <div class="collapse navbar-collapse" id="bs-example-navbar-collapse-1">
            <ul class="nav navbar-nav navbar-right">
              <li><a href="http://tinyurl.com/pl1314" class="links">Procesadores de Lenguajes</a></li><br/>
              <li><a href="http://crguezl.github.io/pl-html" class="links">Apuntes</a></li><br/>
              <li>
                <a href="http://www.ull.es/view/centros/etsii/Inicio/es" class="links">
                  Escuela T&eacute;cnica Superior de Ingenier&iacute;a Inform&aacute;tica</a>
                |
                <a href="http://www.ull.es/" class="links">Universidad de la Laguna</a>
              </li>
            </ul>
          </div>
        </div>
      
      <!-- Seed: 87976226517832053849413405508974334473 -->
<form id="form">
  <ol class="questions">
    <li id="question-0" class="question fillin ">
      <div class="quiz text">
[1 point] 
Dado un conjunto $A$, se define $A^*$ el cierre de Kleene de $A$ como:
\( A^* = \cup_{n=1}^{\infty} A^n \)
Se admite que $A^0 = { \epsilon }$, donde $\epsilon$ denota la
<input type=text id=qfi1-1 class='fillin size-15'></input> <input type=text id=qfi1-2 class='fillin size-3'></input> esto es
la palabra que tiene longitud cero, formada por cero símbolos del conjunto base $A$<div id=qfi1-2r class=quiz></div></br></br>      </div>
    </li>
    <li id="question-1" class="question fillin ">
      <div class="quiz text">
[1 point] 
Una producción de la forma $A \rightarrow A \alpha$.
se dice que es <input type=text id=qfi2-1 class='fillin size-9'></input> por la <input type=text id=qfi2-2 class='fillin size-9'></input>
<div id=qfi2-2r class=quiz></div></br></br>      </div>
    </li>
    <li id="question-2" class="question fillin ">
      <div class="quiz text">
[1 point] Encuentre una gramática equivalente a esta:
      <pre>
        A: A 'a' | 'b'
      </pre>
      pero que no sea recursiva por la izquierda:
      <pre>
        A: <input type=text id=qfi3-1 class='fillin size-5'></input>
        R: /* vacío */ | <input type=text id=qfi3-2 class='fillin size-5'></input>
      </pre>
    <div id=qfi3-2r class=quiz></div></br></br>      </div>
    </li>
    <li id="question-3" class="question fillin ">
      <div class="quiz text">
[1 point] 
Recuerde el <b>analizador sintáctico descendente predictivo recursivo</b> 
para la <a id="grammar">gramática</a>:<br/>
<ul>
  <li> $\Sigma = \{ ; =, ID, P, ADDOP, MULOP, COMPARISON, (, ), NUM \}$
  <li> $V = \{ statements, statement, condition, expression, term, factor \}$
  <li> Productions:
  <ol>
    <li>
    statements  $ \rightarrow$ statement ';' statements  $\vert$ statement
    <li>
    statement  $ \rightarrow$ ID '=' expression  $\vert$ P expression
 $ \vert$ IF condition THEN statement·
    <li> condition $ \rightarrow$ expression COMPARISON expression
    <li>
    expression  $ \rightarrow$ term ADDOP expression  $\vert$ term
    <li>
    term  $ \rightarrow$ factor MULOP term  $\vert$ factor
    <li>
    factor  $ \rightarrow$ '(' expression ')' $\vert$ ID $ \vert$ NUM
  </ol>
  <li> Start symbol: $statements$
</ul>
Rellene las partes que faltan de código CoffeeScript del 
método que se encarga de reconocer el lenguaje generado
por <tt>expression</tt>:
<pre>
  expression = ->
    result = term()
    while <input type=text id=qfi4-1 class='fillin size-9'></input> and <input type=text id=qfi4-2 class='fillin size-14'></input> is "ADDOP"
      type = <input type=text id=qfi4-3 class='fillin size-9'></input>.<input type=text id=qfi4-4 class='fillin size-5'></input>
      match "ADDOP"
      right = <input type=text id=qfi4-5 class='fillin size-6'></input>
      result =
        type: <input type=text id=qfi4-6 class='fillin size-4'></input>
        left: result
        right: right
    result
</pre>
<div id=qfi4-6r class=quiz></div></br></br>      </div>
    </li>
    <li id="question-4" class="question fillin ">
      <div class="quiz text">
[3 points] 
Rellene las partes que faltan de código CoffeeScript del 
método que se encarga de reconocer el lenguaje generado
por <tt>statement</tt> para la <a href="#grammar">gramática
definida anteriormente</a>:
<pre>
  statement = ->
    result = null
    if <input type=text id=qfi5-1 class='fillin size-9'></input> and <input type=text id=qfi5-2 class='fillin size-9'></input>.<input type=text id=qfi5-3 class='fillin size-4'></input> is "ID"
      left =
        type: "ID"
        value: <input type=text id=qfi5-4 class='fillin size-9'></input>.<input type=text id=qfi5-5 class='fillin size-5'></input>

      match "ID"
      match "="
      right = <input type=text id=qfi5-6 class='fillin size-10'></input>()
      result =
        type: "="
        left: left
        right: right
    else if lookahead and lookahead.type is "P"
      match "P"
      right = <input type=text id=qfi5-7 class='fillin size-12'></input>
      result =
        type: "P"
        value: right
    else if lookahead and lookahead.type is "IF"
      match "IF"
      left = <input type=text id=qfi5-8 class='fillin size-11'></input>
      match "THEN"
      right = <input type=text id=qfi5-9 class='fillin size-11'></input>
      result =
        type: "IF"
        left: left
        right: right
    else # Error!
      throw "Syntax Error. Expected identifier but found " + 
        (if lookahead then lookahead.value else "end of input") + 
        " near '#{input.substr(lookahead.from)}'"
    result

</pre>
<div id=qfi5-9r class=quiz></div></br></br>      </div>
    </li>
    <li id="question-5" class="question fillin ">
      <div class="quiz text">
[1 point] Rellene las partes que faltan del código CoffeeScript
que reconoce el sublenguaje generado por <i>condition</i>:
  <pre>
  condition = ->
    left = <input type=text id=qfi6-1 class='fillin size-11'></input>
    type = <input type=text id=qfi6-2 class='fillin size-9'></input>.<input type=text id=qfi6-3 class='fillin size-5'></input>
    match "COMPARISON"
    right = <input type=text id=qfi6-4 class='fillin size-10'></input>()
    result =
      type: type
      left: left
      right: right
    result
  </pre>
<div id=qfi6-4r class=quiz></div></br></br>      </div>
    </li>
    <li id="question-6" class="question fillin ">
      <div class="quiz text">
[1 point] 
Complete este fragmento de <tt>slim</tt> que establece el favicon de 
la página HTML:
<pre>
    link rel="<input type=text id=qfi7-1 class='fillin size-13'></input>" type="image/jpg" href="images/favicon.jpg"
</pre><div id=qfi7-1r class=quiz></div></br></br>      </div>
    </li>
    <li id="question-7" class="question fillin ">
      <div class="quiz text">
[6 points] 
      Para que un repositorio con una aplicación escrita en Ruby-Sinatra
      pueda desplegarse en Heroku con nombre <tt>chuchu</tt> el primer comando   que debemos escribir es:
      <pre>
      heroku <input type=text id=qfi8-1 class='fillin size-6'></input> <input type=text id=qfi8-2 class='fillin size-6'></input>
      </pre>
      Este comando crea un remoto git cuyo nombre es <input type=text id=qfi8-3 class='fillin size-6'></input>
      y cuya URL es 
      <pre>
        git@heroku.com:<input type=text id=qfi8-4 class='fillin size-7'></input>.git
      </pre>
      La URL de publicación/despliegue será:
      http://<input type=text id=qfi8-5 class='fillin size-9'></input>.<input type=text id=qfi8-6 class='fillin size-9'></input>.com/
      <br/>
      Una vez que todo esta listo, para publicar nuestra versión
      en la rama <tt>master</tt> en heroku debemos ejecutar el comando:
      <pre>
      git <input type=text id=qfi8-7 class='fillin size-4'></input> <input type=text id=qfi8-8 class='fillin size-6'></input> master
      <br/>
      </pre>
      Si la versión que queremos publicar en heroku no está en la rama
      <tt>master</tt> sino que está en la rama <tt>tutu</tt> deberemos 
      modificar el comando anterior:
      <pre>
      git push <input type=text id=qfi8-9 class='fillin size-6'></input> <input type=text id=qfi8-10 class='fillin size-4'></input>:<input type=text id=qfi8-11 class='fillin size-6'></input>
      </pre>
      Para ver los logs deberemos emitir el comando:
      <pre>
       heroku <input type=text id=qfi8-12 class='fillin size-4'></input>
      </pre>
    <div id=qfi8-12r class=quiz></div></br></br>      </div>
    </li>
    <li id="question-8" class="question fillin ">
      <div class="quiz text">
[1 point] Con que subcomando del cliente <tt>heroku</tt> abro el navegador
    en la URL del proyecto?<br/>
    <pre>
    heroku <input type=text id=qfi9-1 class='fillin size-4'></input>
    </pre><div id=qfi9-1r class=quiz></div></br></br>      </div>
    </li>
    <li id="question-9" class="question fillin ">
      <div class="quiz text">
[1 point] 
      Escriba la parte que falta para que 
      el programa PEGJS reconozca el lenguaje
      $\{ a^n b^n c^n\ /\ n \ge{} 1\}$
      <pre>
      S = <input type=text id=qfi10-1 class='fillin size-8'></input> 'a'+ B !('a'/'b'/'c')
      A = 'a' A? 'b'
      B = 'b' B? 'c'
      </pre>
    <div id=qfi10-1r class=quiz></div></br></br>      </div>
    </li>
    <li id="question-10" class="question multiplechoice ">
      <div class="quiz text">
[1 point] 
Dado el PEGjs<a id="pegif"></a>:
<pre>
S =   if C:C then S1:S else S2:S { return [ 'ifthenelse', C, S1, S2 ]; }
    / if C:C then S:S            { return [ 'ifthen', C, S]; }
    / O                          { return 'O'; }
_ = ' '*
C = _'c'_                        { return 'c'; }
O = _'o'_                        { return 'o'; }
else = _'else'_                 
if = _'if'_
then = _'then'_    
</pre>
Considere esta entrada:
<pre>
if c then if c then o else o
</pre>
<!-- ['ifthen', 'c', ['ifthenelse', 'c', 'o', 'o']] -->
¿Cuál de los dos árboles es construido para la misma?:
<br></br>      </div>
      <ol class="answers">
        <input type="radio" id="qmc11-1" name="qmc11" class="select">
<tt>['ifthen', 'c', ['ifthenelse', 'c', 'o', 'o']]</tt><br class=qmc11-1br>        </input>
        <div id="qmc11-1r" class="quiz">
        </div>
        <input type="radio" id="qmc11-2" name="qmc11" class="select">
<tt>['ifthenelse', 'c', ['ifthen', 'c', 'o'], 'o']]</tt><br class=qmc11-2br>        </input>
        <div id="qmc11-2r" class="quiz">
        </div>
        <br/>
      </ol>
    </li>
    <li id="question-11" class="question multiplechoice ">
      <div class="quiz text">
[1 point] Si en el <a href="#pegif">peg anterior</a> cambiamos el orden de las dos primeras reglas de <tt>S</tt>:
<pre>
  S =   if C:C then S:S            { return [ 'ifthen', C, S]; }
      / if C:C then S1:S else S2:S { return [ 'ifthenelse', C, S1, S2 ]; }
</pre>
Para la misma entrada:
<pre>
if c then if c then o else o
</pre>
<!-- ['ifthen', 'c', ['ifthenelse', 'c', 'o', 'o']] -->
¿Cuál de las respuestas es correcta?
    <br></br>      </div>
      <ol class="answers">
        <input type="radio" id="qmc12-1" name="qmc12" class="select">
<tt>['ifthenelse', 'c', ['ifthen', 'c', 'o'], 'o']]</tt><br class=qmc12-1br>        </input>
        <div id="qmc12-1r" class="quiz">
        </div>
        <input type="radio" id="qmc12-2" name="qmc12" class="select">
<tt>Syntax Error</tt>. La frase no es aceptada por el peg<br class=qmc12-2br>        </input>
        <div id="qmc12-2r" class="quiz">
        </div>
        <input type="radio" id="qmc12-3" name="qmc12" class="select">
<tt>['ifthen', 'c', ['ifthenelse', 'c', 'o', 'o']]</tt><br class=qmc12-3br>        </input>
        <div id="qmc12-3r" class="quiz">
        </div>
        <br/>
      </ol>
    </li>
    <li id="question-12" class="question fillin ">
      <div class="quiz text">
[1 point] 
Rellene las partes que faltan de este código para que funcione:
<pre>
var PEG = require ("pegjs");
var grammar = "s = ('a' / 'b')+";
var parser = PEG.<input type=text id=qfi13-1 class='fillin size-11'></input>(grammar);
var input = process.argv[<input type=text id=qfi13-2 class='fillin size-3'></input>] || 'abba';
console.log(parser.parse(input))
</pre>
Cuando se ejecuta, este código produce:
<pre>
[~/srcPLgrado/pegjs/examples(master)]$ node abba.pegjs abb
[ 'a', 'b', 'b' ]
</pre>
    <div id=qfi13-2r class=quiz></div></br></br>      </div>
    </li>
    <li id="question-13" class="question fillin ">
      <div class="quiz text">
[1 point] 
<a id="anbncn"></a>
Complete las partes que faltan para que el PEGjs reconozca este
clásico ejemplo de lenguaje que no es independiente del contexto
               $\{ a^nb^nc^n / n \ge{} 1 \}$
<pre>
S = <input type=text id=qfi14-1 class='fillin size-3'></input>(<input type=text id=qfi14-2 class='fillin size-3'></input> <input type=text id=qfi14-3 class='fillin size-3'></input>) 'a'+ B:<input type=text id=qfi14-4 class='fillin size-3'></input> !('c'/[<input type=text id=qfi14-5 class='fillin size-3'></input>]) { return B; }
A = 'a' A:A? 'b' { if (A) { return A+1; } else return 1; }
B = 'b' B:B? 'c' { if (B) { return B+1; } else return 1; }
</pre>
    <div id=qfi14-5r class=quiz></div></br></br>      </div>
    </li>
    <li id="question-14" class="question fillin ">
      <div class="quiz text">
[1 point] 
rellene las partes que faltan del siguiente programa PEGjs que reconoce
los comentarios Pascal:
<pre>
P     =   prog:N+                          { return prog; }
N     =   chars:$(!Begin ANY)+             { return chars;}
        / C
C     = Begin chars:<input type=text id=qfi15-1 class='fillin size-3'></input> End                { return chars.join(''); }
T     =   C 
        / (!<input type=text id=qfi15-2 class='fillin size-5'></input> <input type=text id=qfi15-3 class='fillin size-4'></input> char:ANY)           { return char;}
Begin = '(*'
End   = '*)'
ANY   =   'z'    /* any character */       { return 'z';  }
        / char:<input type=text id=qfi15-4 class='fillin size-4'></input>                        { return char; }    
</pre>
    <div id=qfi15-4r class=quiz></div></br></br>      </div>
    </li>
    <li id="question-15" class="question fillin ">
      <div class="quiz text">
[4 points] 
Rellene las partes que faltan de esta clase que implementa 
persistencia para programas PL0 usando el ORM DataMapper:
<pre>
DataMapper.<input type=text id=qfi16-1 class='fillin size-5'></input>(:default,·
                 ENV['DATABASE_URL'] || "sqlite3://#{Dir.pwd}/database.db" )

class PL0Program
  include <input type=text id=qfi16-2 class='fillin size-10'></input>::<input type=text id=qfi16-3 class='fillin size-8'></input>
··
  <input type=text id=qfi16-4 class='fillin size-8'></input> :name, String, :key => true
  <input type=text id=qfi16-5 class='fillin size-8'></input> :source, String, :length => 1..1024
end

  DataMapper.<input type=text id=qfi16-6 class='fillin size-8'></input>
  DataMapper.<input type=text id=qfi16-7 class='fillin size-13'></input>
</pre>
<div id=qfi16-7r class=quiz></div></br></br>      </div>
    </li>
    <li id="question-16" class="question fillin ">
      <div class="quiz text">
[1 point] 
Rellene las partes que faltan del siguiente fragmento de código
de la ruta <tt>/save</tt>
que guarda el programa solicitado:
<pre>
post '/save' do
  name = params[:fname]
  c  = PL0Program.<input type=text id=qfi17-1 class='fillin size-5'></input>(:name => name)
  if c
    c.source = params["input"]
    c.<input type=text id=qfi17-2 class='fillin size-4'></input>
  else
    c = PL0Program.new
    c.name = params["fname"]
    c.source = params["input"]
    c.<input type=text id=qfi17-3 class='fillin size-4'></input>
  end
  <input type=text id=qfi17-4 class='fillin size-8'></input> '/'
end
</pre>
<div id=qfi17-4r class=quiz></div></br></br>      </div>
    </li>
    <li id="question-17" class="question fillin ">
      <div class="quiz text">
[4 points] 
En la práctica del PEGjs tratabamos las expresiones aritméticas 
mediante estas dos reglas:
<pre>
exp    = t:term   r:(ADD term)*   { return tree(t,r); }
term   = f:factor r:(MUL factor)* { return tree(f,r); }
ADD      = _ op:[+-] _ { return op; }
MUL      = _ op:[*/] _ { return op; }
</pre>
Complete el código de <tt>tree</tt>:
<pre>
{
  var tree = function(f, r) {
    if (r.<input type=text id=qfi18-1 class='fillin size-6'></input> > 0) {
      var last = r.<input type=text id=qfi18-2 class='fillin size-3'></input>();
      var result = {
        type:  <input type=text id=qfi18-3 class='fillin size-7'></input>,
        left: <input type=text id=qfi18-4 class='fillin size-10'></input>,
        right: <input type=text id=qfi18-5 class='fillin size-7'></input>
      };
    }
    else {
      var result = f;
    }
    return result;
  }
}
</pre>
<div id=qfi18-5r class=quiz></div></br></br>      </div>
    </li>
  </ol>
  <div class="btn-footer">
    <button type="button" id="submit" class="btn btn-primary">
Submit    </button>
    <button type="button" id="reset" class="btn btn-warning">
Reset    </button>
    <button type="button" id="deletestorage" class="btn btn-danger">
Delete storage    </button>
  </div>
</form>

      
        <div id="footer">
          <div class="row text-muted">
            <div class="col-md-6">
              Quiz generated using the RuQL Gem. For more information, visit <a href="https://github.com/jjlabrador/ruql">GitHub
              <span class="glyphicon glyphicon-link"></span></a>
            </div>
            <div class="col-md-3"></div>
            <div class="col-md-3">
              Universidad de La Laguna 2014
            </div>
          </div>
        </div>
      
    </div>
    <!-- JavaScripts -->
    <!-- jQuery -->
    
        <script type=text/javascript src=http://code.jquery.com/jquery-2.1.0.min.js></script>
        <!--[if lt IE 8]>
          <script type=text/javascript src=http://code.jquery.com/jquery-1.11.0.min.js></script>
        <![endif]-->
      
    <!-- Codehelper -->
    <script type=text/javascript src=http://www.codehelper.io/api/ips/?js></script>
    <!-- Internationalization -->
    <script>      i18n = {};
      i18n['ES'] = {}
      i18n['ES']['correct'] = "Correcto";
      i18n['ES']['incorrect'] = "Incorrecto";
      i18n['ES']['points'] = "puntos";
      i18n['EN'] = {}
      i18n['EN']['correct'] = "Correct";
      i18n['EN']['incorrect'] = "Incorrect";
      i18n['EN']['points'] = "points";
</script>
    <!-- XRegexp -->
    <script type=text/javascript src=http://cdnjs.cloudflare.com/ajax/libs/xregexp/2.0.0/xregexp-min.js></script>
    <!-- Form validation -->
    <script>      data = {"question-0":{"question_text":"\nDado un conjunto $A$, se define $A^*$ el cierre de Kleene de $A$ como:\n\\( A^* = \\cup_{n=1}^{\\infty} A^n \\)\nSe admite que $A^0 = { \\epsilon }$, donde $\\epsilon$ denota la\n--------------- --- esto es\nla palabra que tiene longitud cero, formada por cero símbolos del conjunto base $A$.","answers":{"qfi1-1":{"answer_text":"/palabra/i","correct":true,"explanation":null,"type":"Regexp"},"qfi1-2":{"answer_text":"/vac[ií]a/i","correct":true,"explanation":null,"type":"Regexp"}},"points":1,"order":true,"question_comment":""},"question-1":{"question_text":"\nUna producción de la forma $A \\rightarrow A \\alpha$.\nse dice que es --------- por la ---------\n","answers":{"qfi2-1":{"answer_text":"/recursiva/i","correct":true,"explanation":null,"type":"Regexp"},"qfi2-2":{"answer_text":"/izquierda/i","correct":true,"explanation":null,"type":"Regexp"}},"points":1,"order":true,"question_comment":""},"question-2":{"question_text":"Encuentre una gramática equivalente a esta:\n      <pre>\n        A: A 'a' | 'b'\n      </pre>\n      pero que no sea recursiva por la izquierda:\n      <pre>\n        A: -----\n        R: /* vacío */ | -----\n      </pre>\n    ","answers":{"qfi3-1":{"answer_text":"/'b'\\s*R/","correct":true,"explanation":null,"type":"Regexp"},"qfi3-2":{"answer_text":"/'a'\\s*R/","correct":true,"explanation":null,"type":"Regexp"}},"points":1,"order":true,"question_comment":""},"question-3":{"question_text":"\nRecuerde el <b>analizador sintáctico descendente predictivo recursivo</b> \npara la <a id=\"grammar\">gramática</a>:<br/>\n<ul>\n  <li> $\\Sigma = \\{ ; =, ID, P, ADDOP, MULOP, COMPARISON, (, ), NUM \\}$\n  <li> $V = \\{ statements, statement, condition, expression, term, factor \\}$\n  <li> Productions:\n  <ol>\n    <li>\n    statements  $ \\rightarrow$ statement ';' statements  $\\vert$ statement\n    <li>\n    statement  $ \\rightarrow$ ID '=' expression  $\\vert$ P expression\n $ \\vert$ IF condition THEN statement·\n    <li> condition $ \\rightarrow$ expression COMPARISON expression\n    <li>\n    expression  $ \\rightarrow$ term ADDOP expression  $\\vert$ term\n    <li>\n    term  $ \\rightarrow$ factor MULOP term  $\\vert$ factor\n    <li>\n    factor  $ \\rightarrow$ '(' expression ')' $\\vert$ ID $ \\vert$ NUM\n  </ol>\n  <li> Start symbol: $statements$\n</ul>\nRellene las partes que faltan de código CoffeeScript del \nmétodo que se encarga de reconocer el lenguaje generado\npor <tt>expression</tt>:\n<pre>\n  expression = ->\n    result = term()\n    while --------- and -------------- is \"ADDOP\"\n      type = ---------.-----\n      match \"ADDOP\"\n      right = ------\n      result =\n        type: ----\n        left: result\n        right: right\n    result\n</pre>\n","answers":{"qfi4-1":{"answer_text":"lookahead","correct":true,"explanation":null,"type":"String"},"qfi4-2":{"answer_text":"lookahead.type","correct":true,"explanation":null,"type":"String"},"qfi4-3":{"answer_text":"lookahead","correct":true,"explanation":null,"type":"String"},"qfi4-4":{"answer_text":"value","correct":true,"explanation":null,"type":"String"},"qfi4-5":{"answer_text":"term()","correct":true,"explanation":null,"type":"String"},"qfi4-6":{"answer_text":"type","correct":true,"explanation":null,"type":"String"}},"points":1,"order":true,"question_comment":""},"question-4":{"question_text":"\nRellene las partes que faltan de código CoffeeScript del \nmétodo que se encarga de reconocer el lenguaje generado\npor <tt>statement</tt> para la <a href=\"#grammar\">gramática\ndefinida anteriormente</a>:\n<pre>\n  statement = ->\n    result = null\n    if --------- and ---------.---- is \"ID\"\n      left =\n        type: \"ID\"\n        value: ---------.-----\n\n      match \"ID\"\n      match \"=\"\n      right = ----------()\n      result =\n        type: \"=\"\n        left: left\n        right: right\n    else if lookahead and lookahead.type is \"P\"\n      match \"P\"\n      right = ------------\n      result =\n        type: \"P\"\n        value: right\n    else if lookahead and lookahead.type is \"IF\"\n      match \"IF\"\n      left = -----------\n      match \"THEN\"\n      right = -----------\n      result =\n        type: \"IF\"\n        left: left\n        right: right\n    else # Error!\n      throw \"Syntax Error. Expected identifier but found \" + \n        (if lookahead then lookahead.value else \"end of input\") + \n        \" near '#{input.substr(lookahead.from)}'\"\n    result\n\n</pre>\n","answers":{"qfi5-1":{"answer_text":"lookahead","correct":true,"explanation":null,"type":"String"},"qfi5-2":{"answer_text":"lookahead","correct":true,"explanation":null,"type":"String"},"qfi5-3":{"answer_text":"type","correct":true,"explanation":null,"type":"String"},"qfi5-4":{"answer_text":"lookahead","correct":true,"explanation":null,"type":"String"},"qfi5-5":{"answer_text":"value","correct":true,"explanation":null,"type":"String"},"qfi5-6":{"answer_text":"expression","correct":true,"explanation":null,"type":"String"},"qfi5-7":{"answer_text":"expression()","correct":true,"explanation":null,"type":"String"},"qfi5-8":{"answer_text":"condition()","correct":true,"explanation":null,"type":"String"},"qfi5-9":{"answer_text":"statement()","correct":true,"explanation":null,"type":"String"}},"points":3,"order":true,"question_comment":""},"question-5":{"question_text":"Rellene las partes que faltan del código CoffeeScript\nque reconoce el sublenguaje generado por <i>condition</i>:\n  <pre>\n  condition = ->\n    left = -----------\n    type = ---------.-----\n    match \"COMPARISON\"\n    right = ----------()\n    result =\n      type: type\n      left: left\n      right: right\n    result\n  </pre>\n","answers":{"qfi6-1":{"answer_text":"expression()","correct":true,"explanation":null,"type":"String"},"qfi6-2":{"answer_text":"lookahead","correct":true,"explanation":null,"type":"String"},"qfi6-3":{"answer_text":"value","correct":true,"explanation":null,"type":"String"},"qfi6-4":{"answer_text":"expression","correct":true,"explanation":null,"type":"String"}},"points":1,"order":true,"question_comment":""},"question-6":{"question_text":"\nComplete este fragmento de <tt>slim</tt> que establece el favicon de \nla página HTML:\n<pre>\n    link rel=\"-------------\" type=\"image/jpg\" href=\"images/favicon.jpg\"\n</pre>","answers":{"qfi7-1":{"answer_text":"/(shortcut\\s+)?icon/i","correct":true,"explanation":null,"type":"Regexp"}},"points":1,"order":true,"question_comment":""},"question-7":{"question_text":"\n      Para que un repositorio con una aplicación escrita en Ruby-Sinatra\n      pueda desplegarse en Heroku con nombre <tt>chuchu</tt> el primer comando   que debemos escribir es:\n      <pre>\n      heroku ------ ------\n      </pre>\n      Este comando crea un remoto git cuyo nombre es ------\n      y cuya URL es \n      <pre>\n        git@heroku.com:-------.git\n      </pre>\n      La URL de publicación/despliegue será:\n      http://---------.---------.com/\n      <br/>\n      Una vez que todo esta listo, para publicar nuestra versión\n      en la rama <tt>master</tt> en heroku debemos ejecutar el comando:\n      <pre>\n      git ---- ------ master\n      <br/>\n      </pre>\n      Si la versión que queremos publicar en heroku no está en la rama\n      <tt>master</tt> sino que está en la rama <tt>tutu</tt> deberemos \n      modificar el comando anterior:\n      <pre>\n      git push ------ ----:------\n      </pre>\n      Para ver los logs deberemos emitir el comando:\n      <pre>\n       heroku ----\n      </pre>\n    ","answers":{"qfi8-1":{"answer_text":"create","correct":true,"explanation":null,"type":"String"},"qfi8-2":{"answer_text":"chuchu","correct":true,"explanation":null,"type":"String"},"qfi8-3":{"answer_text":"heroku","correct":true,"explanation":null,"type":"String"},"qfi8-4":{"answer_text":"chuchu","correct":true,"explanation":null,"type":"String"},"qfi8-5":{"answer_text":"chuchu","correct":true,"explanation":null,"type":"String"},"qfi8-6":{"answer_text":"herokuapp","correct":true,"explanation":null,"type":"String"},"qfi8-7":{"answer_text":"push","correct":true,"explanation":null,"type":"String"},"qfi8-8":{"answer_text":"heroku","correct":true,"explanation":null,"type":"String"},"qfi8-9":{"answer_text":"heroku","correct":true,"explanation":null,"type":"String"},"qfi8-10":{"answer_text":"tutu","correct":true,"explanation":null,"type":"String"},"qfi8-11":{"answer_text":"master","correct":true,"explanation":null,"type":"String"},"qfi8-12":{"answer_text":"logs","correct":true,"explanation":null,"type":"String"}},"points":6,"order":true,"question_comment":""},"question-8":{"question_text":"Con que subcomando del cliente <tt>heroku</tt> abro el navegador\n    en la URL del proyecto?<br/>\n    <pre>\n    heroku ----\n    </pre>","answers":{"qfi9-1":{"answer_text":"open","correct":true,"explanation":null,"type":"String"}},"points":1,"order":true,"question_comment":""},"question-9":{"question_text":"\n      Escriba la parte que falta para que \n      el programa PEGJS reconozca el lenguaje\n      $\\{ a^n b^n c^n\\ /\\ n \\ge{} 1\\}$\n      <pre>\n      S = -------- 'a'+ B !('a'/'b'/'c')\n      A = 'a' A? 'b'\n      B = 'b' B? 'c'\n      </pre>\n    ","answers":{"qfi10-1":{"answer_text":"/\\&\\s*\\(\\s*A\\s*'c'\\s*\\)/","correct":true,"explanation":null,"type":"Regexp"}},"points":1,"order":true,"question_comment":""},"question-10":{"question_text":"\nDado el PEGjs<a id=\"pegif\"></a>:\n<pre>\nS =   if C:C then S1:S else S2:S { return [ 'ifthenelse', C, S1, S2 ]; }\n    / if C:C then S:S            { return [ 'ifthen', C, S]; }\n    / O                          { return 'O'; }\n_ = ' '*\nC = _'c'_                        { return 'c'; }\nO = _'o'_                        { return 'o'; }\nelse = _'else'_                 \nif = _'if'_\nthen = _'then'_    \n</pre>\nConsidere esta entrada:\n<pre>\nif c then if c then o else o\n</pre>\n<!-- ['ifthen', 'c', ['ifthenelse', 'c', 'o', 'o']] -->\n¿Cuál de los dos árboles es construido para la misma?:\n","answers":{"qmc11-1":{"answer_text":"<tt>['ifthen', 'c', ['ifthenelse', 'c', 'o', 'o']]</tt>","correct":true,"explanation":null},"qmc11-2":{"answer_text":"<tt>['ifthenelse', 'c', ['ifthen', 'c', 'o'], 'o']]</tt>","correct":false,"explanation":""}},"points":1,"question_comment":""},"question-11":{"question_text":"Si en el <a href=\"#pegif\">peg anterior</a> cambiamos el orden de las dos primeras reglas de <tt>S</tt>:\n<pre>\n  S =   if C:C then S:S            { return [ 'ifthen', C, S]; }\n      / if C:C then S1:S else S2:S { return [ 'ifthenelse', C, S1, S2 ]; }\n</pre>\nPara la misma entrada:\n<pre>\nif c then if c then o else o\n</pre>\n<!-- ['ifthen', 'c', ['ifthenelse', 'c', 'o', 'o']] -->\n¿Cuál de las respuestas es correcta?\n    ","answers":{"qmc12-1":{"answer_text":"<tt>['ifthenelse', 'c', ['ifthen', 'c', 'o'], 'o']]</tt>","correct":false,"explanation":""},"qmc12-2":{"answer_text":"<tt>Syntax Error</tt>. La frase no es aceptada por el peg","correct":true,"explanation":null},"qmc12-3":{"answer_text":"<tt>['ifthen', 'c', ['ifthenelse', 'c', 'o', 'o']]</tt>","correct":false,"explanation":""}},"points":1,"question_comment":""},"question-12":{"question_text":"\nRellene las partes que faltan de este código para que funcione:\n<pre>\nvar PEG = require (\"pegjs\");\nvar grammar = \"s = ('a' / 'b')+\";\nvar parser = PEG.-----------(grammar);\nvar input = process.argv[---] || 'abba';\nconsole.log(parser.parse(input))\n</pre>\nCuando se ejecuta, este código produce:\n<pre>\n[~/srcPLgrado/pegjs/examples(master)]$ node abba.pegjs abb\n[ 'a', 'b', 'b' ]\n</pre>\n    ","answers":{"qfi13-1":{"answer_text":"buildparser","correct":true,"explanation":null,"type":"String"},"qfi13-2":{"answer_text":"2","correct":true,"explanation":null,"type":"String"}},"points":1,"order":true,"question_comment":""},"question-13":{"question_text":"\n<a id=\"anbncn\"></a>\nComplete las partes que faltan para que el PEGjs reconozca este\nclásico ejemplo de lenguaje que no es independiente del contexto\n               $\\{ a^nb^nc^n / n \\ge{} 1 \\}$\n<pre>\nS = ---(--- ---) 'a'+ B:--- !('c'/[---]) { return B; }\nA = 'a' A:A? 'b' { if (A) { return A+1; } else return 1; }\nB = 'b' B:B? 'c' { if (B) { return B+1; } else return 1; }\n</pre>\n    ","answers":{"qfi14-1":{"answer_text":"&","correct":true,"explanation":null,"type":"String"},"qfi14-2":{"answer_text":"a","correct":true,"explanation":null,"type":"String"},"qfi14-3":{"answer_text":"'c'","correct":true,"explanation":null,"type":"String"},"qfi14-4":{"answer_text":"b","correct":true,"explanation":null,"type":"String"},"qfi14-5":{"answer_text":"^c","correct":true,"explanation":null,"type":"String"}},"points":1,"order":true,"question_comment":""},"question-14":{"question_text":"\nrellene las partes que faltan del siguiente programa PEGjs que reconoce\nlos comentarios Pascal:\n<pre>\nP     =   prog:N+                          { return prog; }\nN     =   chars:$(!Begin ANY)+             { return chars;}\n        / C\nC     = Begin chars:--- End                { return chars.join(''); }\nT     =   C \n        / (!----- ---- char:ANY)           { return char;}\nBegin = '(*'\nEnd   = '*)'\nANY   =   'z'    /* any character */       { return 'z';  }\n        / char:----                        { return char; }    \n</pre>\n    ","answers":{"qfi15-1":{"answer_text":"t*","correct":true,"explanation":null,"type":"String"},"qfi15-2":{"answer_text":"begin","correct":true,"explanation":null,"type":"String"},"qfi15-3":{"answer_text":"!end","correct":true,"explanation":null,"type":"String"},"qfi15-4":{"answer_text":"[^z]","correct":true,"explanation":null,"type":"String"}},"points":1,"order":true,"question_comment":""},"question-15":{"question_text":"\nRellene las partes que faltan de esta clase que implementa \npersistencia para programas PL0 usando el ORM DataMapper:\n<pre>\nDataMapper.-----(:default,·\n                 ENV['DATABASE_URL'] || \"sqlite3://#{Dir.pwd}/database.db\" )\n\nclass PL0Program\n  include ----------::--------\n··\n  -------- :name, String, :key => true\n  -------- :source, String, :length => 1..1024\nend\n\n  DataMapper.--------\n  DataMapper.-------------\n</pre>\n","answers":{"qfi16-1":{"answer_text":"setup","correct":true,"explanation":null,"type":"String"},"qfi16-2":{"answer_text":"datamapper","correct":true,"explanation":null,"type":"String"},"qfi16-3":{"answer_text":"resource","correct":true,"explanation":null,"type":"String"},"qfi16-4":{"answer_text":"property","correct":true,"explanation":null,"type":"String"},"qfi16-5":{"answer_text":"property","correct":true,"explanation":null,"type":"String"},"qfi16-6":{"answer_text":"finalize","correct":true,"explanation":null,"type":"String"},"qfi16-7":{"answer_text":"auto_upgrade!","correct":true,"explanation":null,"type":"String"}},"points":4,"order":true,"question_comment":""},"question-16":{"question_text":"\nRellene las partes que faltan del siguiente fragmento de código\nde la ruta <tt>/save</tt>\nque guarda el programa solicitado:\n<pre>\npost '/save' do\n  name = params[:fname]\n  c  = PL0Program.-----(:name => name)\n  if c\n    c.source = params[\"input\"]\n    c.----\n  else\n    c = PL0Program.new\n    c.name = params[\"fname\"]\n    c.source = params[\"input\"]\n    c.----\n  end\n  -------- '/'\nend\n</pre>\n","answers":{"qfi17-1":{"answer_text":"first","correct":true,"explanation":null,"type":"String"},"qfi17-2":{"answer_text":"save","correct":true,"explanation":null,"type":"String"},"qfi17-3":{"answer_text":"save","correct":true,"explanation":null,"type":"String"},"qfi17-4":{"answer_text":"redirect","correct":true,"explanation":null,"type":"String"}},"points":1,"order":true,"question_comment":""},"question-17":{"question_text":"\nEn la práctica del PEGjs tratabamos las expresiones aritméticas \nmediante estas dos reglas:\n<pre>\nexp    = t:term   r:(ADD term)*   { return tree(t,r); }\nterm   = f:factor r:(MUL factor)* { return tree(f,r); }\nADD      = _ op:[+-] _ { return op; }\nMUL      = _ op:[*/] _ { return op; }\n</pre>\nComplete el código de <tt>tree</tt>:\n<pre>\n{\n  var tree = function(f, r) {\n    if (r.------ > 0) {\n      var last = r.---();\n      var result = {\n        type:  -------,\n        left: ----------,\n        right: -------\n      };\n    }\n    else {\n      var result = f;\n    }\n    return result;\n  }\n}\n</pre>\n","answers":{"qfi18-1":{"answer_text":"length","correct":true,"explanation":null,"type":"String"},"qfi18-2":{"answer_text":"pop","correct":true,"explanation":null,"type":"String"},"qfi18-3":{"answer_text":"/last\\s*\\[\\s*0\\s*\\]/","correct":true,"explanation":null,"type":"Regexp"},"qfi18-4":{"answer_text":"/tree\\s*\\(\\s*f\\s*,\\s*r\\s*\\)/","correct":true,"explanation":null,"type":"Regexp"},"qfi18-5":{"answer_text":"/last\\s*\\[\\s*1\\s*\\]/","correct":true,"explanation":null,"type":"Regexp"}},"points":4,"order":true,"question_comment":""}};

      function findCorrectAnswer(idQuestion, questionType) {
        correctIds = [];
        for (id in data[idQuestion]['answers']) {
          if(data[idQuestion]['answers'][id.toString()]['correct'] == true)
            if (questionType == 0)
              return id.toString();
            else {
              correctIds.push(id.toString());
            } 
        }
        return correctIds;
      }
      
      function checkSelectMultiple(x, checkedIds, correctIds) {
        results = [];
        
        $.each(checkedIds, function(index, value){
          if (correctIds.indexOf(value) == -1) {
            results.push(false);
            printResults(value, 0, data[x.toString()]['answers'][value]['explanation'], 0);
          }
          else {
            results.push(true);
            printResults(value, 1, data[x.toString()]['answers'][value]['explanation'], 0);
          }
        });
        
        nCorrects = 0;
        nIncorrects = 0;
        $.each(results, function(index, value){
          if (value == true)
            nCorrects += 1;
          else
            nIncorrects += 1;
        });
        
        calculateMark(data[x.toString()], x.toString(), null, 3, nCorrects, nIncorrects);
      }
      
      function printResults(id, type, explanation, typeQuestion) {
        if (typeQuestion == 0) {                                        // MultipleChoice and SelectMultiple
          $("br[class=" + id + "br" + "]").detach();
          if (type == 1) {
            if ((explanation == "") || (explanation == null))
              $("div[id ~= " + id + "r" + "]").html("<strong class=correct> " + i18n[language]['correct'] + "</strong></br>");
            else
              $("div[id ~= " + id + "r" + "]").html("<strong class=correct> " + i18n[language]['correct'] + " - " + explanation + "</strong></br>");
          }
          else {
            if ((explanation == "") || (explanation == null))
              $("div[id ~= " + id + "r" + "]").html("<strong class=incorrect> " + i18n[language]['incorrect'] + "</strong></br>");
            else
              $("div[id ~= " + id + "r" + "]").html("<strong class=incorrect> " + i18n[language]['incorrect'] + " - " + explanation + "</strong></br>");
          }
        }
        else {          // FillIn
          for (r in id) {
            input = $("#" + r.toString());
            if (id[r] == true) {
              input.attr('class', input.attr('class') + ' correct');
            }
            else { 
              if ((id[r] == false) || (id[r] != "n/a")) {
                input.attr('class', input.attr('class') + ' incorrect');
              }
            }
            
            if ((id[r] != true) && (id[r] != false) && (id[r] != "n/a")) {
              if (explanation[id[r].toString()] != null)
                $("div[id ~= " + r.toString() + "r" + "]").html(" <div class=explanation>" + explanation[id[r].toString()] + "</div>");
            }
            else {
              if (explanation[r] != null)
                $("div[id ~= " + r + "r" + "]").html(" <div class=explanation>" + explanation[r] + "</div>");
            }
          }
        }
      }
      
      function calculateMark(question, id, result, typeQuestion, numberCorrects, numberIncorrects) {
        if (typeQuestion == 2) {
          if (result)
            $("#" + id).append("<strong class=mark> " + question['points'].toFixed(2) + "/" + question['points'].toFixed(2) + " " + i18n[language]['points'] + "</strong></br></br>");
          else
            $("#" + id).append("<strong class=mark> 0.00/" + question['points'].toFixed(2) + " " + i18n[language]['points'] + "</strong></br></br>");
        }
        else if (typeQuestion == 1) {
          size = 0;
          for (y in question['answers'])
            if (question['answers'][y]['correct'] == true)
              size += 1;
              
          pointsUser = ((question['points'] / size) * numberCorrects).toFixed(2);
          $("#" + id).append("<strong class=mark> " + pointsUser + "/" + question['points'].toFixed(2) + " " + i18n[language]['points'] + "</strong></br></br>");
        }
        else {
          totalCorrects = 0;
          for (y in question['answers']) {
            if (question['answers'][y]['correct'] == true)
              totalCorrects += 1;
          }
          
          correctAnswerPoints = question['points'] / totalCorrects;
          penalty = correctAnswerPoints * numberIncorrects;
          mark = (correctAnswerPoints * numberCorrects) - penalty;
          
          if (mark < 0)
            mark = 0;
            
          $("#" + id).append("<strong class=mark> " + mark.toFixed(2) + "/" + question['points'].toFixed(2) + " " + i18n[language]['points'] + "</strong></br></br>");        
        }
      }
      
      function checkFillin(correctAnswers, userAnswers, distractorAnswers, typeCorrection) {
        correction = {};
        checkedAnswers = {};
        
        if (typeCorrection == 0) {          // Order doesn't matter
          for (u in userAnswers) {
            if (userAnswers[u] != undefined) {    // No empty field
              matchedCorrect = false;
              for (y in correctAnswers) {
                if (checkAnswers[u] == undefined) {
                  if ((typeof(correctAnswers[y]) == "string") || (typeof(correctAnswers[y]) == "number")) {    // Answer is a String or a Number
                    if (userAnswers[u] == correctAnswers[y]) {
                      correction[u] = true;
                      checkedAnswers[u] = userAnswers[u];
                      matchedCorrect = true;
                      break;
                    }
                  }
                  else {  // Answer is a Regexp
                    if (userAnswers[u].match(correctAnswers[y])) {
                      correction[u] = true;
                      checkedAnswers[u] = userAnswers[u];
                      matchedCorrect = true;
                      break;
                    }
                  }
                }
              }
              if (!matchedCorrect)
                correction[u] = false;
            }
            else
              correction[u] = "n/a";
          }
        }
        else {                            // Order matters
          for (u in userAnswers) {
            if (userAnswers[u] != undefined) {
              if ((typeof(correctAnswers[u]) == "string") || (typeof(correctAnswers[u]) == "number")) {
                if (userAnswers[u] == correctAnswers[u])
                  correction[u] = true;
                else
                  correction[u] = false;
              }
              else {
                if (userAnswers[u].match(correctAnswers[u]))
                  correction[u] = true;
                else
                  correction[u] = false;
              }
            }
            else
              correction[u] = "n/a";
          }
        }
        
        if (Object.keys(userAnswers).length == 1) {
          for (u in userAnswers) {
            if (correction[u] == false) {
              for (y in distractorAnswers) {
                if ((typeof(distractorAnswers[y]) == "string") || (typeof(distractorAnswers[y]) == "number")) {
                  if (userAnswers[u] == distractorAnswers[y])
                    correction[u] = y.toString();
                }
                else {
                  if (userAnswers[u].match(distractorAnswers[y]))
                    correction[u] = y.toString();
                }
              }
            }
          }
        }
        return correction;
      }
      
      function checkAnswers() {
        
        for (x in data) {
          if ($("#" + x.toString() + " strong").length == 0) {
            correct = false;
            answers = $("#" + x.toString() + " input");
            
            if (answers.attr('class').match("fillin")) {
              correctAnswers = {};
              distractorAnswers = {};
              explanation = {};
              stringAnswer = false;
              
              for (ans in data[x.toString()]['answers']) {
                if (data[x.toString()]['answers'][ans]['correct'] == true) {
                  if (data[x.toString()]['answers'][ans]['type'] == "Regexp") {
                    string = data[x.toString()]['answers'][ans]['answer_text'].split('/');
                    regexp = string[1];
                    options = string[2];
                    correctAnswers[ans.toString()] = RegExp(regexp, options);
                  }
                  else { // String or Number
                    correctAnswers[ans.toString()] = data[x.toString()]['answers'][ans]['answer_text'];
                    stringAnswer = true;
                  }
                }
                else {
                  if (data[x.toString()]['answers'][ans]['type'] == "Regexp") {
                    string = data[x.toString()]['answers'][ans]['answer_text'].split('/');
                    regexp = string[1];
                    options = string[2];
                    distractorAnswers[ans.toString()] = RegExp(regexp, options);
                  }
                  else {// String or Number
                    distractorAnswers[ans.toString()] = data[x.toString()]['answers'][ans]['answer_text'];
                    stringAnswer = true;
                  }
                }
                explanation[ans] = data[x.toString()]['answers'][ans]['explanation'];
              }
              
              userAnswers = {};
              for (i = 0; i < answers.length; i++) {
                if (answers[i].value == '')
                  userAnswers[answers[i].id.toString()] = undefined;
                else
                  if (stringAnswer)
                    userAnswers[answers[i].id.toString()] = answers[i].value.toLowerCase();
                  else
                    userAnswers[answers[i].id.toString()] = answers[i].value;
              }
              
              if (data[x.toString()]['order'] == false)
                results = checkFillin(correctAnswers, userAnswers, distractorAnswers, 0);
              else
                results = checkFillin(correctAnswers, userAnswers, distractorAnswers, 1);
                
              allEmpty = true;
              nCorrects = 0;
              
              for (r in results) {
                if (results[r] == true)
                  nCorrects += 1;
                if (results[r] != "n/a")
                  allEmpty = false;
              }
              
              if (!allEmpty) {
                printResults(results, null, explanation, 1);
                calculateMark(data[x.toString()], x.toString(), null, 1, nCorrects, null);
              }
            }
            
            else if (answers.attr('class') == "select") {
              idCorrectAnswer = findCorrectAnswer(x.toString(), 0);
              
              if ($("#" + x.toString() + " :checked").size() != 0) {
                if ($("#" + x.toString() + " :checked").attr('id') == idCorrectAnswer) {
                  printResults($("#" + x.toString() + " :checked").attr('id'), 1, "", 0);
                  correct = true;
                }
                else {
                  id = $("#" + x.toString() + " :checked").attr('id');
                  printResults(id, 0, data[x.toString()]['answers'][id]['explanation'], 0);
                }
                calculateMark(data[x.toString()], x.toString(), correct, 2, null, null);
              }
            }
            
            else {
              if ($("#" + x.toString() + " :checked").size() != 0) {
                answers = $("#" + x.toString() + " :checked");
                checkedIds = [];
                
                $.each(answers, function(index, value){
                  checkedIds.push(value['id']);
                });
                
                correctIds = [];
                correctIds = findCorrectAnswer(x.toString(), 1);
                checkSelectMultiple(x, checkedIds, correctIds);
              }
            }
          }
        }
      }
      
      function storeAnswers() {
        if(typeof(Storage) !== "undefined") {
          // Store
          //localStorage.lastname = "Smith";
          // Retrieve
          //document.getElementById("storage").innerHTML=localStorage.lastname;
          inputText = $('input:text').filter(function() { return $(this).val() != ""; });
          for (i = 0; i < inputText.length; i++) {
            idAnswer = inputText[i].id;
            localStorage[idAnswer] = inputText[i].value;
          }
          
          inputRadioCheckBox = $('input:checked');
          for (i = 0; i < inputRadioCheckBox.length; i++) {
            idAnswer = inputRadioCheckBox[i].id;
            nquestion = parseInt(idAnswer.split('-')[0].substr(3)) - 1;
            localStorage[idAnswer] = data["question-" + nquestion.toString()]['answers'][idAnswer]['answer_text'];
          }
        }
        else {
          alert("Sorry! No Web Storage supported.");
        }
      }
      
      function deleteAnswers() {
        localStorage.clear();
        alert("Local storage deleted");
      }
      
      if (typeof(codehelper_ip) == "undefined")
        language = "EN";
      else
        language = codehelper_ip.Country;
      
      $("#submit").click(function() {
        checkAnswers();
        filledAllQuiz = true;
        
        for (x in data) {
          if ($("#" + x.toString() + " strong").length == 0)
            filledAllQuiz = false; 
        }
        if (filledAllQuiz)
          $("#submit").detach();
        
        storeAnswers();
      });

      $("#reset").click(function() {
        window.location.reload();
      });
      
      $("#deletestorage").click(function() {
        deleteAnswers();
      });
      
      $(document).ready(function() {
        if (localStorage.length != 0) {
          for (x in localStorage) {
            if (x.match(/qfi/))                                         // FillIn question
              $("#" + x.toString()).val(localStorage[x.toString()]);
            else {                                                      // SelectMultiple/MultipleChoice question
              $("#" + x.toString()).attr('checked', 'checked');
            }
          }
        }
      });
</script>
    <!-- Any JavaScript included by the user -->
    
    <!-- Latest compiled and minified JavaScript for Bootstrap -->
    <script src="http://netdna.bootstrapcdn.com/bootstrap/3.1.1/js/bootstrap.min.js"></script>
    <!-- -->
  </body>
</html>

\end{rawhtml}

\section{05}

\subsection{Repaso para la prueba del 14/05/2014}
\begin{enumerate}

\item
¿Que retorna?
\begin{verbatim}
"hello small world and blue sky".match(/(\S+)\s+(\S+)/);
\end{verbatim}

\item
Indique que casa con el primer paréntesis y que con el segundo en las siguientes expresiones regulares:
\begin{verbatim}
> x = "I have 2 numbers: 53147"
> pats = [ /(.*)(\d*)/, 
           /(.*)(\d+)/, 
           /(.*?)(\d*)/, 
           /(.*?)(\d+)/, 
           /(.*)(\d+)$/, 
           /(.*?)(\d+)$/, 
           /(.*)\b(\d+)$/, 
           /(.*\D)(\d+)$/ ]
\end{verbatim}
Es decir, compute la salida de:
\begin{verbatim}
   pats.map( function(r) { return r.exec(x).slice(1); })
\end{verbatim}
\item
¿Que retorna el matching?:
\begin{verbatim}
>  a = "hola juan"
 => "hola juan" 
> a.match(/(?:hola )*(juan)/)
\end{verbatim}
\item ¿Que salidas se obtienen?
\begin{verbatim}
> "a\na".match(/a$/)
________________________________
> "a\na".match(/a$/m)
________________________________
> "a\na".match(/^a/gm)
____________
> "a\na".match(/^a/g)
_______
\end{verbatim}
\item
Escriba  la expresión regular que da lugar a este resultado (enumerar las líneas):
\begin{verbatim}
> x = "one\ntwo\nthree\nfour"
'one\ntwo\nthree\nfour'
> a = (c = 1, x.replace(_____, function(t) { return  c++ + ' ' + t; }))
'1 one\n2 two\n3 three\n4 four'
> console.log(a)
1 one
2 two
3 three
4 four
undefined
\end{verbatim}
\item
Supongamos dado el método
\begin{verbatim}
String.prototype.repeat = function( num ) {
    return new Array( num + 1 ).join( this );
}
\end{verbatim}
de manera que podamos escribir expresiones como:
\begin{verbatim}
> x = 'a'.repeat(40)
'aaaaaaaaaaaaaaaaaaaaaaaaaaaaaaaaaaaaaaaa'
\end{verbatim}
Encontremos una solución de la ecuación diofántica \verb|3x + 2y + 5z = 40|
\begin{verbatim}
> m = x.match(/^_______________________________$/).slice(1)
[ 'aaaaaaaaaaaaaaaaaaaaaaaaaaaaaaaaa',
  'aa',
  'aaaaa' ]
\end{verbatim}
Calculemos las longitudes de las tres cadenas:
\begin{verbatim}
> r = m.map(function(s) { return s.length; })
[ 33, 2, 5 ]
\end{verbatim}
Dividamos por los coeficientes para obtener la solución:
\begin{verbatim}
> coef = [3, 2, 5]
> i = 0; w = r.map(function(x) { return x/coef[i++]; }
[ 11, 1, 1 ]
\end{verbatim}
Encuentre la expresión regular usada.
\item 
Escriba una expresión regular que reconozca cadenas de dobles comillas como \verb|"hello world"|
y en las que las comillas puedan aparecer escapadas como en \verb|"Hello \"Jane\" and Jakes"|

\item
Escriba una expresión regular que reconozca los números en punto flotante como
\verb|2.34|, \verb|-5.2e-1| y \verb|0.9e3|

\item
\label{item:ccomments}
¿Que queda en \verb|m[0]|?
\begin{verbatim}
m = 'main() /* 1c */ { /* 2c */ return; /* 3c */ }'.match(new RegExp('/\\*.*\\*/'))
\end{verbatim}
¿Por qué?
\item 
¿Por qué debemos duplicar el carácter de escape \verb|\| en  la expresión regular \verb|new RegExp('/\\*.*\\*/')| de la pregunta anterior \ref{item:ccomments}?
\item
Se quiere poner un espacio en blanco después de la aparición de cada coma:
\begin{verbatim}
> 'ab,cd,4,3,   de,   fg'.replace(/,/, ', ')
=> "ab, cd, 4, 3,    de,    fg" 
\end{verbatim}
pero se quiere que la sustitución no tenga lugar si la coma esta incrustada entre
dos dígitos. Además se pide que si hay ya un espacio después de la coma,
no se duplique

Como función de reemplazo use:
\begin{verbatim}
f = function(match, p1, p2, offset, string) { return (p1 || p2 + " "); }
\end{verbatim}

\item
Escribe un patrón regular
que reconozca las cadenas  que representan números no primos en unario
de manera que el primer paréntesis case con el divisor mas grande del número.

\item
Escribe un patrón regular
que reconozca las cadenas  que representan números no primos en unario
de manera que el primer paréntesis case con el divisor mas pequeño del número.

\item Escriba una expresión regular que reconozca los comentarios del lenguaje JavaScript de la forma
\verb|// ...  |

\item Escriba una expresión regular que reconozca los comentarios del lenguaje JavaScript de la forma
\verb|/* ...  */|


\item Rellene lo que falta para que la salida sea la que aparece en la sesión de node:
\begin{verbatim}
> re = __________
> str = "John Smith"
'John Smith'
> newstr = str.replace(re, "______")
'Smith, John'
\end{verbatim}
\item  Rellene las partes que faltan:
\begin{verbatim}
> re = /d(b+)(d)/ig
/d(b+)(d)/gi
> z = "dBdxdbbdzdbd"
'dBdxdbbdzdbd'
> result = re.exec(z)
[ ______, _____, ______, index: __, input: 'dBdxdbbdzdbd' ]
> re.lastIndex
______
> result = re.exec(z)
[ ______, _____, ______, index: __, input: 'dBdxdbbdzdbd' ]
> re.lastIndex
______
> result = re.exec(z)
[ ______, _____, ______, index: __, input: 'dBdxdbbdzdbd' ]
> re.lastIndex
______
> result = re.exec(z)
_____
\end{verbatim}
\item Escriba la expresión regular \verb|r| para que produzca el resultado final:
\begin{verbatim}
> x = "hello"
> r = /l(___)/
> z = r.exec(x)
[ 'l', index: 3, input: 'hello' ]
\end{verbatim}
\item 
\begin{verbatim}
> z = "dBdDBBD"
> re = /d(b+)(d)/ig
> re.lastIndex = ________
> result = re.exec(z)
[ 'DBBD',
  'BB',
  'D',
  index: 3,
  input: 'dBdDBBD' ]
\end{verbatim}
\item  Conteste:
\begin{enumerate}
\item Explique que hace el siguiente fragmento de código:
\begin{verbatim}
> RegExp.prototype.bexec = function(str) {
...   var i = this.lastIndex;
...   var m = this.exec(str);
...   if (m && m.index == i) return m;
...   return null;
... }
[Function]
\end{verbatim}
\item Rellene las salidas que faltan:
\begin{verbatim}
> re = /d(b+)(d)/ig
/d(b+)(d)/gi
> z = "dBdXXXXDBBD"
'dBdXXXXDBBD'
> re.lastIndex = 3
> re.bexec(z)
_____________________________________________________
> re.lastIndex = 7
> re.bexec(z)
_____________________________________________________
\end{verbatim}
\end{enumerate}
\item 
Escriba una expresión JavaScript que permita reemplazar todas las apariciones de palabras repetidas en una String por una sóla aparición de la misma
\item 
Supongamos que se usa una función como segundo argumento de \verb|replace|.
¿Que argumentos recibe?
\item 
¿Cual es la salida?
\begin{verbatim}
> "bb".match(/b|bb/)

> "bb".match(/bb|b/)

\end{verbatim}

\item  El siguiente fragmento de código tiene por objetivo
escapar las entidades HTML para que no sean intérpretadas como código HTML.
Rellene las partes que faltan.
\begin{verbatim}
var entityMap = {
    "&": "&___;",
    "<": "&__;",
    ">": "&__;",
    '"': '&quot;',
    "'": '&#39;',
    "/": '&#x2F;'
  };

function escapeHtml(string) {
  return String(string).replace(/_________/g, function (s) {
    return ____________;
  });
\end{verbatim}
\item ¿Cual es la salida?
\begin{verbatim}
> a = [1,2,3]
[ 1, 2, 3 ]
> b = [1,2,3]
[ 1, 2, 3 ]
> a == b
________
\end{verbatim}
\item
¿Como se llama el método que permite obtener una representación como cadena de un objeto?
¿Que parámetros espera? ¿Como afectan dichos parámetros?
\item ¿Cual debe ser el valor del atributo \verb|rel| para usar la imagen como favicon?
\begin{verbatim}
<link rel="_____________" href="etsiiull.png" type="image/x-icon"> 
\end{verbatim}
\item
Escriba un código JavaScript que defina una clase \verb|Persona| con atributos \verb|nombre|
y \verb|apellidos| y que disponga de un método \verb|saluda|.
\item
Reescriba la solución al problema anterior haciendo uso del método \verb|template|
de  \verb|underscore| y ubicando el template dentro de un tag \verb1script1.
\item Rellene lo que falta:
\begin{verbatim}
[~/srcPLgrado/temperature/tests(master)]$ cat tests.js 
var assert = chai.______;

suite('temperature', function() {
    test('[1,{a:2}] == [1,2]', function() {
      assert._________([1, {a:2}], [1, {a:2}]);
    });
    test('5X = error', function() {
        original.value = "5X";
        calculate();
        assert._____(converted.innerHTML, /ERROR/);
    });
});
\end{verbatim}
% \item añadir sinatra app
\item
¿Cómo se llama el directorio por defecto desde el que una aplicación sinatra sirve los ficheros estáticos?
\item
Explique la línea:
\begin{verbatim}
set :public_folder, File.dirname(__FILE__) + '/starterkit'
\end{verbatim}
¿Que es \verb|__FILE__|? ¿Que es \verb|File.dirname(__FILE__)|?
¿Que hace el método \verb|set|? (Véase 
\htmladdnormallink{http://www.sinatrarb.com/configuration.html}{http://www.sinatrarb.com/configuration.html})
% takes a setting name and value and creates an attribute on the application object
\item Escriba un programa sinatra que cuando se visite la URI \verb|/chuchu| muestre
una página que diga \verb|"hello world!"|
\item
¿Cual es el signifcado de \verb|__END__| en un programa Ruby?
\item  
\label{sinatalayout}
Esta y las preguntas 
\ref{sinatraindex} y
\ref{sinatrachuchu} se refieren al mismo programa ruby sinatra.
Explique este fragmento de dicho programa ruby sinatra. 
\begin{verbatim}
@@layout
  <!DOCTYPE html>
  <html>
    <head>
        <meta charset="utf-8" />
        <title>Demo</title>
    </head>
    <body>
        <a href="http://jquery.com/">jQuery</a>
        <div class="result"></div>
        <script src="jquery.js"></script>
        <%= yield %>
    </body>
  </html>
\end{verbatim}
\begin{enumerate}
\item ¿En que lugar del fichero que contiene el programa está ubicada esta sección? 
\item ¿Cómo se llama el lenguaje en el que esta escrita esta sección?
\item ¿Para que sirve la sección \verb|layout|?
\item ¿Cual es la función del \verb|<div class="result"></div>|?
\item ¿Para que sirve el \verb|<%= yield %>|?
\end{enumerate}
\item 
\label{sinatraindex}
Explique este fragmento de un programa ruby sinatra.
\begin{verbatim}
@@index
  <script>
  $( document ).ready(function() {
      $( "a" ).click(function( event ) {
          event.preventDefault();
          $.get( "/chuchu", function( data ) {
            $( ".result" ).html( data );
            alert( "Load was performed." );
          });
      });
  });
  </script>
\end{verbatim}
\begin{enumerate}
\item ¿Cuando ocurre el evento \verb|ready|?
\item ¿Que hace \verb|event.preventDefault()|?
\item ¿Que hace \verb|$.get( "/chuchu", function( data ) { ... }|?
¿Cuando se dispara la callback?
\item ¿que hace la línea \verb|$( ".result" ).html( data )|?
\end{enumerate}
% \item añadir jquery ajax
\item Explique este fragmento de código ruby-sinatra:
\label{sinatrachuchu}
\begin{verbatim}
get '/chuchu' do
  if request.xhr? 
    "hello world!"
  else 
    erb :tutu
  end
end
\end{verbatim}
\item  En el siguiente programa - que calcula la conversión
de temperaturas entre grados Farenheit y Celsius - rellene las partes que faltan:
\begin{enumerate}
\item  index.html:
\begin{verbatim}
<html>
  <head>
      <meta http-equiv="Content-Type" content="text/html; charset=_____">
      <title>JavaScript Temperature Converter</title>
      <link ____="global.css" ___="stylesheet" ____="text/css">

     <script type="_______________" src="temperature.js"></script>
  </head>
  <____>
    <h1>Temperature Converter</h1>
    <table>
      <tr>
        <th>Enter  Temperature (examples: 32F, 45C, -2.5f):</th>
        <td><input id="________" ________="calculate();"></td>
      </tr>
      <tr>
        <th>Converted Temperature:</th>
        <td><span class="output" id="_________"></span></td>
      </tr>
    </table>
  </____>
</html>
\end{verbatim}

\item Rellene las partes del código JavaScript que faltan en \verb|temperature.js|:
\begin{verbatim}
"use strict"; // Use ECMAScript 5 strict mode in browsers that support it
function calculate() {
  var result;
  var original       = document.getElementById("________");
  var temp = original.value;
  var regexp = /_______________________________/;
  
  var m = temp.match(______);
  
  if (m) {
    var num = ____;  // paréntesis correspondiente
    var type = ____;
    num = parseFloat(num);
    if (type == 'c' || type == 'C') {
      result = (num * 9/5)+32;
      result = ______________________________ // 1 sólo decimal y el tipo
    }
    else {
      result = (num - 32)*5/9;
      result = ____________________________ // 1 sólo decimal y el tipo
    }
    converted._________ = result; // Insertar "result" en la página
  }
  else {
    converted._________ = "ERROR! Try something like '-4.2C' instead";
  }
}
\end{verbatim}
\end{enumerate}
\item  ¿Que hace \verb|autofocus|?
\begin{verbatim}
<td><textarea autofocus cols = "80" rows = "5" id="original"></textarea></td> 
\end{verbatim}
\item  ¿Que hacen las siguientes pseudo-clases estructurales CSS3?
\begin{verbatim}
tr:nth-child(odd)    { background-color:#eee; }
tr:nth-child(even)    { background-color:#00FF66; }
\end{verbatim}
\item ¿Que contiene el objeto \verb|window| en un programa JavaScript que se ejecuta en un navegador?

\item 
\begin{enumerate}
\item 
¿Que es \htmladdnormallink{Local Storage}{http://diveinto.html5doctor.com/storage.html}? ¿Que hace la siguiente línea?
\begin{verbatim}
  if (window.localStorage) localStorage.original  = temp;
\end{verbatim}
\item  ¿Cuando se ejecutará esta callback? ¿Que hace?
\begin{verbatim}
window.onload = function() {
  // If the browser supports localStorage and we have some stored data
  if (window.localStorage && localStorage.original) {
    document.getElementById("original").value = localStorage.original;
  }
};
\end{verbatim}
\end{enumerate}

\item  ¿Cómo se hace para que elementos de la página web permanezcan ocultos para 
posteriormente mostrarlos? ¿Que hay que hacer en el HTML, en la hoja de estilo y en el JavaScript?
\item Rellene los estilos para los elementos de las clases para que su visibilidad
case con la que su nombre indica:
\begin{verbatim}
.hidden      { display: ____; }
.unhidden    { display: _____; }
\end{verbatim}
\item 
Los siguientes textos corresponden  a los ficheros de 
la práctica 
de construcción de un analizador léxico de los ficheros de configuración INI. 
Rellena las partes que faltan.
\begin{enumerate}
\item  Rellena las partes que faltan en el contenido del fichero \verb|index.html|. 
Comenta que hace el tag \verb|<input>|.
Comenta que hace el tag \verb|<pre>|.
\begin{verbatim}
<html>
  <head>
     <meta http-equiv="Content-Type" content="text/html; charset=UTF-8">
     <title>INI files</title>
     <link href="global.css" rel="__________" type="text/css">

     <script type="_______________" src="underscore.js"></script>
     <script type="_______________" src="jquery.js"></script>
     <script type="_______________" src="______"></script>
  </head>
  <body>
    <h1>INI files</h1>
    <input type="file" id="_________" />
    <div id="out" class="hidden">
    <table>
      <tr><th>Original</th><th>Tokens</th></tr>
      <tr>
        <td>
          <pre class="input" id="____________"></pre>
        </td>
        <td>
          <pre class="output" id="___________"></pre>
        </td>
      </tr>
    </table>
    </div>
  </body>
</html>
\end{verbatim}

\item 
A continuación siguen los contenidos del fichero \verb|ini.js| conteniendo el JavaScript.
\begin{enumerate}
\item 
Rellena las partes que faltan. 
El siguiente ejemplo de fichero \verb|.ini| le puede ayudar
a recordar la parte de las expresiones regulares 
\begin{verbatim}
; last modified 1 April 2001 by John Doe
[owner]
name=John Doe
organization=Acme Widgets Inc.
\end{verbatim}
\item 
Explica 
el uso del template.
\item 
Explica el uso de JSON.stringify
\end{enumerate}
\begin{verbatim}
"use ______"; // Use ECMAScript 5 strict mode in browsers that support it

$(document)._____(function() {
   $("#fileinput").______(calculate);
});

function calculate(evt) {
  var f = evt.target.files[0]; 

  if (f) {
    var r = new __________();
    r.onload = function(e) { 
      var contents = e.target.______;
      
      var tokens = lexer(contents);
      var pretty = tokensToString(tokens);
      
      out.className = 'unhidden';
      initialinput._________ = contents;
      finaloutput._________ = pretty;
    }
    r.__________(f); // Leer como texto
  } else { 
    alert("Failed to load file");
  }
}

var temp = '<li> <span class = "<%= ______ %>"> <%= _ %> </span>\n';

function tokensToString(tokens) {
   var r = '';
   for(var i in tokens) {
     var t = tokens[i];
     var s = JSON.stringify(t, undefined, 2); //______________________________
     s = _.template(temp, {t: t, s: s});
     r += s;
   }
   return '<ol>\n'+r+'</ol>';
}

function lexer(input) {
  var blanks         = /^___/;
  var iniheader      = /^________________/;
  var comments       = /^________/;
  var nameEqualValue = /^________________________/;
  var any            = /^_______/;

  var out = [];
  var m = null;

  while (input != '') {
    if (m = blanks.____(input)) {
      input = input.substr(m.index+___________);
      out.push({ type : ________, match: _ });
    }
    else if (m = iniheader.exec(input)) {
      input = input.substr(___________________);
      _______________________________________ // avanzemos en input
    }
    else if (m = comments.exec(input)) {
      input = input.substr(___________________);
      _________________________________________
    }
    else if (m = nameEqualValue.exec(input)) {
      input = input.substr(___________________);
      _______________________________________________
    }
    else if (m = any.exec(input)) {
      _______________________________________
      input = '';
    }
    else {
      alert("Fatal Error!"+substr(input,0,20));
      input = '';
    }
  }
  return out;
}
\end{verbatim}
\end{enumerate}


\end{enumerate}

