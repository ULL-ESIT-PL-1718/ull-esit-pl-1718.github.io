\parrafo{jQuery}

\htmladdnormallink{jQuery}{http://jquery.com/} (\htmladdnormallink{Descarga la librería}{http://jquery.com/download/})

\cei{jQuery} is a cross-platform JavaScript library designed to simplify
the client-side scripting of HTML. 

\begin{itemize}
\item
It was released in January 2006
at BarCamp NYC by John Resig. 

\item
It is currently developed by a team of
developers led by Dave Methvin. 

\item
jQuery is the most popular JavaScript library in
use today

\item
jQuery's syntax is designed to make it easier 
\begin{itemize}
\item
to navigate a document,
\item
select DOM elements, 
\item
create animations, 
\item
handle events, and 
\item
develop Ajax applications. 
\end{itemize}

\item
The set of jQuery core features — DOM element selections, traversal
and manipulation — enabled by its selector engine (named "Sizzle"
from v1.3), created a new "programming style", fusing algorithms and
DOM-data-structures; and influenced the architecture of other JavaScript
frameworks like \wikip{YUI v3}{YUI\_Library} and \wikip{Dojo}{Dojo\_Toolkit}.
\item
\end{itemize}

\parrafo{How JQuery Works}

  \begin{itemize}
  \item
  Véase
  \htmladdnormallink{How jQuery Works}{http://learn.jquery.com/about-jquery/how-jquery-works/}
  \item
    \htmladdnormallink{https://github.com/crguezl/how-jquery-works-tutorial}{https://github.com/crguezl/how-jquery-works-tutorial} en GitHub
  \item
  \begin{verbatim}
  [~/javascript/jquery(master)]$ pwd -P
  /Users/casiano/local/src/javascript/jquery
  \end{verbatim}
  \item \htmladdnormallink{w3schools JQuery tutorial}{http://www.w3schools.com/jquery/default.asp}
  \end{itemize}

\begin{verbatim}
~/javascript/jquery(master)]$ cat index.html 
<!doctype html>
<html>
<head>
    <meta charset="utf-8" />
    <title>Demo</title>
</head>
<body>
    <a href="http://jquery.com/">jQuery</a>
    <script src="starterkit/jquery.js"></script>
    <script>
 
    // Your code goes here.
 
    </script>
</body>
</html>
\end{verbatim}

To ensure that their code runs after the browser finishes loading the document, many JavaScript programmers wrap their code in an onload function:

\begin{verbatim}
window.onload = function() { alert( "welcome" ); }
\end{verbatim}

Unfortunately, the code doesn't run until all images are finished downloading, including banner ads. To run code as soon as the document is ready to be manipulated, jQuery has a statement known as the ready event:

\begin{verbatim}
$( document ).ready(function() {
    // Your code here.
});
\end{verbatim}
For \verb|click| and most other events, you can prevent the default
behavior by calling 
\htmladdnormallink{event.preventDefault()}{http://api.jquery.com/event.preventdefault/}
in the event handler.
If this method is called, the default action of the event will not be triggered.
For example, clicked anchors will not take the browser to a new URL.

\begin{verbatim}
[~/javascript/jquery(master)]$ cat index2.html 
<!doctype html>
<html>
<head>
    <meta charset="utf-8" />
    <title>Demo</title>
</head>
<body>
    <a href="http://jquery.com/">jQuery</a>
    <script src="starterkit/jquery.js"></script>
    <script>
 
    $( document ).ready(function() {
        $( "a" ).click(function( event ) {
            alert( "The link will no longer take you to jquery.com" );
            event.preventDefault();
        });
    });
 
    </script>
</body>
</html>
\end{verbatim}
Borrowing from CSS 1–3, and then adding its own, jQuery offers a powerful set of tools for matching a set of elements in a document.

See jQuery \htmladdnormallink{Category: Selectors}{http://api.jquery.com/category/selectors/}.

Another common task is adding or removing a class.
jQuery also provides some handy effects.

\begin{verbatim}
[~/javascript/jquery(master)]$ cat index3.html 
<!doctype html>
<html>
<head>
    <meta charset="utf-8" />
    <style>
        a.test { font-weight: bold; }
    </style>
    <title>Demo</title>
</head>
<body>
    <a href="http://jquery.com/">jQuery</a>
    <script src="starterkit/jquery.js"></script>
    <script>
 
    $( document ).ready(function() {
        $( "a" ).click(function( event ) {
            $( "a" ).addClass( "test" );
            alert( "The link will no longer take you to jquery.com" );
            event.preventDefault();
            $( "a" ).removeClass( "test" );
            $( this ).hide( "slow" );
            $( this ).show( "slow" );
        });
    });
 
    </script>
</body>
</html>
\end{verbatim}
\begin{itemize}
\item
In JavaScript \verb|this| always refers to the {\it owner} of the function
we're executing, or rather, {\it to the object that a function is a method of}.

\item
When we define our function \verb|tutu()| in a page, its owner is the
page, or rather, the \verb|window| object (or \verb|global| object) of JavaScript. 

\item
An \verb|onclick| property, though, is owned by the HTML element it belongs to.

\item
The method
\htmladdnormallink{.addClass( className )}{http://api.jquery.com/addclass/}
adds the specified class(es) to each of the set of matched elements.

\verb|className| is a String containing 
one or more space-separated classes to be added to the class attribute of each matched element.

This method does not replace a class. It simply adds the class, appending
it to any which may already be assigned to the elements.
\item
The method \verb|.removeClass( [className ] )|
removes a single class, multiple classes, or all classes from each element in the set of matched elements.

If a class name is included as a parameter, then only that class will be
removed from the set of matched elements. If no class names are specified
in the parameter, all classes will be removed.

This method is often used with \verb|.addClass()| 
to switch elements' classes from one to another, like so:

\begin{verbatim}
$( "p" ).removeClass( "myClass noClass" ).addClass( "yourClass" );
\end{verbatim}

\end{itemize}


\parrafo{Ejemplo usando Ajax con jQuery y Express.js}

\htmladdnormallink{Código del server}{https://github.com/crguezl/how-jquery-works-tutorial/tree/getallparams}:

\begin{verbatim}
[~/javascript/jquery/how-jquery-works-tutorial(getallparams)]$ cat app.js
var express = require('express');
var app = express();
var path = require('path');

app.use(express.static('public'));

// view engine setup
app.set('views', path.join(__dirname, 'views'));
app.set('view engine', 'ejs');

app.get('/', function (req, res) {
  res.render('index', { title: 'Express' });
})

app.get('/chuchu', function (req, res) {
  var isAjaxRequest = req.xhr;
  console.log(isAjaxRequest);
  if (isAjaxRequest) {
    console.log(req.query);
    res.send('{"answer": "Server responds: hello world!"}')
  }
  else {
    res.send('not an ajax request');
  }
});

var server = app.listen(3000, function () {

  var host = server.address().address
  var port = server.address().port

  console.log('Example app listening at http://%s:%s', host, port)

});
\end{verbatim}

\begin{itemize}
\item \verb|jQuery.get( url [, data ] [, success(data, textStatus, jqXHR) ] [, dataType ] )|
load data from the server using a HTTP GET request.

\item \verb|url|

Type: String

A string containing the URL to which the request is sent.
\item \verb|data|

Type: PlainObject or String

A plain object or string that is sent to the server with the request.
\item \verb|success(data, textStatus, jqXHR)|

Type: Function()

A callback function that is executed if the request succeeds.
\item \verb|dataType|

Type: String

The type of data expected from the server. Default: Intelligent Guess 
(\verb|xml|, \verb|json|, \verb|script|, \verb|or| \verb|html|).
\end{itemize}

To use callbacks, it is important to know how to pass them into their parent function.

En el directorio \verb|views| hemos puesto el template:
\begin{verbatim}
[~/javascript/jquery/how-jquery-works-tutorial(getallparams)]$ cat views/index.ejs 
<!doctype html>
<html>
  <head>
    <title><%- title %></title>
  </head>
  <body>
    <h1><%- title  %></h1>
    <ul>
      <li><a href="http://jquery.com/" id="jq">jQuery</a>
      <li><a href="/chuchu">Visit chuchu!</a>
    </ul>
    <div class="result"></div>
    <script src="https://code.jquery.com/jquery-2.1.3.js"></script>
    <script>
      $( document ).ready(function() {
          $( "#jq" ).click(function( event ) {
              event.preventDefault();
              $.get( "/chuchu", {nombres: ["juan", "pedro"]}, function( data ) {
                $( ".result" ).html( data["answer"]);
                console.log(data);
              }, 'json');
          });
      });
    </script>
  </body>
</html>
\end{verbatim}
\begin{itemize}
\item
\verb|req.query|

An object containing a property for each query string parameter in the route. If there is no query string, it is the empty object, \verb|{}|.

\begin{verbatim}
// GET /search?q=tobi+ferret
req.query.q
// => "tobi ferret"

// GET /shoes?order=desc&shoe[color]=blue&shoe[type]=converse
req.query.order
// => "desc"

req.query.shoe.color
// => "blue"

req.query.shoe.type
// => "converse"
\end{verbatim}
\end{itemize}

Estas son las dependencias:
\begin{verbatim}
[~/javascript/jquery/how-jquery-works-tutorial(getallparams)]$ cat package.json 
{
  "name": "ajaxjquery",
  "version": "0.0.0",
  "description": "",
  "main": "hello.js",
  "dependencies": {
    "express": "*",
    "ejs": "*",
    "gulp-shell": "*",
    "body-parser": "~1.12.0"
  },
  "devDependencies": {},
  "scripts": {
    "test": "node hello.js"
  },
  "author": "",
  "license": "ISC"
}
\end{verbatim}
Además hemos instalado a nivel global \verb|gulp| y \verb|node-supervisor|.

Podemos arrancar el servidor usando este \verb|gulpfile|:

\begin{verbatim}
[~/javascript/jquery/how-jquery-works-tutorial(getallparams)]$ cat gulpfile.js 
var gulp    = require('gulp');
var shell = require('gulp-shell');

gulp.task('default', ['server']);

// npm install supervisor -g
gulp.task('server', function () {
  return gulp.src('').pipe(shell([ 'node-supervisor app.js' ]));
});

gulp.task('open', function() {
  return gulp.src('').
           pipe(shell("open https://github.com/crguezl/how-jquery-works-tutorial/tree/getallparams"));
});
\end{verbatim}

\parrafo{Ejemplo de como Desplegar una Aplicación  Express sobre Node.JS en Heroku}

Véase:
\begin{itemize}
\item
La rama heroku del repo \htmladdnormallink{how-jquery-works-tutorial}{https://github.com/crguezl/how-jquery-works-tutorial/tree/heroku}
\item
El tutorial de Heroku 
\htmladdnormallink{Getting Started with Node.js on Heroku}{https://devcenter.heroku.com/articles/getting-started-with-nodejs}
\item
El capítulo sobre Heroku en los apuntes de LPP
\end{itemize}



\parrafo{Ajax, jQuery y Sinatra}

JavaScript enables you to freely
pass functions around to be executed at a later time. A \cei{callback} is a
function that is passed as an argument to another function and is usually 
executed
after its parent function has completed. 

Callbacks are special because
they wait to execute until their parent finishes or some event occurs. 

Meanwhile, the
browser can be executing other functions or doing all sorts of other work.
\begin{verbatim}
[~/javascript/jquery(master)]$ cat app.rb
require 'sinatra'

set :public_folder, File.dirname(__FILE__) + '/starterkit'

get '/' do
  erb :index
end

get '/chuchu' do
  if request.xhr?
    "hello world!"
  else 
    erb :tutu
  end
end

__END__

@@layout
  <!DOCTYPE html>
  <html>
    <head>
        <meta charset="utf-8" />
        <title>Demo</title>
    </head>
    <body>
        <a href="http://jquery.com/">jQuery</a>
        <div class="result"></div>
        <script src="jquery.js"></script>
        <%= yield %>
    </body>
  </html>

@@index
  <script>
  $( document ).ready(function() {
      $( "a" ).click(function( event ) {
          event.preventDefault();
          $.get( "/chuchu", function( data ) {
            $( ".result" ).html( data );
            alert( "Load was performed." );
          });
      });
  });
  </script>

@@tutu
  <h1>Not an Ajax Request!</h1>
\end{verbatim}

\begin{itemize}
\item \verb|jQuery.get( url [, data ] [, success(data, textStatus, jqXHR) ] [, dataType ] )|
load data from the server using a HTTP GET request.

\item \verb|url|

Type: String

A string containing the URL to which the request is sent.
\item \verb|data|

Type: PlainObject or String

A plain object or string that is sent to the server with the request.
\item \verb|success(data, textStatus, jqXHR)|

Type: Function()

A callback function that is executed if the request succeeds.
\item \verb|dataType|

Type: String

The type of data expected from the server. Default: Intelligent Guess 
(\verb|xml|, \verb|json|, \verb|script|, \verb|or| \verb|html|).
\end{itemize}

To use callbacks, it is important to know how to pass them into their parent function.


Executing callbacks with arguments can be tricky.

This code example will not work:

\begin{verbatim}
$.get( "myhtmlpage.html", myCallBack( param1, param2 ) );
\end{verbatim}
The reason this fails is that the code executes 

\begin{verbatim}
myCallBack( param1, param2) 
\end{verbatim}

immediately and then passes \verb|myCallBack()|'s return value as the second
parameter to \verb|$.get()|. 

We actually want to pass the function \verb|myCallBack|,
not \verb|myCallBack( param1, param2 )|'s return value (which might or might not
be a function). 

So, how to pass in \verb|myCallBack()| and include arguments?

To defer executing \verb|myCallBack()| with its parameters, you can use
an anonymous function as a wrapper.


\begin{verbatim}
[~/javascript/jquery(master)]$ cat app2.rb
require 'sinatra'

set :public_folder, File.dirname(__FILE__) + '/starterkit'

get '/' do
  erb :index
end

get '/chuchu' do
  if request.xhr? # is an ajax request
    "hello world!"
  else 
    erb :tutu
  end
end

__END__

@@layout
  <!DOCTYPE html>
  <html>
    <head>
        <meta charset="utf-8" />
        <title>Demo</title>
    </head>
    <body>
        <a href="http://jquery.com/">jQuery</a>
        <div class="result"></div>
        <script src="jquery.js"></script>
        <%= yield %>
    </body>
  </html>

@@tutu
  <h1>Not an Ajax Request!</h1>

@@index
  <script>
    var param = "chuchu param";
    var handler = function( data, textStatus, jqXHR, param ) {
      $( ".result" ).html( data );
      alert( "Load was performed.\n"+
             "$data = "+data+
             "\ntextStatus = "+textStatus+
             "\njqXHR = "+JSON.stringify(jqXHR)+
             "\nparam = "+param );
    };
    $( document ).ready(function() {
        $( "a" ).click(function( event ) {
            event.preventDefault();
            $.get( "/chuchu", function(data, textStatus, jqXHR ) {
              handler( data, textStatus, jqXHR, param);
            });
        });
    });
  </script>
\end{verbatim}
El ejemplo en \verb|app2.rb| puede verse desplegado en Heroku:
\htmladdnormallink{http://jquery-tutorial.herokuapp.com/}{http://jquery-tutorial.herokuapp.com/}





