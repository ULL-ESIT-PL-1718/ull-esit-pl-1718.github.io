\section{Definiendo Funciones}

\section{Invocando Funciones}

\section{Argumentos y Parámetros}

\section{Funciones como Valores}

\section{Funciones como Espacios de Nombres}

\section{Clausuras}

\section{Propiedades, Métodos y Constructor}

\subsection{La propiedad {\tt length}}

\subsection{La Propiedad {\tt property}}

\subsection{Los Métodos {\tt call} y {\tt apply}}
\label{subsection:callyapply}
Los métodos \tei{call} y \tei{apply} nos permiten
invocar una función como si fuera un método de algún otro 
objeto.

\begin{verbatim}
[~/Dropbox/src/javascript/learning]$ cat call.js 
var Bob = {
  name: "Bob",
  greet: function() {
    console.log("Hi, I'm " + this.name);
  }
}
 
var Alice = {
  name: "Alice",
};

Bob.greet.call(Alice);
\end{verbatim}

\begin{verbatim}
[~/Dropbox/src/javascript/learning]$ node call.js
Hi, I'm Alice
\end{verbatim}

\begin{enumerate}
\item 
\htmladdnormallink{Function.apply and Function.call in JavaScript}{http://odetocode.com/blogs/scott/archive/2007/07/05/function-apply-and-function-call-in-javascript.aspx}
\begin{verbatim}
> function f() { console.log(this.x); }
undefined
> f.toString()
'function f() { console.log(this.x); }'
> z = { x : 99 }
{ x: 99 }
> f.call(z)
99
undefined
> 
\end{verbatim}
\item 
\begin{verbatim}
> o  = { x : 15 }
{ x: 15 }
> function f(m) { console.log(m+" "+this.x); }
undefined
> f("invoking f")
invoking f 10
undefined
> f.call(o, "invoking f via call");
invoking f via call 15
undefined
\end{verbatim}
\end{enumerate}


\section{Programación Funcional}

