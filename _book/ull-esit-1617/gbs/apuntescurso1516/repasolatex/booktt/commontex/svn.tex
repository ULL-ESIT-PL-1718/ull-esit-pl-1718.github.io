En esta sección indicamos los pasos para utilizar subversión bajo el protocolo
SSH. Se asume que ha configurado sub conexión SSH con el servidor de subversion
para que admita autentificación automática.
Véase 
\htmladdnormallink{
El capítulo sobre la SSH en los apuntes de Programación en Paralelo II
(http://nereida.deioc.ull.es/~pp2/perlexamples/chapter\_ssh.html)
}
{
http://nereida.deioc.ull.es/~pp2/perlexamples/chapter_ssh.html
} para aprender a establecer autentificación automática vía SSH.

\parrafo{Use Subversion: Creación de un Repositorio}
Parece que en \verb|banot| esta instalado subversion. 
Para crear un repositorio emita el comando \verb|svnadmin create|:

\begin{verbatim}
-bash-3.1$ uname -a
Linux banot.etsii.ull.es 2.6.24.2 #3 SMP Fri Feb 15 10:39:28 WET 2008 i686 i686 i386 GNU/Linux
-bash-3.1$ svnadmin create /home/loginname/repository/
-bash-3.1$ ls -l repository/
total 28
drwxr-xr-x 2 loginname apache 4096 feb 28 11:58 conf
drwxr-xr-x 2 loginname apache 4096 feb 28 11:58 dav
drwxr-sr-x 5 loginname apache 4096 feb 28 12:09 db
-r--r--r-- 1 loginname apache    2 feb 28 11:58 format
drwxr-xr-x 2 loginname apache 4096 feb 28 11:58 hooks
drwxr-xr-x 2 loginname apache 4096 feb 28 11:58 locks
-rw-r--r-- 1 loginname apache  229 feb 28 11:58 README.txt
\end{verbatim}

Una alternativa a considerar es ubicar el repositorio 
en un dispositivo de almacenamiento portable (pendriver)

\parrafo{Añadiendo Proyectos}

Ahora esta en condiciones de añadir proyectos 
al repositorio creado usando \verb|svn import|:
\begin{verbatim}
[loginname@tonga]~/src/perl/> uname -a
Linux tonga 2.6.24.2 #1 SMP Thu Feb 14 15:37:31 WET 2008 i686 i686 i386 GNU/Linux
[loginname@tonga]~/src/perl/> pwd
/home/loginname/src/perl
[loginname@tonga]~/src/perl/> ls -ld /home/loginname/src/perl/Grammar-0.02
drwxr-xr-x 5 loginname Profesor 4096 feb 28  2008 /home/loginname/src/perl/Grammar-0.02
[loginname@tonga]~/src/perl/> svn import -m 'Grammar Extended Module' \
                                        Grammar-0.02/ \
                                        svn+ssh://banot/home/loginname/repository/Grammar
Añadiendo      Grammar-0.02/t
Añadiendo      Grammar-0.02/t/Grammar.t
Añadiendo      Grammar-0.02/lib
Añadiendo      Grammar-0.02/lib/Grammar.pm
Añadiendo      Grammar-0.02/MANIFEST
Añadiendo      Grammar-0.02/META.yml
Añadiendo      Grammar-0.02/Makefile.PL
Añadiendo      Grammar-0.02/scripts
Añadiendo      Grammar-0.02/scripts/grammar.pl
Añadiendo      Grammar-0.02/scripts/Precedencia.yp
Añadiendo      Grammar-0.02/scripts/Calc.yp
Añadiendo      Grammar-0.02/scripts/aSb.yp
Añadiendo      Grammar-0.02/scripts/g1.yp
Añadiendo      Grammar-0.02/Changes
Añadiendo      Grammar-0.02/README

Commit de la revisión 2.
\end{verbatim}

En general, los pasos para crear un nuevo proyecto son:

\begin{verbatim}
* mkdir /tmp/nombreProyecto
* mkdir /tmp/nombreProyecto/branches
* mkdir /tmp/nombreProyecto/tags
* mkdir /tmp/nombreProyecto/trunk
* svn mkdir file:///var/svn/nombreRepositorio/nombreProyecto -m 'Crear el proyecto nombreProyecto'
* svn import /tmp/nombreProyecto \
            file:///var/svn/nombreRepositorio/nombreProyecto \
            -m "Primera versión del proyecto nombreProyecto"
\end{verbatim}


\parrafo{Obtener una Copia de Trabajo}

La copia en \verb|Grammar-0.02| ha sido usada para la creación del proyecto, 
pero no pertenece aún al proyecto. Es necesario descargar la copia del
proyecto que existe en el repositorio. Para ello usamos
\verb| svn checkout|:
\begin{verbatim}
[loginname@tonga]~/src/perl/> rm -fR Grammar-0.02
[loginname@tonga]~/src/perl/> svn checkout svn+ssh://banot/home/loginname/repository/Grammar Grammar
A    Grammar/t
A    Grammar/t/Grammar.t
A    Grammar/MANIFEST
A    Grammar/META.yml
A    Grammar/lib
A    Grammar/lib/Grammar.pm
A    Grammar/Makefile.PL
A    Grammar/scripts
A    Grammar/scripts/grammar.pl
A    Grammar/scripts/Calc.yp
A    Grammar/scripts/Precedencia.yp
A    Grammar/scripts/aSb.yp
A    Grammar/scripts/g1.yp
A    Grammar/Changes
A    Grammar/README
Revisión obtenida: 2
\end{verbatim}
Ahora disponemos de una copia de trabajo del proyecto en
nuestra máquina local:
\begin{verbatim}
[loginname@tonga]~/src/perl/> tree Grammar
Grammar
|-- Changes
|-- MANIFEST
|-- META.yml
|-- Makefile.PL
|-- README
|-- lib
|   `-- Grammar.pm
|-- scripts
|   |-- Calc.yp
|   |-- Precedencia.yp
|   |-- aSb.yp
|   |-- g1.yp
|   `-- grammar.pl
`-- t
    `-- Grammar.t

3 directories, 12 files
[loginname@tonga]~/src/perl/>       
[loginname@tonga]~/src/perl/> cd Grammar
[loginname@tonga]~/src/perl/Grammar/> ls -la
total 44
drwxr-xr-x 6 loginname Profesor 4096 feb 28  2008 .
drwxr-xr-x 5 loginname Profesor 4096 feb 28  2008 ..
-rw-r--r-- 1 loginname Profesor  150 feb 28  2008 Changes
drwxr-xr-x 3 loginname Profesor 4096 feb 28  2008 lib
-rw-r--r-- 1 loginname Profesor  614 feb 28  2008 Makefile.PL
-rw-r--r-- 1 loginname Profesor  229 feb 28  2008 MANIFEST
-rw-r--r-- 1 loginname Profesor  335 feb 28  2008 META.yml
-rw-r--r-- 1 loginname Profesor 1196 feb 28  2008 README
drwxr-xr-x 3 loginname Profesor 4096 feb 28  2008 scripts
drwxr-xr-x 6 loginname Profesor 4096 feb 28  2008 .svn
drwxr-xr-x 3 loginname Profesor 4096 feb 28  2008 t
\end{verbatim}
Observe la presencia de los subdirectorios de control \verb|.svn|.

\parrafo{Actualización del Proyecto}

Ahora podemos modificar el proyecto y hacer públicos los cambios 
mediante \verb|svn commit|:
\begin{verbatim}
loginname@tonga]~/src/perl/Grammar/> svn rm META.yml
D         META.yml
[loginname@tonga]~/src/perl/Grammar/> ls -la
total 40
drwxr-xr-x 6 loginname Profesor 4096 feb 28  2008 .
drwxr-xr-x 5 loginname Profesor 4096 feb 28 12:34 ..
-rw-r--r-- 1 loginname Profesor  150 feb 28 12:34 Changes
drwxr-xr-x 3 loginname Profesor 4096 feb 28 12:34 lib
-rw-r--r-- 1 loginname Profesor  614 feb 28 12:34 Makefile.PL
-rw-r--r-- 1 loginname Profesor  229 feb 28 12:34 MANIFEST
-rw-r--r-- 1 loginname Profesor 1196 feb 28 12:34 README
drwxr-xr-x 3 loginname Profesor 4096 feb 28 12:34 scripts
drwxr-xr-x 6 loginname Profesor 4096 feb 28  2008 .svn
drwxr-xr-x 3 loginname Profesor 4096 feb 28 12:34 t
[loginname@tonga]~/src/perl/Grammar/> echo "Modifico README" >> README
[loginname@tonga]~/src/perl/Grammar/> svn commit -m 'Just testing ...'
Eliminando     META.yml
Enviando       README
Transmitiendo contenido de archivos .
Commit de la revisión 3.
\end{verbatim}
Observe que ya no es necesario especificar el lugar en el que se encuentra
el repositorio: esa información esta guardada en los subdirectorios
de administración de subversion \verb|.svn|

El servicio de subversion parece funcionar desde fuera de la red del centro.
Véase la conexión desde una maquina exterior:

\begin{verbatim}
pp2@nereida:/tmp$  svn checkout svn+ssh://loginname@banot.etsii.ull.es/home/loginname/repository/Grammar Grammar
loginname@banot.etsii.ull.es's password:
loginname@banot.etsii.ull.es's password:
A    Grammar/t
A    Grammar/t/Grammar.t
A    Grammar/MANIFEST
A    Grammar/lib
A    Grammar/lib/Grammar.pm
A    Grammar/Makefile.PL
A    Grammar/scripts
A    Grammar/scripts/grammar.pl
A    Grammar/scripts/Calc.yp
A    Grammar/scripts/Precedencia.yp
A    Grammar/scripts/aSb.yp
A    Grammar/scripts/g1.yp
A    Grammar/Changes
A    Grammar/README
Revisión obtenida: 3
\end{verbatim}

\parrafo{Comandos Básicos}

\begin{itemize}
\item
    Añadir y eliminar directorios o ficheros individuales al proyecto
\begin{verbatim}
      svn add directorio_o_fichero
      svn remove directorio_o_fichero
\end{verbatim}
\item Guardar los cambios
\begin{verbatim}
      svn commit -m "Nueva version"
\end{verbatim}
\item Actualizar el proyecto
\begin{verbatim}
      svn update
\end{verbatim}
\item Ver el estado de los ficheros
\begin{verbatim}
      svn status -vu
\end{verbatim}
\item Crear un tag

\begin{verbatim}
      svn copy svn+ssh://banot/home/logname/repository/PL-Tutu/trunk \
               svn+ssh://banot/home/casiano/repository/PL-Tutu/tags/practica-entregada \
	       -m 'Distribución como fué entregada en la UDV'
\end{verbatim}
\end{itemize}

\parrafo{Referencias}

Consulte
\htmladdnormallink{
http://svnbook.red-bean.com/
}
{
http://svnbook.red-bean.com/
}.
Vea la página de la ETSII
\htmladdnormallink{
http://www.etsii.ull.es/svn
}
{
http://www.etsii.ull.es/svn
}.

En KDE puede instalar el cliente gráfico KDEsvn.

\parrafo{Autentificación Automática}

Para evitar la solicitud de claves cada vez que se comunica 
con el repositorio establezca autentificación SSH automática.
Para ver como hacerlo puede consultar las instrucciones en:

\htmladdnormallink{
http://search.cpan.org/~casiano/GRID-Machine/lib/GRID/Machine.pod\#INSTALLATION
}
{
http://search.cpan.org/~casiano/GRID-Machine/lib/GRID/Machine.pod\#INSTALLATION
}

Consulte también las páginas del manual Unix de
\verb|ssh|, \verb|ssh-key-gen|, \verb|ssh_config|, \verb|scp|, \verb|ssh-agent|, \verb|ssh-add|, \verb|sshd|


