
Cuando entra a una de las máquinas de los laboratorios de la ETSII
encontrará montado el directorio \verb|/soft/perl5lib/| y en 
él algunas distribuciones de módulos Perl que he instalado.

\begin{verbatim}
export MANPATH=$MANPATH:/soft/perl5lib/man/
export PERL5LIB=/soft/perl5lib/lib/perl5:/soft/perl5lib/lib/perl/5.10.0/:/soft/perl5lib/share/perl/5.10.0:/soft/perl5lib/share/perl/5.10/
export PATH=.:/home/casiano/bin:$PATH:/soft/perl5lib/bin
\end{verbatim}
%\begin{itemize}
%\item Perl 5.10
%
%\begin{verbatim}
%casiano@millo:~$ cat bin/setperl5_10lib
%# October 2008
%# source ~/bin/setperl5_10lib
%export PERL5LIB=/soft/perl5lib/perl5_10_1/lib/:/soft/perl5lib/perl5_10_1/lib/perl5:/soft/perl5lib/perl5_10_1/share/perl/5.8.8/
%export MANPATH=$MANPATH:/soft/perl5lib/perl5_10_1/man/
%# /soft/perl5lib/perl5_10_1/bin/ must be after ...
%export PATH=$PATH:/soft/perl5lib/perl5_10_1/bin/
%
%\end{verbatim}
%\item
%Damian Conway ha liberado una nueva versión de Regexp::Grammars con número 1.002.
%
%Puede verla en CPAN:
%
%\begin{verbatim}
%http://search.cpan.org/perldoc?Regexp::Grammars
%\end{verbatim}
%
%La he instalado en la ETSII.
%Si quieren usar esta antes de la antigua asegurense que el directorio
%
%\begin{verbatim}
%/soft/perl5lib/perl5_10_1/lib/site_perl/5.10.1/
%\end{verbatim}
%
%figura en la variable PERL5LIB antes que el directorio
%
%\begin{verbatim}
% /soft/perl5lib/perl5_10_1/lib/perl5/
%\end{verbatim}
%
%en el cual se encuentra la vieja versión.
%
%He aqui un ejemplo con el depurador que ilustra la influencia del orden:
%
%\begin{verbatim}
%casiano@millo:~/Lregexp-grammar-examples/calculator$ perl5.10.1 -wde 0
%
%Loading DB routines from perl5db.pl version 1.32
%Editor support available.
%
%Enter h or `h h' for help, or `man perldebug' for more help.
%
%main:   0
%
%  DB unshift @INC, '/soft/perl5lib/perl5_10_1/lib/perl5/'
%
%  DB use Regexp::Grammars
%
%  DB x $Regexp::Grammars::VERSION
%0  1.001005001
% 
%  DB q
%\end{verbatim}
%
%Ahora añadimos en primer lugar el camino a la versión nueva:
%
%\begin{verbatim}
%casiano@millo:~/Lregexp-grammar-examples/calculator$ perl5.10.1 -wde 0
%
%Loading DB routines from perl5db.pl version 1.32
%Editor support available.
%
%Enter h or `h h' for help, or `man perldebug' for more help.
%
%main:   0
%  DB unshift @INC, '/soft/perl5lib/perl5_10_1/lib/site_perl/5.10.1/'
%
%  DB use Regexp::Grammars
%
%  DB x $Regexp::Grammars::VERSION
%0  1.002
%\end{verbatim}
%
%
%\item Perl nativo (5.8.* a 5.9.*, Perl que viene con la distribución de Linux)
%
%Para usar los módulos instalados por el profesor con \verb|perl5.8.*| 
%establezca las siguientes variables de entorno para poder 
%acceder a las mismas:
%
%\begin{verbatim}
%[casiano@tonga]~/src/perl/tests/gtk2/> cat ~/bin/setperl5lib
%# October 2008
%# source ~/bin/setperl5lib
%export PERL5LIB=/soft/perl5lib/share/perl/5.8.8/:/soft/perl5lib/lib/perl/5.8.8:/soft/perl5lib/lib/perl/5.8:/soft/perl5lib/share/perl/5.8/
%export MANPATH=$MANPATH:/soft/perl5lib/man/
%export PATH=$PATH:/soft/perl5lib/bin
%\end{verbatim}
%En una \verb|bash| ejecute:
%\begin{verbatim}
%$ source ~/bin/setperl5lib
%\end{verbatim}
%para establecer las variables de entorno. También puede 
%añadir estas líneas en su fichero \verb|~/.profile|
%
%\end{itemize}
%
%No es buena idea sustituir el intérprete nativo de perl por otra versión.
%La distribución de Linux instalada depende de la versión del intérprete
%Perl. No cambie el \verb|PATH| de manera que el intérprete \verb|perl5.10.1| 
%se ejecute antes que el nativo.
%
%En general, para acceder a la documentación de un módulo use \verb|perldoc| en vez de \verb|man|.
%
%Para ejecutar \verb|perl5.10| escriba el nombre seguido de la versión:
%\begin{verbatim}
%casiano@millo:~$ perl5.10.1 -V
%Summary of my perl5 (revision 5 version 10 subversion 1) configuration:
%
%  Platform:
%    osname=linux, osvers=2.6.24-21-generic, archname=i686-linux-thread-multi
%    uname='linux millo 2.6.24-21-generic #1 smp tue oct 21 23:43:45 utc 2008 i686 gnulinux '
%    config_args='-de -Dprefix=/soft/perl5lib/perl5_10_1/ -Dusethreads'
%    hint=recommended, useposix=true, d_sigaction=define
%    useithreads=define, usemultiplicity=define
%    useperlio=define, d_sfio=undef, uselargefiles=define, usesocks=undef
%    use64bitint=undef, use64bitall=undef, uselongdouble=undef
%    usemymalloc=n, bincompat5005=undef
%  Compiler:
%    cc='cc', ccflags ='-D_REENTRANT -D_GNU_SOURCE -fno-strict-aliasing -pipe -fstack-protector -I/usr/local/include -D_LARGEFILE_SOURCE -D_FILE_OFFSET_BITS=64',
%    optimize='-O2',
%    cppflags='-D_REENTRANT -D_GNU_SOURCE -fno-strict-aliasing -pipe -fstack-protector -I/usr/local/include'
%    ccversion='', gccversion='4.2.4 (Ubuntu 4.2.4-1ubuntu4)', gccosandvers=''
%    intsize=4, longsize=4, ptrsize=4, doublesize=8, byteorder=1234
%    d_longlong=define, longlongsize=8, d_longdbl=define, longdblsize=12
%    ivtype='long', ivsize=4, nvtype='double', nvsize=8, Off_t='off_t', lseeksize=8
%    alignbytes=4, prototype=define
%  Linker and Libraries:
%    ld='cc', ldflags =' -fstack-protector -L/usr/local/lib'
%    libpth=/usr/local/lib /lib /usr/lib
%    libs=-lnsl -ldl -lm -lcrypt -lutil -lpthread -lc
%    perllibs=-lnsl -ldl -lm -lcrypt -lutil -lpthread -lc
%    libc=/lib/libc-2.7.so, so=so, useshrplib=false, libperl=libperl.a
%    gnulibc_version='2.7'
%  Dynamic Linking:
%    dlsrc=dl_dlopen.xs, dlext=so, d_dlsymun=undef, ccdlflags='-Wl,-E'
%    cccdlflags='-fPIC', lddlflags='-shared -O2 -L/usr/local/lib -fstack-protector'
%
%
%Characteristics of this binary (from libperl):
%  Compile-time options: MULTIPLICITY PERL_DONT_CREATE_GVSV
%                        PERL_IMPLICIT_CONTEXT PERL_MALLOC_WRAP USE_ITHREADS
%                        USE_LARGE_FILES USE_PERLIO USE_REENTRANT_API
%  Built under linux
%  Compiled at Aug 30 2009 09:34:30
%  %ENV:
%    PERL5LIB="/soft/perl5lib/share/perl/5.8.8/:/soft/perl5lib/lib/perl/5.8.8:/soft/perl5lib/lib/perl/5.8:/soft/perl5lib/share/perl/5.8/"
%  @INC:
%    /soft/perl5lib/share/perl/5.8.8/
%    /soft/perl5lib/lib/perl/5.8.8
%    /soft/perl5lib/lib/perl/5.8
%    /soft/perl5lib/share/perl/5.8/
%    /soft/perl5lib/perl5_10_1/lib/5.10.1/i686-linux-thread-multi
%    /soft/perl5lib/perl5_10_1/lib/5.10.1
%    /soft/perl5lib/perl5_10_1/lib/site_perl/5.10.1/i686-linux-thread-multi
%    /soft/perl5lib/perl5_10_1/lib/site_perl/5.10.1
%\end{verbatim}
%Para acceder a la documentación relacionada con Perl 5.10 use el script \verb|perl5.10.1.doc|:
%\begin{verbatim}
%casiano@millo:~$ perl5.10.1.doc -h
%perldoc [options] PageName|ModuleName|ProgramName...
%perldoc [options] -f BuiltinFunction
%perldoc [options] -q FAQRegex
%
%Options:
%    -h   Display this help message
%    -V   report version
%    -r   Recursive search (slow)
%    -i   Ignore case
%    -t   Display pod using pod2text instead of pod2man and nroff
%             (-t is the default on win32 unless -n is specified)
%    -u   Display unformatted pod text
%    -m   Display module's file in its entirety
%    -n   Specify replacement for nroff
%    -l   Display the module's file name
%    -F   Arguments are file names, not modules
%    -v   Verbosely describe what's going on
%    -T   Send output to STDOUT without any pager
%    -d output_filename_to_send_to
%    -o output_format_name
%    -M FormatterModuleNameToUse
%    -w formatter_option:option_value
%    -L translation_code   Choose doc translation (if any)
%    -X   use index if present (looks for pod.idx at /soft/perl5lib/perl5_10_1/lib/5.10.1/i686-linux-thread-multi)
%    -q   Search the text of questions (not answers) in perlfaq[1-9]
%
%PageName|ModuleName...
%         is the name of a piece of documentation that you want to look at. You
%         may either give a descriptive name of the page (as in the case of
%         `perlfunc') the name of a module, either like `Term::Info' or like
%         `Term/Info', or the name of a program, like `perldoc'.
%
%BuiltinFunction
%         is the name of a perl function.  Will extract documentation from
%         `perlfunc'.
%
%FAQRegex
%         is a regex. Will search perlfaq[1-9] for and extract any
%         questions that match.
%
%Any switches in the PERLDOC environment variable will be used before the
%command line arguments.  The optional pod index file contains a list of
%filenames, one per line.
%                                                       [Perldoc v3.14_04]
%\end{verbatim}

Visite esta página de vez en cuando. Es posible que añada algún nuevo camino
de búsqueda de librerías y/o ejecutables.

