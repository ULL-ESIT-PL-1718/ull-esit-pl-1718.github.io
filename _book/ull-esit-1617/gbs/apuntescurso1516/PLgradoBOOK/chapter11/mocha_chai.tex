\parrafo{Pruebas: Mocha y Chai}
\label{parrafo:mochaychai}
Mocha is a test framework while Chai is an expectation one. 

Mocha is the \red{simple, flexible, and fun} JavaScript unit-testing framework
that runs in Node.js or in the browser. 

It is open source (MIT licensed),
and we can learn more about it at
\htmladdnormallink{https://github.com/mochajs/mocha}{https://github.com/mochajs/mocha}

Let's say
Mocha setups and describes test suites and Chai provides convenient
helpers to perform all kinds of assertions against your JavaScript code.

\parrafo{Pruebas: Estructura}

Podemos instalar \verb|mocha| globalmente:
\begin{verbatim}
$ npm install -g mocha
\end{verbatim}
pero podemos también añadirlo en \verb|package.json| como una \verb|devDependencies|:
\begin{verbatim}
[/tmp/pl-grado-temperature-converter(karma)]$ head -n 5 package.json 
{
  "dependencies": {},
  "devDependencies": {
    "mocha": "latest"
  },
\end{verbatim}

Y ahora podemos instalar todas las dependencias usando  \verb|npm install|:
\begin{verbatim}
$ npm install
npm http GET https://registry.npmjs.org/mocha
npm http 200 https://registry.npmjs.org/mocha
npm http GET https://registry.npmjs.org/commander/2.3.0
...
\end{verbatim}

En este caso \verb|mocha| es instalado localmente, no globalmente:
\begin{verbatim}
[/tmp/pl-grado-temperature-converter(karma)]$ ls -ltr node_modules/
total 0
drwxr-xr-x  12 casiano  staff  408  5 feb 18:40 mocha
\end{verbatim}

Una vez instalado Mocha, creamos la estructura para las pruebas:

\begin{verbatim}
$ mocha init tests
\end{verbatim}
esto en el caso de que lo hayamos instalado globalmente o bien
\begin{verbatim}
$ node_modules/mocha/bin/mocha init tests
\end{verbatim}
si lo hemos instalado localmente.

\begin{verbatim}
$ tree tests
tests
|-- index.html
|-- mocha.css
|-- mocha.js
`-- tests.js
\end{verbatim}

Añadimos \verb|chai.js|
(Véase 
\htmladdnormallink{http://chaijs.com/guide/installation/}{http://chaijs.com/guide/installation/}) al directorio \verb|tests|.

Chai is a platform-agnostic BDD/TDD assertion library featuring several interfaces 
(for example, should, expect, and assert). 
It is open source (MIT licensed), and we can learn more about it at
\htmladdnormallink{http://chaijs.com/}{http://chaijs.com/}

We can also install Chai on the command line using npm, as follows:
\begin{verbatim}
            npm install chai --save-dev
\end{verbatim}


The latest tagged version will be available for hot-linking at 
\htmladdnormallink{http://chaijs.com/chai.js}{http://chaijs.com/chai.js}.

If you prefer to host yourself, use the \verb|chai.js| file from the root of the 
\htmladdnormallink{github project at https://github.com/chaijs/chai}{https://github.com/chaijs/chai}. 
\begin{verbatim}
[/tmp/pl-grado-temperature-converter(karma)]$ 
$ curl https://raw.githubusercontent.com/chaijs/chai/master/chai.js -o tests/chai.js
  % Total    % Received % Xferd  Average Speed   Time    Time     Time  Current
                                 Dload  Upload   Total   Spent    Left  Speed
100  118k  100  118k    0     0  65521      0  0:00:01  0:00:01 --:--:-- 65500
\end{verbatim}
Ya tenemos nuestro fichero \verb|tests/chai.js|:
\begin{verbatim}
[/tmp/pl-grado-temperature-converter(karma)]$ head tests/chai.js 

;(function(){

/**
 * Require the module at `name`.
 *
 * @param {String} name
 * @return {Object} exports
 * @api public
 */
\end{verbatim}
Quedando el árbol como sigue:
\begin{verbatim}
[~/srcPLgrado/temperature(master)]$ tree tests/
tests/
|-- chai.js
|-- index.html
|-- mocha.css
|-- mocha.js
`-- tests.js

0 directories, 5 files
\end{verbatim}

\parrafo{Pruebas: {\tt index.html}}

Modificamos el fichero \verb|tests/index.html| que fué generado por \verb|mocha init|
para 
\begin{itemize}
\item
Cargar \verb|chai.js|
\item
Cargar \verb|temperature.js|
\item
Usar el estilo \verb|mocha.setup('tdd')|:
\item
Imitar la página \verb|index.html| con los correspondientes \verb|input| y 
\verb|span|:
\begin{verbatim}
    <input id="original" placeholder="32F" size="50">
    <span class="output" id="converted"></span>
\end{verbatim}
\end{itemize}
quedando así:
\begin{verbatim}
[~/srcPLgrado/temperature(master)]$ cat tests/index.html 
<!DOCTYPE html>
<html>
  <head>
    <title>Mocha</title>
    <meta http-equiv="Content-Type" content="text/html; charset=UTF-8">
    <meta name="viewport" content="width=device-width, initial-scale=1.0">
    <link rel="stylesheet" href="mocha.css" />
  </head>
  <body>
    <div id="mocha"></div>
    <input id="original" placeholder="32F" size="50">
    <span class="output" id="converted"></span>

    <script src="chai.js"></script>
    <script src="mocha.js"></script>
    <script src="../temperature.js"></script>
    <script>mocha.setup('tdd')</script>
    <script src="tests.js"></script>

    <script>
      mocha.run();
    </script>
  </body>
</html>
\end{verbatim}

\parrafo{Pruebas: Añadir los tests}

The "TDD" interface provides 
\begin{itemize}
\item \verb|suite()|
\item  \verb|test()|
\item  \verb|setup()|
\item  \verb|teardown()|.
\end{itemize}

\begin{verbatim}
[~/srcPLgrado/temperature(master)]$ cat tests/tests.js 
var assert = chai.assert;

suite('temperature', function() {
    test('32F = 0C', function() {
        original.value = "32F";
        calculate();
        assert.deepEqual(converted.innerHTML, "0.0 Celsius");
    });
    test('45C = 113.0 Farenheit', function() {
        original.value = "45C";
        calculate();
        assert.deepEqual(converted.innerHTML, "113.0 Farenheit");
    });
    test('5X = error', function() {
        original.value = "5X";
        calculate();
        assert.match(converted.innerHTML, /ERROR/);
    });
});
\end{verbatim}

The \cei{BDD} interface provides \verb|describe()|, \verb|it()|, \verb|before()|, \verb|after()|, \verb|beforeEach()|, and \verb|afterEach()|:

\begin{verbatim}
describe('Array', function(){
  before(function(){
    // ...
  });

  describe('#indexOf()', function(){
    it('should return -1 when not present', function(){
      [1,2,3].indexOf(4).should.equal(-1);
    });
  });
});
\end{verbatim}
The \cei{Chai should} style allows for the same chainable assertions as the 
\cei{expect interface}, however it extends each object with a \verb|should| 
property to start your chain. 

\parrafo{Chai Assert Style}

The \cei{assert style} is exposed through assert interface. 

This provides the classic assert-dot notation, similiar to that packaged with node.js. 

This \verb|assert| module, however, provides several additional tests and is browser compatible.

\begin{verbatim}
var assert = require('chai').assert
  , foo = 'bar'
  , beverages = { tea: [ 'chai', 'matcha', 'oolong' ] };

assert.typeOf(foo, 'string', 'foo is a string');
assert.equal(foo, 'bar', 'foo equal `bar`');
assert.lengthOf(foo, 3, 'foo`s value has a length of 3');
assert.lengthOf(beverages.tea, 3, 'beverages has 3 types of tea');
\end{verbatim}
In all cases, the assert style allows you to include an optional message as the last parameter in the assert statement. 

These will be included in the error messages should your assertion not pass.

\parrafo{Assert API, Expect/Should API}

\begin{itemize}
\item
Here  is the documentation of the 
\htmladdnormallink{Assert API}{http://chaijs.com/api/assert/}.


\item
Here  is the documentation of the 
\htmladdnormallink{Should/Expect API}{http://chaijs.com/api/bdd/}.
\end{itemize}

\parrafo{Chai Expect Style}

The BDD style is exposed through expect or should interfaces. In both scenarios, you chain together natural language assertions.

\begin{verbatim}
var expect = require('chai').expect
  , foo = 'bar'
  , beverages = { tea: [ 'chai', 'matcha', 'oolong' ] };

expect(foo).to.be.a('string');
expect(foo).to.equal('bar');
expect(foo).to.have.length(3);
expect(beverages).to.have.property('tea').with.length(3);
\end{verbatim}
Expect also allows you to include arbitrary messages to prepend to any failed assertions that might occur.

\begin{verbatim}
var answer = 43;
\end{verbatim}

\begin{verbatim}
// AssertionError: expected 43 to equal 42.
expect(answer).to.equal(42); 
\end{verbatim}

\begin{verbatim}
// AssertionError: topic [answer]: expected 43 to equal 42.
expect(answer, 'topic [answer]').to.equal(42);
\end{verbatim}
This comes in handy when being used with non-descript topics such as booleans or numbers.

\parrafo{Ejecución Simple}
Ahora podemos ejecutar las pruebas abriendo en el navegador el
fichero \verb|tests/index.html|:
\begin{verbatim}
$ open tests/index.html 
\end{verbatim}

Esta información aparece también en las secciones {\it Unit Testing: Mocha}
\ref{subsection:mocha} de
estos apuntes.


