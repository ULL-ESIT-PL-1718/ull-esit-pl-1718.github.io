\item 
\begin{enumerate}
\item 
Complete las partes que faltan de esta calculadora escrita en Jison:
\begin{verbatim}
%___
%%

\s+                   /* skip whitespace */
[0-9]+("."[0-9]+)?\b  return 'NUMBER'
"*"                   return '*'
"/"                   return '/'
"-"                   return '-'
"+"                   return '+'
"^"                   return '^'
"!"                   return '!'
"%"                   return '%'
"("                   return '('
")"                   return ')'
"PI"                  return 'PI'
"E"                   return 'E'
<<EOF>>               return 'EOF'
.                     return 'INVALID'

/___

%left _______
%left _______
%____'^'
%_____    '!'
%______   '%'
%left UMINUS

%start ___________

%% /* language grammar */

expressions
    : e EOF
        { typeof console !== 'undefined' ? console.log($1) : print($1);
          return $1; }
    ;

e
    : e '+' e
        {___________}
    | e '-' e
        {___________}
    | e '*' e
        {___________}
    | e '/' e
        {___________}
    | e '^' e
        {__ = Math.pow(______);}
    | e '!'
        {{
          $$ = (function fact (n) { return n==0 ? _ : _____________ })($1);
        }}
    | e '%'
        {$$ = $1/100;}
    | '-' e %____ ______
        {_________}
    | '(' e ')'
        {________}
    | NUMBER
        {____________________}
    | E
        {$$ = Math.E;}
    | PI
        {$$ = Math.PI;}
    ;
\end{verbatim}
\item 
¿Cómo  habría que modificar la calculadora  para que las frases de
la forma \verb|4-5-8| se interpretaran como \verb|4-(5-8)|?
\item 
¿Cómo  habría que modificar la calculadora  para que las frases de
la forma \verb|4-5*8| se interpretaran como \verb|(4-5)*8)|?
\item 
¿Cómo  habría que modificar la calculadora  para que las frases de
la forma \verb|-5*8| se interpretaran como \verb|-(5*8)|?
\item 
¿Cómo habría que modificar el analizador léxico para que se admitieran 
comentarios tipo javascript \verb|/* ... */|?
\item 
¿Con que comando compilamos la gramática Jison anterior para producir el parser?
\end{enumerate}
\item 
Complete el algoritmo de análisis LR.
\begin{itemize}
\item
$|x|$ denota la longitud de la cadena $x$. 
\item
La función \verb|top(k)| devuelve el elemento que ocupa la 
posición \verb|k| contando desde el \emph{top} de la pila
\item
La función \verb|pop(k)| extrae \verb|k| elementos de la pila.
\item
La notación \verb|state.attr| hace referencia al atributo
asociado con cada estado.
\item Denotamos por \verb|Sem| el hash de acciones semánticas
\end{itemize}
\begin{verbatim}
 push(__);
 b = yylex();
 for( ; ; ;) {
   s = top(0); a = b;
   switch (action[_][_]) {
     case "shift t" : 
       t.attr = _.____;
       push(_); 
       b = _______;
       break;
     case "reduce A ->alpha" : 
       eval(Sem{_ __ _____}(top(|alpha|-1).attr, ... , top(0).attr)); 
       pop(|_____|); 
       push(goto[______][_]); 
       break;
     case "accept" : return (1); 
     default : yyerror("syntax error");
   }
 }
\end{verbatim}
\item 
\begin{enumerate}
\item 
Escriba un autómata finito no determinista que reconozca el lenguaje 
de los prefijos viables de la gramática:
\begin{verbatim}
e: e '-' e | NUM ;
\end{verbatim}
\item 
Usando la construcción del subconjunto encuentre un autómata finito determinista que sea equivalente al construído en el apartado anterior
\item
Encuentre el conjunto de terminales que puede aparecer a continuación de 
\verb|e| en una derivación de la gramática. No se olvide de considerar el terminale \verb|END-OF-INPUT|. Use el carácter \verb|'$'| para representarlo.
\item
Construya la tabla de acciones (\verb|action|) del analizador SLR para esta gramática
\end{enumerate}

\item 
Complete las partes que faltan del fichero \verb|auth.rb| que permite 
autentificación usando OAuth mediante la gema \verb|omniauth|:
%use provider omniauth.auth name image
\begin{verbatim}
require 'omniauth-oauth2'
require 'omniauth-google-oauth2'

___ OmniAuth::Builder do
  config = YAML.load_file 'config/config.yml'
  ________ :google_oauth2, config['identifier'], config['secret']
end

get '/auth/:name/callback' do
  session[:auth]  = @auth = request.env['________.____']
  session[:name]  = @auth['info'].____
  session[:image] = @auth['info']._____
  flash[:notice] = 
      %Q{<div class="success">Authenticated as #{session[:name]}.</div>}
  redirect '/'
end

get '/auth/failure' do
  flash[:notice] = params[:message] 
  redirect '/'
end
\end{verbatim}

\item 
Complete el código para el análisis de ámbito en esta extensión de la 
calculadora que admite funciones anidadas.

En primer lugar tenemos las estructuras de datos necesarias:
\begin{verbatim}
%{
var symbolTables = [{ name: '', father: null, vars: {} }];
var scope = 0; 
var symbolTable = symbolTables[scope];
\end{verbatim}

\begin{enumerate}
\item  Complete este accessor para \verb|scope|:
\begin{verbatim}
function getScope() {
  return _____;
}
\end{verbatim}

\item  La función \verb|getFormerScope| se llama cuando se sale de un ámbito:
\begin{verbatim}
function getFormerScope() {
   _______;
   symbolTable = symbolTables[_____];
}
\end{verbatim}
\item  La función \verb|makeNewScope| se llama cada vez que se entra en un nuevo ámbito:
\begin{verbatim}
function makeNewScope(id) {
   _______;
   symbolTable.vars[id].symbolTable = symbolTables[_____] =  
                 { name: id, father: symbolTable, vars: {} };
   symbolTable = symbolTables[_____];
   return symbolTable;
}
\end{verbatim}
\item 
Necesitamos una función \verb|findSymbol| para encontrar donde está
la definición del objeto cuyo nombre es \verb|x|:
\begin{verbatim}
function findSymbol(x) {
  var f;
  var s = scope;
  do {
    f = symbolTables[s].vars[x];
    ___;
  } while (______ && !f);
  s++;
  return [f, s];
}
\end{verbatim}
\item 
Cuando se detecta una llamada se hacen comprobaciones 
en la tabla de símbolos:
\begin{verbatim}
function functionCall(name, arglist) {
  var info = findSymbol(name);
  var s = info[1];
  info = info[0];

  if (!info || info.type != 'FUNC') {
    _________________________________________;
  }
  else if(arglist.length != info.arity) {
    __________________________________________;
  }
  ....
}
\end{verbatim}

\item  Complete el código que falta en la acción
semántica asociada con la definición de una función:
\begin{verbatim}
dec 
    : DEF functionname  optparameters "{" decs statements "}" 
                  { 
                     ______________();

                     $$ = translateFunction(...);
                  }
    | VAR varlist ';'   { ... }
    ;
\end{verbatim}
\item 
En la produción anterior aparecía \verb|functionname| el cual produce simplemente un identificador:
\begin{verbatim}
functionname
    : ID 
                  {
                     if (symbolTable.vars[$ID]) 
                       throw ___________________________________________;
                     symbolTable.vars[$ID] = { type: ______, name: ___ };

                     ____________($ID);

                     $$ = $ID;
                  }
    ;
\end{verbatim}
¿Por qué se crea esta regla de producción? 
Rellene el código que falta.
\item 
Cuandos se analiza la definición de parámetros es preciso 
crear la entrada y guardar la información relevante. Complete 
el código:
\begin{verbatim}
parameters
    : ID                { 
                          symbolTable.vars[___] = { type : '_____' };
                          $$ = [ $ID ]; 
                        }
    | parameters "," ID { 
                          symbolTable.vars[___] = { type : '_____' };
                          $$ = $1; 
                          $$.push($ID); 
                        }
    ;
\end{verbatim}
\item  Cuando se usa un identificador hay que comprobar que su 
uso es conforme a su definición:
\begin{verbatim}
e
    : ID "=" e
        { 
           var info = findSymbol($ID);
           var s = info[1];
           info = info[0];

           if (info && info.type === 'VAR') { 
             $$ = binary($e,unary("&"+$ID+", "+(getScope()-s)), "=");
           }
           else if (info && info.type === 'PAR') { 
             $$ = binary($e,unary("&$"+$ID+", "+(getScope()-s)), "=");
           }
           else if (info && info.type === 'FUNC') { 
              ______________________________________________________;
           }
           else { 
              ______________________________________________;
           }
        }
    | ....
\end{verbatim}
\end{enumerate}
\item Responda a las siguientes preguntas sobre transformaciones árbol:
\begin{enumerate}
\item Defina alfabeto con aridad
\item Defina el conjunto de todos los árboles sobre un alfabeto
\item Defina gramática árbol
\item Defina lenguaje generado por una gramática árbol
\item ¿Que es un patrón árbol? ¿Que significa que un árbol casa con un patrón árbol?
\item Defina que es un esquema de transformaciones árbol
\item Escriba en pseudocódigo un esquema de transformaciones árbol 
para el plegado de constantes
\end{enumerate}


\item 
Calcule los conjuntos \verb|FIRST| y \verb|FOLLOW| para las variables
sintácticas de esta gramática:
\begin{verbatim}
S : A B 'a' A | 'b' S B
  ;
A : /* vacío */ | A 'a' 
  ;
B : /* vacío */ | B 'b'
  ;
\end{verbatim}

\item 
Escriba un traductor de expresiones en infijo a postfijo usando Jison.
Una expresión como \verb|a+2*b| deberá traducirse como
\verb|a 2 b * +|.

\item 
Explique que código se debería añadir al
ejercicio anterior para traducir una expresion
\verb|if expression then expression else expression|.

\item 
Haga un diagrama de un stackframe de una llamada explicando
las diferentes partes para un lenguaje que admite procedimientos 
anidados.
